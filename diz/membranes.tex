Recently, an interdisciplinary research between the fields of Computer Science and Biology has been rapidly growing. 

% Bioinformatics (slaves of biologists)

Bioinformatic has undergone a fast evolving process, especially the areas of genomics and proteomics. Bioinformatics can be seen as the application of computing tools and techniques for the management of biological data. Just to mention a few, the design of efficient algorithms for sequence alignment, the investigation of methods for prediction of the 3D structure of molecules and proteins and the development of data structures to effectively store huge amount of structured data.

% Natural computing (those inspired by nature)

On the other hand, the birth of biologically inspired frameworks started the investigation of mathematical models and their properties and technological requirements for their implementation by biological hardware.
Those frameworks are inspired by the nature in the way it ``computes'', and has gone through the evolution for billions of years.

Neural networks, genetic algorithms and DNA computing are already well established research fields.

However, nature computes not only at the neural or genetic level, but also at the cellular level. In general, any non-trivial biological system has a hierarchical structure where objects and information flows between regions, what can be interpreted as a computation process.

% The notion of membrane

The regions are typically delimited by various types of membranes at different levels from cell membranes, through skin membrane to virtual membranes which delimits different parts of an ecosystem.
This hierarchical system can be seen in other field such as distributed computing, where again well delimited computing units coexist and are hierarchically arranged in complex systems from single processors to the internet.

Membranes keep together certain chemicals or information and selectively determines which of them may pass through.

% The notion of membrane structure

From these observations, Paun \cite{Paun98} introduces the notion of a membrane structure as a mathematical representation of hierarchical architectures composed of membranes. It is usually represented as a Venn diagram with all the considered sets being subsets of a unique set and not allowed to be intersected. Every two sets are either one the subset of the other, or disjoint. Outermost membrane (also called skin membrane) delimits the finite ``inside'' and the infinite ``outside''.

% Membrane models

Recently, several computational models based on the membrane structure have been created.

% P systems

P systems were introduced in 1998 as a system with membrane structure that contains multisets of objects and rewrite rules that are executed in a maximaly parallel manner. Since then, a huge amount of variants have been created with various computation powers. We have investigated the power of several such variants. Additionaly, we propose a new variant with a vacuum, where the rules can describe what will happen to an object that arrives to an empty membrane.

P systems with existing and newly proposed variants will be discussed in the chapter \ref{cha:p_systems}.

Aside from P systems, other models based on the membrane structure have been created such as the Calculus of Looping Sequences (CLS), which was inspired by P systems.

% CLS

\section{The Calculus of Looping Sequences} % (fold)
\label{sec:calculus_of_looping_sequences}

Barbuti in \cite{Barbuti07CLS} concluded that there is a need for a formalism having a simple notation, having the ability of describing biological systems at different levels of abstractions, having some notions of compositionality and being flexible enough to allow describing new kinds of phenomena as they are discovered, without being specialized to the description of a particular class of systems. The Calculus of Looping Sequences (CLS) was introduced in \cite{Barbuti07CLS}.

The membrane structure in CLS is defined recursively, consisting of terms $T$ and sequences $S$:
$$ T ::= S \mid (S)^L\rfloor T \mid T|T$$
$$ S ::= \eps \mid a \mid S\cdot S $$
, where $a$ is a symbol from the alphabet, $\eps$ represents the empty sequence, $\cdot$ is a sequencing operator, $(S)^L$ is a looping operator, $|$ is a parallel composition operator and $\rfloor$ is a containment operator.

Membranes are represented by a sequence that is looped around.
Several extensions of the CLS have been proposed. In CLS+ the looping operator can be applied to a parallel composition of sequences, which represents a notion of a fluid membrane.
CLS+ can be translated to CLS (see \cite{Barbuti07CLS}).
Milazzo in his PhD thesis \cite{Milazzo07CLS} includes also a simulation of a P system using CLS. The major difficulty is simulating the maximal parallelism of rule application.

% section calculus_of_looping_sequences (end)