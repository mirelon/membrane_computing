% !TEX root = diz.tex
\chapter*{Conclusions}
\addcontentsline{toc}{chapter}{Conclusions}
We have studied several variants of sequential P systems in order to obtain universality without using maximal parallelism. A variant with rewriting rules that can use inhibitors was shown to be universal in both generating and accepting case. The generating model is able to simulate maximal parallel P system and the accepting model can simulate a register machine.

In addition, we have defined a new notion of vacuum, which is immediately created in the region that becomes empty. A new P system variant was introduced, which allowed the vacuum to be used on the left side of P system rewriting rules. The accepting case of this variant was shown to be universal by direct implementation of a register machine.

A lot of other variants deserve to be combined with the vacuum, so we suggest to research them more
thoroughly(non-cooperative rules, rules with priorities, decaying objects, deterministic steps, . . . ).