% !TEX root = diz.tex
\chapter*{Conclusions}
\addcontentsline{toc}{chapter}{Conclusions}
We have studied several variants of sequential P systems in order to obtain universality without using maximal parallelism. A variant with rewriting rules that can use inhibitors was shown to be universal in both generating and accepting case. The generating model is able to simulate maximal parallel P system and the accepting model can simulate a register machine.
The constructive proof for the generating case is valuable not only for the universality, but also can be seen as a method of conversion between P systems in sequential manner and maximally parallel manner, which may be essential for future works on P systems and other multiset rewriting systems. The simulation also shows a method how to synchronize application of multiple rules in a membrane and how to synchronize this parallel rule application across whole membrane structure. Sequential variants are promising alternative to traditional maximal parallel variants and will be good subject for the further research. Future plans include research of other more restricted variants such as omitting cooperation in the rules or restricting the power of inhibitors.

In addition, we have defined a new variants of zero-testing, aiming to fit in layers between mere reformulations of the basic sequential P system and universal sequential P systems with inhibitors and possibly to reveal some unexpected connection with other models of computation. We studies variants with various forms of detection of empty membranes - a notion specific for membrane systems. The results obtained have been just the computational completeness. However, one variant with objects avoiding empty regions is more promising for our goal because the standard contruction of register machine do not work. We conjecture this variant is not universal, possibly equivalent with Petri nets or other model of computation weaker than Turing machine.

There are many features not yet combined, so we suggest them for the further research (non-cooperative rules, rules with priorities, decaying objects, deterministic steps, \ldots).

Aside from the research of the computational power, there are many open problems in the area of decision problems of certain properties. Interesting ideas for future work can be taken from \cite{Bottoni06Inhibitors}. They define an abstract notion of negative application conditions for general rewriting systems, which is for multiset rewriting rendered as the usage of inhibitors. Although they considered only nondeleting rules (after application of each rule the resulting multiset is a superset of the current multiset), interesting results were shown that the termination of rewriting was shown to be decidable.

We have investigated the decidability problems of existence of (in)finite computation for a universal class of P systems with active membranes. We have shown and published our results that are on both sides of the decidability barrier. Regarding the open problem stated in \cite{Ibarra05Active} about sequential active P systems with hard membranes (without communication between membranes), it could be interesting to find a connection between the universality and decidability of these termination problems.

We research sequential P systems with active membranes also in combination with notions inspired by reaction systems. Variants using sets instead of multisets are shown to be computational complete. We have provided a proof by a simulation of a register machine. We have proposed alternative definitions for membrane creation: inject-or-create and wrap-or-create. In either case the resulting system has been shown to be universal. We suggest investigating of decidability properties of these models as well as other inspirations from reaction systems, e. g. non-permanency of objects. There are no results yet in this area and our proposals could be set as a single topic for the future study.

