% !TEX root = ../diz.tex
Although P systems were inspired by the functioning of a biological cell, the applications has already far exceeded the scope of a cell. P systems are already being used in many problems ranging from modeling biological processes and population dynamics through social network problems to image processing and effectively solving hard problems.

\subsection{MeCoSym} % (fold)
\label{sub:mecosym}

There is an online collection of various case studies on the MeCoSim webpage \cite{MeCoSimWeb}. The number of scenarios and the level of detail vary among the different case studies, possibly including detailed descriptions, references to related publications, charts and videos.

MeCoSim (Membrane Computing Simulator) \cite{Perez10MeCoSim} is a software that offers the users a General Purpose Application to model, design, simulate, analyze and verify different types of models based on P systems. Some of the main features of MeCoSim are the following:
\begin{itemize}
  \item Simulation of models of P systems under different initial conditions. It enables the load of P-Lingua based models, parsing, edition, debugging, and different simulation types.
  \item Visualization capabilities for analyzing P systems: alphabet, membrane structure, multisets and graphs viewers.
  \item Highly customizable platform for defining inputs, outputs, parameters and graphs for each model of a family of P systems.
  \item Repositories system for the visual management of available online resources, including plugins, custom applications, models and scenarios.
  \item Export option for releasing end-user applications for solving practical problems, abstracting P system functionalities.
  \item Plugins architecture, permitting the extension of the functionality with Java jars or external non-Java programs. MeCoSim is written in Java.
  \item Auto-update capability, using the latest release of the program whenever it runs.
\end{itemize}

MeCoSim development started in 2010, and a number of case studies have been analyzed since then, covering many variants of P systems, different areas of interest and application domains. The examples contain, at least, some snippets of code, along with the minimal needed files to run some scenario in MeCoSim.

% subsection mecosym (end)

\subsection{Population dynamics} % (fold)
\label{sub:population_dynamics}

The Bearded Vulture (Gypaetus barbatus) is an endangered species in Europe that feeds almost exclusively on bone remains of wild and domestic ungulates. Spanish researchers in \cite{Cardona:2009:Vultures} presented a model of an ecosystem related to the Bearded Vulture in the Pyrenees (NE Spain), by using P systems. The evolution of six species is studied: the Bearded Vulture and five subfamilies of domestic and wild ungulates upon which the vulture feeds. P systems provide a high level computational modeling framework which integrates the structural and dynamic aspects of ecosystems in a comprehensive and relevant way. P systems explicitly represent the discrete character of the components of an ecosystem by using rewriting rules on multisets of objects which represent individuals of the population and bones. The inherent stochasticity and uncertainty in ecosystems is captured by using probabilistic strategies.

This research was extended in \cite{Cardona11Zebra} where they also proposed model for exotic invasive species of Zebra Mussel for the design of a plan for the management of the reservoir of Ribarroja in the area of Ebro River basin because this species create significant environmental problems and economic costs due to its ability to block all types of infrastructure and water pipes.

% subsection population_dynamics (end)

\subsection{Cellular Signalling Pathways} % (fold)
\label{sub:cellular_signalling_pathways}

Perez in \cite{Perez06EGFR} proposed a model for EGFR Signalling Cascade. More than 60 proteins were included and 160 chemical reactions. Membrane structure consists of 3 regions: the environment, the cell surface and the cytoplasm. 

% subsection cellular_signalling_pathways (end)

\subsection{Solving SAT in linear time} % (fold)
\label{sub:solving_sat_in_linear_time}


Polynomial time solutions to NP-complete problems by means of P systems are achieved by trading time (number of computation steps) for space (number of membranes and objects). This is inspired by the capabililty of cells to produce an exponential number of new membranes in polynomial time.

However, many simulators of P system are inefficient since they cannot handle the parallelism of these devices. Nowadays, we are witnessing the consolidation of the GPUs as a parallel framework to compute general purpose applications such as bitcoin mining. The simulation of P systems with active membranes using GPUs is analysed in \cite{Cecilia10SAT} and an efficient linear solution to the SAT problem is illustrated.

They compared it to the classical simulator and reported up to 94x of speedup for 256 literals in the formula. The major constraint of this parallel simulation is the GPU memory size, which can be overcome with a data partition on a cluster of GPUs. 

% subsection solving_sat_in_linear_time (end)

\subsection{Implementation of P systems in vitro} % (fold)
\label{sub:implementation_of_p_systems_in_vitro}

``In vitro'' are studies in experimental biology that uses components of an organism that have been isolated from their usual biological surroundings in order ot provide a more convenient analysis.

% TODO: citation needed Research Topics Arising from the (Planned) P Systems Implementation Experiment in Technion
There was a planned experiment of computing the Fibonacci sequence using P systems in vitro (see \cite{Gershoni:2008:InVitro}) using test tubes as membranes and DNA molecules as objects, evolving under the control of enzymes.
Number of objects in a multiset was represented by a pre-defined ``mole'' of the substance and synchronization was obtained by ``waiting enough'', such that all reactions that can take place in a test tube actually take place. Hence, the variant where reactions don't cycle is required (Local loop-free P systems).
Communication was done by moving all the relevant objects to the next tube in a mechanical way. The final result was read by spectrometry.

The proposed framework has many difficulties. The notable ones are:

\begin{itemize}
  \item Is there any chance to solve NP-complete problems in this frameworks?
  \item Are LL-free P systems universal?
\end{itemize}


% subsection implementation_of_p_systems_in_vitro (end)