% !TEX root = ../diz.tex
Besozzi in his PhD thesis (see \cite{Besozzi:PhD:2004}) formulates three criteria that a good P system variant should satisfy:

\begin{enumerate}
	\item It should be as much realistic as possible from the biological point of view, in order not to widen the distance between the inspiring cellular reality and the idealized theory.
	\item It should result in computational completeness and efficiency, which would mean to obtain universal (and hence, programmable) computing devices, with a powerful and useful intrinsic parallelism.
	\item It should present mathematical minimality and elegance, to the aim of proposing an alternative framework for the analysis of computational models.
\end{enumerate}

In membrane computing, many models are equal in power with Turing machines. To be correct, we should speak about Turing completeness (or computational completeness). Because the proofs are always constructive, using the constructions from these proofs on a universal Turing machine or from a equivalent device, we obtain universal P systems, which are able to simulate any other P system of the given type. That is why we often speak about universality results, and not about computational completeness.

\subsection{Accepting vs generating} % (fold)
\label{sub:accepting_vs_generating}

In the Chomsky hierarchy, there are language acceptors (finite automata, Turing machines) and language generators (formal grammars).

% Accepting grammars

Bordhin in \cite{Bordihn99acceptingpure} extends grammars to allow for accepting languages by interchanging the left side with the right side of a rule. The mode will apply rewriting rules to an input word and accept it when it reaches the starting nonterminal. However, the input word consists of terminal symbols, which could not be rewritten when using original definition, hence they consider the pure version of various grammar types where they give up the distinction between terminal and nonterminal symbols.

% Accepting vs generating common results

The regular, context-free, context-sensitive and recursively enumerable languages were shown to have equal power in accepting and generating mode.
Some other grammars (programmed grammars with appearance checking) are shown to be more powerful in accepting mode than in generating mode.
For deterministic Lindenmayer systems, the generating and accepting mode are incomparable.

% Definition of accepting P system

It can be interesting to investigate accepting and generating mode also in P system variants. The original P system from section \ref{sec:definitions} defines the language as a generator. The accepting mode is usually defined as in \cite{Besozzi:PhD:2004}. A fixed membrane is specified as the input membrane. The system is said to accept a multiset (the initial contents of the input membrane), a number (the size of the input membrane) or a relation (various combinations of the occurrences of certain symbols in the input membrane) iff there is a computation that reaches a given halting configuration (or just a configuration).

% Accepting vs generating P system results

In many variants the generating and accepting mode are equivalent. Barbuti in \cite{Barbuti:2010:AcceptingGenerating} showed some variants where the P system in accepting mode is strictly stronger than in generating mode.

% subsection accepting_vs_generating (end)

\subsection{Active vs passive membranes} % (fold)
\label{sub:active_vs_passive_membranes}

Most of the studied P system variants assume that the number of membranes can only decrease during a computation, by dissolving membranes as a result of applying evolution rules to the objects present in the system \cite{Paun99ActiveMembranes}.
A natural possibility is to let the number of membranes also to increase during a computation, for instance, by division, as it is well-known in biology. Actually, the membranes from biochemistry are not at all passive, like those in the models briefly described above.
For example, the passing of a chemical compound through a membrane is often done by a direct interaction with the membrane itself (with the so-called protein channels or protein gates present in the membrane); during this interaction, the chemical compound which passes through membrane can be modified, while the membrane itself can in this way be modified (at least locally).

In \cite{Paun99ActiveMembranes} P\u{a}un considers P systems with active membranes where the central role in the computation is played by the membranes: evolution rules are associated both with objects and membranes, while the communication through membranes is performed with the direct participation of the membranes; moreover, the membranes can not only be dissolved, but they also can multiply by division. An elementary membrane can be divided by means of an interaction with an object from that membrane.

There are also other ways to define the creation of new membranes. In section \ref{sec:active_membranes} we present a variant with a special evolution rule which creates new membrane when a given multiset ob objects is present in the region. 

% Polarization

\subsubsection{P systems with polarized membranes} % (fold)
\label{ssub:p_systems_with_polarized_membranes}

Polarizing membranes \cite{Paun99ActiveMembranes} is a special extension of active P systems, where each membrane is supposed to have an electrical polarization (we will say charge), one of the three possible: positive, negative, or neutral. If in a membrane we have two immediately lower membranes of opposite polarizations, one positive and one negative, then that membrane can also divide in such a way that the two membranes of opposite charge are separated; all membranes of neutral charge and all objects are duplicated and a copy of each of them is introduced in each of the two new membranes.
The skin membrane is never divided.
If at the same time a membrane is divided and there are objects in this membrane which are being rewritten in the same step, then in the new copies of the membrane the result of the evolution is included.

In this way, the number of membranes can grow, even exponentially. As expected, by making use of this increased parallelism we can compute faster.
For example, the SAT problem, which is NP complete, can be solved in linear time, when we consider the steps of computation as the time units.
Moreover, the model is shown to be computationally universal.

% subsubsection p_systems_with_polarized_membranes (end)

% subsection active_vs_passive_membranes (end)

\subsection{Cooperative vs non-cooperative} % (fold)
\label{sub:context_in_rules}

% Cooperative / Non-cooperative

If a P system contains evolution rules whose radius is greater than 1, then it is said to be cooperative, otherwise it is a non-cooperative system. Non-cooperative P systems have all rules restricted to contain only one object on the left side.
P systems with cooperative rules are universal \cite{Paun98}, while P systems with non-cooperative rules only characterize Parikh image of context-free languages ($PsCF$) \cite{Sburlan05dragos}. It seems to relate with context-free and context-sensitive grammars in Chomsky's hierarchy mentioned in section \ref{sec:chomsky_hierarchy}. 

% Catalytic P systems

P\u{a}un \cite{Paun98} also defines catalytic P systems where catalysts are a specified subset of the alphabet $C\subseteq \Sigma$. In a catalytic system the only possible rule of radius greater than 1 is of the form $ca\rightarrow cv$, where $c\in C, a\in \Sigma\setminus C$ and $v\in (\Sigma\setminus C)^*$ and that is the only form of a rule that can contain a catalyst. If $C=\emptyset$ then the system is called non-catalytic. Catalysts are not modified by applying the catalytic rule. Surprisingly, P systems with catalytic rules are universal, even two membranes in the P system are sufficient to achieve universality.

% Two catalysts

Freund in \cite{Freund2005TwoCatalysts} showed that two catalysts and one membrane are enough for universality and raised an open problem whether one catalyst is sufficient. He conjectured that for computationally universal P systems the results obtained in this paper are optimal not only with respect to the number of membranes (2), but also with respect to the number of catalysts.

% Catalysts are too powerful

From some point of view, catalysts are way too powerful in restricting the parallelism - they directly participate in the rules, hence the number of catalytic rules that can be applied in one step, is bounded by number of catalysts.

A variant with promoters and inhibitors have been proposed (see \cite{Ionescu:jucs_10_5:on_p_systems_with}).

% Promoters

In the case of promoters, the rules are possible only in the presence of certain symbols. An object $p$ is a promoter for a rule $u\rightarrow v$ and we denote this by $u\rightarrow v|_{p}$, if the rule is applicable only in the presence of object $p$. Note that unlike in the case with catalysts, promoters allow the associated rules to be applied as many times as possible.

% Inhibitors

If an object $i$ is an inhibitor for a rule $u\rightarrow v$, we denote this by $u\rightarrow v|_{\neg i}$. Such rule is applicable as long as inhibitor $i$ is not present in the region.

% One catalyst with promoters / inhibitors

Ionescu in \cite{Ionescu:jucs_10_5:on_p_systems_with} showed that P systems with non-cooperative catalytic rules with only one catalyst and with promoters / inhibitors are universal.
One of our results (see section \ref{sec:inhibitors}) uses inhibitors to achieve universality for sequential P systems.

% Zero catalysts with inhibitors

Non-cooperative rules with no catalysts and with inhibitors were studied in \cite{Sburlan:2006:FurtherResultsPromotersInhibitors}, the equivalence with the Parikh image of Lindenmayer systems ($ET0L$ as defined in section \ref{sec:lindenmayer_systems}) was proved.

% Simple cooperative system

Dang \cite{Ibarra04dang} proposes a simple cooperative system ($SCO$) as a P system where the only rules allowed are of the form $a\rightarrow v$ or of the form $aa\rightarrow v$, where $a$ is an object and $v$ is a multiset of objects not containing $a$. This variant is investigated with various modes of parallelism, so their results will be mentioned in the subsection \ref{sub:parallelism_options}

% Purely catalytic systems

Another interesting variant is where the only rules are the catalytic rules of the form $ca\rightarrow cv$, so there are no rules of radius 1. This variant is called purely catalytic systems and was introduced by Ibarra in \cite{Ibarra:03:Catalytic}. The computational completeness was shown to hold with just three catalysts \cite{Freund2005TwoCatalysts}, while when initialized with just one catalyst, the equivalence with Petri nets has been proven \cite{Ibarra04Catalytic}.

% subsection context_in_rules (end)

\subsection{Rules with priorities} % (fold)
\label{sub:rules_with_priorities}

In the original definition of a P system \cite{Paun98}, a partial order relation over set of rewriting rules have been specified. The rule can be used only if no rule of a higher priority in the region can be applied at the same time.

Sos\'ik in \cite{Sosik:2002:WithoutPriorities} showed that the priorities may be omitted from the maximal parallel model without loss of computational power, thus maintaining universality.

Sequential P systems (see subsection \ref{sub:parallelism_options}) with prioritized rules are universal \cite{Ibarra05Active}, while without the priorities they are not.

As mentioned in subsection \ref{sub:context_in_rules}, P systems with non-cooperative rules only characterize the Parikh image of context-free languages ($PsCF$) \cite{Sburlan05dragos}, it was an open problem, whether priorities could improve the computational power. In \cite{Sburlan05Priorities} Sburnan showed that the variant with prioritized rules is more powerful and characterize exactly the Parikh image of $ET0L$ (see section \ref{sec:lindenmayer_systems} and subsection \ref{sec:parikh_s_mapping} in preliminaries).

We may conclude here that adding a priority relation over rules is a powerful feature which often boosts the computational power by allowing finer control mechanisms and limiting otherwise uncontrolled nondeterministic selection of rules to be applied.

% subsection rules_with_priorities (end)

\subsection{Energy in P systems} % (fold)
\label{sub:energy_in_p_systems}

Various notions of energy has been proposed for use in P systems. P\u{a}un in \cite{Paun:2001:Energy} considers a P system where each evolution rule ``produces'' or ``consumes'' some quantity of energy, in amounts which are expressed as integer numbers. In each moment and in each membrane the total energy involved in an evolution step should be positive, but if ``Too much'' energy is present in a membrane, then the membrane will be destroyed (dissolved). This variant was investigated in two cases, both were shown to be universal:

\begin{enumerate}
	\item when using only two membranes and unbounded amount of energy,
	\item when using arbitrarily many membranes and a bounded energy associated with rules
\end{enumerate}

Freund in \cite{Freund:2004:SequentialEnergy} introduced a new variant where the rules are assigned directly to membranes (every rule consume objects on one side of the membrane and produce objects on the other side) and every membrane carries an energy value that can be changed during a computation by objects passing through the membrane.

This variant is universal even in sequential mode if we allow priorities on the objects. When omitting the priority relation, only the family of Parikh sets generated by context-free matrix grammars ($PsMAT$ as defined in section \ref{sec:matrix_grammars}) is obtained.

% subsection energy_in_p_systems (end)

\subsection{Symport / antiport rules} % (fold)
\label{sub:symport_antiport_rules}

P\u{a}un in \cite{Paun:2002:SymportAntiport} proposes a new way of communicating between membranes.

Symports allow two chemicals to pass together through a membrane in the same direction using symport rules of type $(ab,in)$ or $(ab,out)$.
Antiports allow two chemicals to pass simultaneously through a membrane in opposite directions using antiport rules of type $(a,in;b,out)$.

A P system with symport/antiport rules uses only these two types of rules and nothing else. Surprisingly, this P system variant has been shown to be computationally complete. Five membranes are enough for this result as shown in \cite{Paun:2002:SymportAntiport}. If more than two chemicals may collaborate when passing through membranes, two membranes are sufficient for universality.

Dang in \cite{Dang04Sequential} showed some results about the sequential P systems with symport/antiport rules. When working in just one membrane, its reachability set is the same as of vector addition systems. Thus it is not universal in contrast to the maximal parallel variant, which is universal.

% subsection symport_antiport_rules (end)

\subsection{Parallelism options} % (fold)
\label{sub:parallelism_options}

Original definition of P system (see \cite{Paun98}) uses maximal parallelism when doing a step of computation. There is an obvious biological motivation relying on the assumption that ``if we wait long enough, then all reaction which may take place will take place''. This condition is rather powerful, because it decreases the non-determinism of the system's evolution. For various reasons ranging from looking for more realistic models to just the mathematical challenge, the maximal parallelism was questioned.

% Sequential mode

Dang in \cite{Dang04Sequential} investigates the sequential mode. In each step, from the set of applicable rules across all membrane one is nondeterministically chosen and applied.

\begin{definition}
  \label{def:computation_step_of_a_sequential_P_system}
  A {\bf computation step of a sequential P system} is a relation $\Rightarrow$ on the set of configurations such that $C_1 \Rightarrow C_2$ holds iff there is an applicable rule in a membrane in $C_1$ such that applying that rule can result in $C_2$.
\end{definition}

For purely catalytic systems with 1 membrane, the sequential mode generates only the semilinear sets and thus is strictly weaker than the maximally parallel version.
Sequential version of symport / antiport systems are equivalent to vector addition systems making it strictly weaker than the original maximally parallel version.

Investigation of the sequential mode continues in \cite{Ibarra05Active}. Sequential P system without priorities with cooperative rules with rules for membrane dissolution are not universal by showing they can be simulated by vector addition systems with states (VASS).
This holds even when the membrane creation is allowed for bounded number of created membranes. However, if any number of membranes are allowed to be created, the system becomes universal. This result was shown by simulation of the register machine (see section \ref{sec:register_machines}).

We have further investigated this universal sequential variant in chapter \ref{cha:on_the_edge_of_universality_of_sequential_p_systems} where we will show some of our interesting results on the decidability problems of existence of finite / infinite computation.

Even though the variant without rules for creation of new membranes is not universal, we have studied the effect of adding inhibitors, which was shown to be universal.

% Restricting maximal parallelism

Dang in \cite{Ibarra04dang} proposes several restricted versions of parallelism.
$n$-Max-Parallel version nondeterministically selects a maximal subset of at most n rules to apply. It is proved that 9-Max-Parallel SCO (defined in the subsection \ref{sub:context_in_rules}) is universal.
$\leq n$-Parallel version is similar, but does not require the condition of a maximal subset of rules. It is shown to be weaker than $n$-Max-Parallel version.
$n$-Parallel version requires the size of the subset of rules to apply to be exactly $n$.
All three versions are equal to the sequential mode when $n=1$. For non-universality results, Dang showed the variants are able to be simulated by vector addition systems.

% Minimal parallelism

Ciobanu in \cite{Ciobanu05MinimalParallelism} proposes even more restricted version of parallelism: minimal parallelism. For each region, if at least one rule can be applied, then at least one rule will be applied. The symport / antiport rules variant has been shown to be universal. In \cite{Ciobanu:2007:MinimalParallelism} the results are extended to a variant with active membranes with unbounded membrane creation and polarisations, which was shown to be universal.

% Asynchronous

Freund in \cite{Freund:2004:Async} studied the asynchronous mode of P systems, where in each step, arbitrary many rules can be applied. The application of rules is hence done in parallel way, but are not synchronized or somewhat controlled.

Obviously, for P systems without priorities this is equivalent with just letting them work in the sequential mode, but for P systems with priorities this observation would not be true any more, because the outcome of performing one rule might affect the applicability of another rule.

% subsection parallelism_options (end)

\subsection{Toxic objects} % (fold)
\label{sub:toxic_objects}

Some of the P system variants are aimed at increasing modeling power. Introducting toxic objects \cite{Alhazov14Toxic} does not increase computational power nor modify the decidability of behavioral properties, but such P systems allow for smaller descriptional complexity, especially for smaller number of rules. Moreover, they have a clear direct biological motivation.
Toxic objects are a specified subset of alphabet. They must not stay idle as otherwise the computation is abandoned without yielding a result. 
Alhazov in \cite{Alhazov14Toxic} used toxic objects to improve the known results for catalytic and purely catalytic P systems by significantly reducing the number of rules needed for the constructions in proofs. 

% subsection toxic_objects (end)