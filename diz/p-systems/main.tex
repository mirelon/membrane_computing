% !TEX root = ../diz.tex
In subsection \ref{sub:membrane_systems} we briefly introduced the notions of membrane, membrane structure and P system. Chapter \ref{cha:preliminaries} introduced some basic definitions useful for better understanding of further reading. In this Chapter we define a P system (section \ref{sec:definitions}), its variants with recent advances (section \ref{sec:p_system_variants}) and case studies (section \ref{sec:case_studies}).

A P system \cite{Paun98} is a computing model which abstracts from the way the alive cells process chemical compounds in their compartmental structure. In short, in the regions defined by a membrane structure we have objects which can be described by symbols. Their multiplicity matters, that is, we work with multisets of objects placed in the regions of the membrane structure. Objects evolve according to given rules. Any object, alone or together with another objects, can be transformed in other objects, can pass through a membrane, and can dissolve the membrane in which it is placed. All objects evolve at the same time, in parallel manner across all membranes. The evolution rules are hierarchized by a priority relation, which is a partial order. By using the rules in a nondeterministic, maximally parallel manner, one gets transitions between the system configurations. A sequence of transitions is a computation. With a halting computation we can associate a result, in the form of the objects present in a given membrane in the halting configuration, or expelled from the system during the computation.

% P system

These aspects all together forms a P system as introduced in \cite{Paun98}. An example of a P system containing rules for computing the Fibonacci sequence can be seen in the figure \ref{fig:p_system_fibonacci}.

In section~\ref{sec:definitions} we will provide a formal definition of a P system. We will follow in section \ref{sec:p_system_variants} by presenting most common P system variants along with a survey of known results and in section \ref{sec:case_studies} we show how P systems can be of use in practice case studies.

\section{Definitions} % (fold)
\label{sec:definitions}
{\bf P system with active objects} is a tuple $(V, \mu, w_1, w_2,\dots , w_m, R_1, R_2, \dots , R_m)$, where:
\begin{itemize}
  \item $V$ is the alphabet of symbols,
  \item $\mu$ is a membrane structure consisting of $m$ membranes labeled with numbers $1,2,\dots,m$,
  \item $w_1,w_2,\dots w_m$ are multisets of symbols present in the regions $1,2,\dots,m$ of the membrane structure,
  \item $R_1,R_2,\dots R_m$ are finite sets of rewriting rules associated with the regions $1,2,\dots,m$ of the membrane structure. $R_i\in V^+\times\dots$
\end{itemize}

Each rewriting rule may specify for each symbol on the right side, whether it stays in the current region, moves through the membrane to the parent region ($\uparrow$)
or through membrane to all of the child regions ($\downarrow$)
or to a specific child region ($\downarrow_m$, where $m$ is a label of a membrane).
We denote these transfers with arrows immediately after the symbol.
An example of such rule is the following: $a|b|b\rightarrow a|b\downarrow |c\uparrow|c$.

A {\bf configuration} of a P system is represented by its membrane structure and the multisets of objects in the regions.

A {\bf computation step} of P system is a relation $\Rightarrow$ on the set of configurations such that $C_1 \Rightarrow C_2$ iff:

For every region in $C_1$ (suppose it contains a multiset of objects $w$) the multiset in corresponding region in $C_2$ is the result of applying a multiset of simultaneously applicable multiset rewriting rules to $w$.

In the default P system, which works in maximal parallel mode, a maximal multiset of these rules is applied in each step and region.

For example, let us have two regions with multisets $a|a$ and $b$. In the first region there is a rule $a\rightarrow b$ and in the second membrane there is a rule $b\rightarrow a|a$. The only possible result of a computation step is $b|b$, $a|a$. The first rule was applied twice and the second rule once. No more object could be consumed by rewriting rules.

A {\bf computation} of a P system consists of a sequence of steps. The step $S_i$ is applied to result of previous step $S_{i-1}$. So when $S_i = (C_j,C_{j+1})$, $S_{i-1} = (C_{j-1},C_j)$.

% Result of a computation

There are two possible ways of assigning a result of a computation:

\begin{enumerate}
    \item By considering the multiplicity of objects present in a designated membrane in a halting configuration. In this case we obtain a vector of natural numbers. We can also represent this vector as a multiset of objects or as Parikh image of a language.
    \item By concatenating the symbols which leave the system, in the order they are sent out of the skin membrane (if several symbols are expelled at the same time, then any ordering of them is considered). In this case we generate a language.
\end{enumerate}

The result of a computation is clearly only one multiset or a string, but for one initial configuration there can be multiple possible computations. It follows from the fact that there exist more than one maximal multiset of rules that can be applied in each step.

% section definitions (end)

\section{P system variants} % (fold)
\label{sec:p_system_variants}
Besozzi in his PhD thesis (see \cite{Besozzi:PhD:2004}) formulates three criteria that a good P system variant should satisfy:

\begin{enumerate}
	\item It should be as much realistic as possible from the biological point of view, in order not to widen the distance between the inspiring cellular reality and the idealized theory.
	\item It should result in computational completeness and efficiency, which would mean to obtain universal (and hence, programmable) computing devices, with a powerful and useful intrinsic parallelism;
	\item It should present mathematical minimality and elegance, to the aim of proposing an alternative framework for the analysis of computational models.
\end{enumerate}

In membrane computing, many models are equal in power with Turing machines. We should say they are Turing complete (or computationally complete), but because the proofs are always constructive, starting the constructions from these proofs from universal Turing machines or from equivalent devices, we obtain universal P systems (able
to simulate any other P system of the given type). That is why we speak about universality results, and not about computational completeness.

\subsection{Accepting vs generating} % (fold)
\label{sub:accepting_vs_generating}

In the Chomsky hierarchy, there are language acceptors (finite automata, Turing machines) and language generators (formal grammars).

% Accepting grammars

Bordhin in \cite{Bordihn99acceptingpure} extends grammars to allow for accepting languages by interchanging the left side with the right side of a rule. The mode will apply rewriting rules to an input word and accept it when it reaches the starting nonterminal. However, the input word consists of terminal symbols, which could not be rewritten when using original definition, hence they consider the pure version of various grammar types where they give up the distinction between terminal and nonterminal symbols.

% Accepting vs generating common results

The regular, context-free, context-sensitive and recursively enumerable languages were shown to have equal power in accepting and generating mode.
Some other grammars (programmed grammars with appearance checking) are shown to be more powerful in accepting mode than in generating mode.
For deterministic Lindenmayer systems, the generating and accepting mode are incomparable.

% Accepting vs generating P system results

It can be interesting to investigate accepting and generating mode also in P system variants. Barbuti in \cite{Barbuti:2010:AcceptingGenerating} shown that in the nondeterministic case, when either promoters or cooperative rules are allowed, acceptor P systems have shown to be universal. The same in known to hold for the corresponding classes of nondeterministic generator P systems. In the deterministic case, acceptor P systems have been shown to be universal only if cooperative rules are allowed. Universality has been shown not to hold for the corresponding classes of generator P systems.

% subsection accepting_vs_generating (end)

\subsection{Active vs passive membranes} % (fold)
\label{sub:active_vs_passive_membranes}

% TODO: need citations
Most of the studied P system variants assumes that the number of membranes can only decrease during a computation, by dissolving membranes as a result of applying evolution rules to the objects present in the system.
A natural possibility is to let the number of membranes also to increase during a computation, for instance, by division, as it is well-known in biology. Actually, the membranes from biochemistry are not at all passive, like those in the models briefly described above.
For example, the passing of a chemical compound through a membrane is often done by a direct interaction with the membrane itself (with the so-called protein channels or protein gates present in the membrane); during this interaction, the chemical compound which passes through membrane can be modified, while the membrane itself can in this way be modified (at least locally).

In \cite{Paun99ActiveMembranes} P\u{a}un considers P systems with active membranes where the central role in the computation is played by the membranes: evolution rules are associated both with objects and membranes, while the communication through membranes is performed with the direct participation of the membranes; moreover, the membranes can not only be dissolved, but they also can multiply by division. An elementary membrane can be divided by means of an interaction with an object from that membrane.

% Polarization

Each membrane is supposed to have an electrical polarization (we will say charge), one of the three possible: positive, negative, or neutral. If in a membrane we have two immediately lower membranes of opposite polarizations, one positive and one negative, then that membrane can also divide in such a way that the two membranes of opposite charge are separated; all membranes of neutral charge and all objects are duplicated and a copy of each of them is introduced in each of the two new membranes.
The skin is never divided.
If at the same time a membrane is divided and there are objects in this membrane which are being rewritten in the same step, then in the new copies of the membrane the result of the evolution is included.

In this way, the number of membranes can grow, even exponentially. As expected, by making use of this increased parallelism we can compute faster.
For example, the SAT problem, which is NP complete, can be solved in linear time, when we consider the steps of computation as the time units.
Moreover, the model is shown to be computationally universal.

% subsection active_vs_passive_membranes (end)

\subsection{Context in rules} % (fold)
\label{sub:context_in_rules}

% Cooperative / Non-cooperative

Rewriting rules in P systems can be cooperative and non-cooperative, like in Chomsky's context-free and context-sensitive grammars. Non-cooperative rules are restricted to use only one object on the left side and cooperative rules do not have this restriction.
P systems with cooperative rules are universal \cite{Paun98}, while P systems with non-cooperative rules only characterize Parikh image of context-free languages ($PsCF$) \cite{Sburlan05dragos}.

% Catalytic P systems

P\u{a}un \cite{Paun98} also defines P systems with catalysts where catalysts are a specified subset of the alphabet. Rewriting rules can contain catalysts, which are not modified by applying the rule. Surprisingly, P systems with catalytic rules are universal, actually two membranes in the P system are sufficient to achieve universality.

In systems where only catalytic rules (purely catalytic systems \cite{Ibarra:03:Catalytic}), three catalysts are enough \cite{Freund2005TwoCatalysts}.

% Two catalysts

Freund in \cite{Freund2005TwoCatalysts} also shows that two catalysts and one membrane are enough and raised an open problem whether one catalyst is sufficient. He conjectured that for computationally universal P systems the results obtained in this paper are optimal not only with respect to the number of membranes (2), but also with respect to the number of catalysts.

% Catalysts are too powerful

From some point of view, catalysts are way too powerful in restricting the parallelism - they directly participate in the rules, hence the number of catalytic rules that can be applied in one step, is bounded by number of catalysts.

A variant with promoters and inhibitors have been proposed (see \cite{Ionescu:jucs_10_5:on_p_systems_with}).

% Promoters

In the case of promoters, the rules are possible only in the presence of certain symbols. An object $p$ is a promoter for a rule $u\rightarrow v$ and we denote this by $u\rightarrow v|_{p}$, if the rule is active only in the presence of object $p$. Note that unlike in the case with catalysts, promoters allow the associated rules to be applied as many times as possible.

% Inhibitors

An object $i$ is inhibitor for a rule $u\rightarrow v$ and we denote this by $u\rightarrow v|_{\neg i}$, if the rule is active only if inhibitor $i$ is not present in the region.
One of our results (see section \ref{sec:inhibitors}) uses inhibitors as a tool to achieve universality for sequential P systems.


% One catalyst with promoters / inhibitors

Ionescu in \cite{Ionescu:jucs_10_5:on_p_systems_with} shows that P systems with non-cooperative catalytic rules with only one catalyst and with promoters / inhibitors are universal.

% Zero catalysts with inhibitors

Non-cooperative rules with no catalysts and with inhibitors were studied in \cite{Sburlan:2006:FurtherResultsPromotersInhibitors}, the equivalence with Lindenmayer systems ($ET0L$ as defined in section \ref{sec:lindenmayer_systems}) was proved.

% Simple cooperative system

Dang \cite{Ibarra04dang} proposes a simple cooperative system ($SCO$) as a P system where the only rules allowed are of the form $a\rightarrow v$ or of the form $aa\rightarrow v$, where $a$ is a symbol and $v$ is a (possibly null) string of symbols not containing $a$. This variant is investigated with various modes of parallelism, so their results will be mentioned in the subsection \ref{sub:parallelism_options}

% subsection context_in_rules (end)

\subsection{Rules with priorities} % (fold)
\label{sub:rules_with_priorities}

In the original definition of a P system \cite{Paun98}, a partial order relation over set of rewriting rules have been specified. The rule can be used only if no rule of a higher priority in the region can be applied at the same time.

Sos\'ik in \cite{Sosik:2002:WithoutPriorities} showed that the priorities may be omitted from the model without loss of computational power.

% subsection rules_with_priorities (end)

\subsection{Energy in P systems} % (fold)
\label{sub:energy_in_p_systems}

Various notions of energy has been proposed for use in P systems. P\u{a}un in \cite{Paun:2001:Energy} considers a P system where each evolution rule ``produces'' or ``consumes'' some quantity of energy, in amounts which are expressed as integer numbers. In each moment and in each membrane the total energy involved in an evolution step should be positive, but if ``Too much'' energy is present in a membrane, then the membrane will be destroyed (dissolved). This variant was investigated in two cases, both were shown to be universal:

\begin{enumerate}
	\item when using only two membranes and unbounded amount of energy,
	\item when using arbitrarily many membranes and a bounded energy associated with rules
\end{enumerate}

Freund in \cite{Freund:2004:SequentialEnergy} introduced a new variant where the rules are assigned directly to membranes (every rule consume objects on one side of the membrane and produce objects on the other side) and every membrane carries an energy value that can be changed during a computation by objects passing through the membrane.

This variant is universal even in sequential mode if we allow priorities on the objects. When omitting the priority relation, only the family of Parikh sets generated by context-free matrix grammars ($PsMAT$ as defined in section \ref{sec:matrix_grammars}) is obtained.

% subsection energy_in_p_systems (end)

\subsection{Symport / antiport rules} % (fold)
\label{sub:symport_antiport_rules}

P\u{a}un in \cite{Paun:2002:SymportAntiport} proposes a new way of communicating between membranes.

Symports allow two chemicals to pass together through a membrane in the same direction using symport rules of type $(ab,in)$ or $(ab,out)$.
Antiports allow two chemicals to pass simultaneously through a membrane in opposite directions using antiport rules of type $(a,in;b,out)$.

Surprisingly, a P system variant, where only the symport / antiport rules are used are computationally complete. Five membranes are enough for this result. If more than two chemicals may collaborate when passing through membranes, two membranes are sufficient for universality. These results are proven in \cite{Paun:2002:SymportAntiport}.

% subsection symport_antiport_rules (end)

\subsection{Parallelism options} % (fold)
\label{sub:parallelism_options}

Original definition of P system (see \cite{Paun98}) uses maximal parallelism when doing a step of computation. There is an obvious biological motivation relying on the assumption that ``if we wait long enough, then all reaction which may take place will take place''. This condition is rather powerful, because it decreases the non-determinism of the system's evolution. For various reasons ranging from looking for more realistic models to just the mathematical challenge, the maximal parallelism was questioned.

% Sequential mode

Dang in \cite{Dang04Sequential} investigates the sequential mode. In each step, from the set of applicable rules across all membrane one is nondeterministically chosen and applied. For purely catalytic systems with 1 membrane, the sequential mode generates only the semilinear sets and thus is strictly weaker than the maximally parallel version.
% TODO: add semilinear sets
Sequential version of symport / antiport systems are equivalent to vector addition systems making it strictly weaker than the original maximally parallel version.

Investigation of the sequential mode continues in \cite{Ibarra05Active}. Sequential P system without priorities with cooperative rules with rules for membrane dissolution are not universal by showing they can be simulated by vector addition systems with states (VASS).
This holds even when the membrane creation is allowed for bounded number of created membranes. However, if any number of membranes are allowed to be created, the system becomes universal. This result was shown by simulation of the register machine (see section \ref{sec:register_machines}).

We have further investigated this variant (sequential P system without priorities with cooperative rules) in chapter \ref{cha:on_the_edge_of_universality_of_sequential_p_systems} by allowing rules with inhibitors, which resulted in universality.


% Restricting maximal parallelism

Dang in \cite{Ibarra04dang} proposes several restricted versions of parallelism.
$n${\bf -Max-Parallel} version nondeterministically selects a maximal subset of at most n rules to apply. It is proved that 9{\bf -Max-Parallel} SCO (defined in the subsection \ref{sub:context_in_rules}) is universal.
$\leq n${\bf -Parallel} version is similar, but does not require the condition of a maximal subset of rules. It is shown to be weaker than $n${\bf -Max-Parallel} version.
$n${\bf -Parallel} version requires the size of the subset of rules to apply to be exactly $n$.
All three versions are equal to the sequential mode when $n=1$. For non-universality results, Dang used the proof technique by simulation by vector addition systems. Our future research may be inspired by this technique.

% Minimal parallelism

Ciobanu in \cite{Ciobanu:2007:MinimalParallelism} proposes a minimal parallelism: for each region if at least one rule can be applied, then at least one rule will be applied. The symport / antiport rules variant and variant with active membranes were both shown to be universal.

% Asynchronous

Freund in \cite{Freund:2004:Async} studied the asynchronous mode of P systems, where in each step, arbitrary many rules can be applied. The application of rules is hence done in parallel way, but are not synchronized or somewhat controlled. In many cases the sequential and asynchronous modes were shown to be equivalent.

% subsection parallelism_options (end)


% section p_system_variants (end)

\section{Case studies} % (fold)
\label{sec:case_studies}

% !TEX root = ../diz.tex
Although P systems were inspired by the functioning of a biological cell, the applications has already far exceeded the scope of a cell. P systems are already being used in many problems ranging from modeling biological processes and population dynamics through social network problems to image processing and effectively solving hard problems.

\subsection{MeCoSym} % (fold)
\label{sub:mecosym}

There is an online collection of various case studies on the MeCoSim webpage \cite{MeCoSimWeb}. The number of scenarios and the level of detail vary among the different case studies, possibly including detailed descriptions, references to related publications, charts and videos.

\index{P systems!simulator} MeCoSim (Membrane Computing Simulator) \cite{Perez10MeCoSim} is a software that offers the users a General Purpose Application to model, design, simulate, analyze and verify different types of models based on P systems. Some of the main features of MeCoSim are the following:
\begin{itemize}
  \item Simulation of models of P systems under different initial conditions. It enables the load of P-Lingua based models, parsing, edition, debugging, and different simulation types.
  \item Visualization capabilities for analyzing P systems: alphabet, membrane structure, multisets and graphs viewers.
  \item Highly customizable platform for defining inputs, outputs, parameters and graphs for each model of a family of P systems.
  \item Repositories system for the visual management of available online resources, including plugins, custom applications, models and scenarios.
  \item Export option for releasing end-user applications for solving practical problems, abstracting P system functionalities.
  \item Plugins architecture, permitting the extension of the functionality with Java jars or external non-Java programs. MeCoSim is written in Java.
  \item Auto-update capability, using the latest release of the program whenever it runs.
\end{itemize}

MeCoSim development started in 2010, and a number of case studies have been analyzed since then, covering many variants of P systems, different areas of interest and application domains. The examples contain, at least, some snippets of code, along with the minimal needed files to run some scenario in MeCoSim.

% subsection mecosym (end)

\subsection{Population dynamics} % (fold)
\label{sub:population_dynamics}

The Bearded Vulture (Gypaetus barbatus) is an endangered species in Europe that feeds almost exclusively on bone remains of wild and domestic ungulates. Spanish researchers in \cite{Cardona:2009:Vultures} presented a model of an ecosystem related to the Bearded Vulture in the Pyrenees (NE Spain), by using P systems. The evolution of six species is studied: the Bearded Vulture and five subfamilies of domestic and wild ungulates upon which the vulture feeds. P systems provide a high level computational modeling framework which integrates the structural and dynamic aspects of ecosystems in a comprehensive and relevant way. P systems explicitly represent the discrete character of the components of an ecosystem by using rewriting rules on multisets of objects which represent individuals of the population and bones. The inherent stochasticity and uncertainty in ecosystems is captured by using probabilistic strategies.

This research was extended in \cite{Cardona11Zebra} where they also proposed model for exotic invasive species of Zebra Mussel for the design of a plan for the management of the reservoir of Ribarroja in the area of Ebro River basin because this species create significant environmental problems and economic costs due to its ability to block all types of infrastructure and water pipes.

% subsection population_dynamics (end)

\subsection{Cellular Signalling Pathways} % (fold)
\label{sub:cellular_signalling_pathways}

Perez in \cite{Perez06EGFR} proposed a model for EGFR Signalling Cascade. Cellular signalling pathways are fundamental to the control and regulation of cell behaviour. More than 60 proteins and 160 chemical reactions were included in the model. Membrane structure consists of 3 regions: the environment, the cell surface and the cytoplasm. 

% subsection cellular_signalling_pathways (end)

\subsection{Solving SAT in linear time} % (fold)
\label{sub:solving_sat_in_linear_time}

The discovery of the NP-Complete problems has created a vigorous industry producing proofs of NP-Completeness \cite{NPCompleteness}. But does NP-Completeness really imply intractability of interesting instances? This question seems to divide the Computer Science community into two camps, the pessimists who believe that NP-Complete problems require exponential resources and are therefore intractable and the optimists who expect a subexponential algorithm to emerge eventually. There is however a viable third belief, namely that the resources required do rise exponentially but slowly enough for large practical problems to be tractable.

A number of algorithms have been reported, using an approach more sophisticated than the obvious search of all possible solutions to achieve run times significantly less than the naive approach would suggest, but on a sequential machine these do not yet seem to solve real large problems in reasonable time. On the other hand the naive approach to most NP-Complete problems parallelises easily but yields an algorithm which is too slow even on large parallel machines. However, where the sophisticated algorithms can be parallelised, the result may well be that very large problems which have appeared intractable can be solved quite easily with machines which exist today and will be cheap tomorrow.

Polynomial time solutions to NP-complete problems are usually achieved by trading time for space \cite{Stamp03TimeMemoryTradeoff}.

P systems with their inherent parallelism seem to be a natural choice due to the capabililty of cells to produce an exponential number of new membranes in polynomial time. However, many simulators of P system are inefficient since they cannot handle the parallelism of these devices. Nowadays, we are witnessing the consolidation of the GPUs as a parallel framework to compute general purpose applications such as bitcoin mining.

The simulation of \index{P systems!with active membranes}P systems with active membranes using GPUs is analyzed in \cite{Cecilia10SAT} and an efficient linear solution to the SAT problem is illustrated. They compared it to the classical simulator and reported up to 94x of speedup for 256 literals in the formula. The major constraint of this parallel simulation is the GPU memory size, which can be overcome with a data partition on a cluster of GPUs.

As time passes, GPU mining of bitcoins is largely dead these days. The majority is now mining with Application Specific Integrated Circuits (ASIC) \cite{Smith08ASIC}. The proposal of ASIC for P systems could be the possible aim of future research of P system simulators. 

% subsection solving_sat_in_linear_time (end)

\subsection{Implementation of P systems in vitro} % (fold)
\label{sub:implementation_of_p_systems_in_vitro}

``In vitro'' are studies in experimental biology that uses components of an organism that have been isolated from their usual biological surroundings in order ot provide a more convenient analysis.

This topic could let to the construction of computers based on P systems, and this kind of experiments could be for P systems what are DNA experiments for DNA computing: in vitro use of molecules to calculate. The speed of these processes occurring at membranes (both artificial and natural) is much higher as compared with DNA computing experiments, providing us with a faster computation \cite{Ardelean06InVitro}.

There was a planned experiment of computing the Fibonacci sequence using P systems in vitro (see \cite{Gershoni:2008:InVitro}) using test tubes as membranes and DNA molecules as objects, evolving under the control of enzymes. Number of objects in a multiset was represented by a pre-defined ``mole'' of the substance and synchronization was obtained by ``waiting enough'', such that all reactions that can take place in a test tube actually take place. Hence, they reuqired a variant where reactions do not cycle and proposes the \index{P systems!local loop-free} Local loop-free P systems, which were shown to be universal. Communication was done by moving all the relevant objects to the next tube in a mechanical way. The final result was read by spectrometry.

Many questions and open problems arised from this planned experiment, e.g. if there is any chance to effectively solve NP-Complete problems in this framework. 

There are various opinions on the possibility of computer made of real biological cells. As for now it seems too difficult to grasp. Probably, there are also physical constrains with respect to the physical stability in time of a such complicated proteic structure, as compared with the physical stability of a silicon based component now commonly used in computers. Here, we should not to forget that first bulbs made by Edison had a rather short working life too. The ability of scientists to design chemicals, proteins, with changed desired chemical and physical characteristics has improved significantly in the last decade, and the hardware of a P systems based computer probably needs even more progress in this demiurgic (thus dangerous) activity.

% subsection implementation_of_p_systems_in_vitro (end)


% section case_studies (end)