In this section we introduce several well-known families of languages.

\begin{definition}
A {\bf regular grammar} is a formal grammar, where the rewriting rules are of the form $u\rightarrow v$, where $u\in N$ and $v\in T^*(N\cup \{\eps\})$.
\end{definition}

\begin{definition}
A {\bf regular language} is a language generated by a regular grammar. The family of regular languages is denoted $R$.
\end{definition}

\begin{definition}
A {\bf context-free grammar} is a formal grammar, where rewriting rules are of the form $u\rightarrow v$, where $u\in N$ and $v\in (N\cup T)^*$.
\end{definition}

\begin{definition}
A {\bf context-free language} is a language generated by a context-free grammar. The family of context-free languages is denoted $CF$.
\end{definition}

\begin{definition}
A {\bf context-sensitive grammar} is a formal grammar, where rewriting rules are of the form $u\rightarrow v$, where $u\in (N\cup T)^*N(N\cup T)^*$, $v\in (N\cup T)^*$ and $|u| < |v|$.
\end{definition}

\begin{definition}
A {\bf context-sensitive language} is a language generated by a context-sensitive grammar. The family of context-sensitive languages is denoted $CS$.
\end{definition}

These families of languages forms the Chomsky hierarchy by means of inclusions: $R \subset CF \subset CS \subset RE$.
