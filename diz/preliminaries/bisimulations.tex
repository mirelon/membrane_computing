% !TEX root = ../diz.tex
\begin{definition}
  A {\bf state transition system} is a pair $(S, \rightarrow)$, where $S$ is a set of states and $\rightarrow\subseteq S\times S$ is a binary transition relation over $S$.
\end{definition}
  If $p,q\in S$, then $(p,q)\in \rightarrow$ is usually written as $p\rightarrow q$. This represents the fact that there is a transition from state $p$ to state $q$.

\begin{definition}
  A {\bf labeled state transition system} (LTS) is a tuple $(S, A, \rightarrow)$, where $S$ is a set of states, $A$ is a set of labels and $\rightarrow\subseteq S\times A\times S$ is a ternary transition relation.
\end{definition}

If $p,q\in S$ and $a\in A$, then $(p,a,q)\in \rightarrow$ is usually written as $p\xrightarrow{a} q$. This represents the fact that there is a transition from state $p$ to state $q$ with a label $a$.

\begin{definition}
  Let $(S_1, A, \rightarrow)$ and $(S_2, A, \rightarrow)$ be two labeled transition systems.
  A {\bf simulation} is a binary relation $R\subseteq S_1\times S_2$ such that if $(s_1,s_2)\in R$ then for each $s_1\xrightarrow{a} t_1$ there is some $s_2\xrightarrow{a} t_2$ such that $(t_1, t_2)\in R$.
\end{definition}

\begin{definition}
  Let $(S_1, A, \rightarrow)$ and $(S_2, A, \rightarrow)$ be two labeled transition systems.
  A {\bf bisimulation} is a binary relation $R\subseteq S_1\times S_2$ such that if $(s_1,s_2)\in R$ then:
  \begin{enumerate}
    \item for each $s_1\xrightarrow{a} t_1$ there is some $s_2\xrightarrow{a} t_2$ such that $(t_1, t_2)\in R$,
    \item for each $s_2\xrightarrow{a} t_2$ there is some $s_1\xrightarrow{a} t_1$ such that $(t_1, t_2)\in R$.
  \end{enumerate}
\end{definition}

Sometimes, we want to allows systems to perform internal (silent) steps, of which the impact is considered unobservable. There are multiple notions of the bisimulation. The definition above is called a strong bisimulation.

\begin{definition}
  A {\bf labeled transition system with silent actions} is a triple $(S, A, \rightarrow)$, where $S$ is a set of states, $A$ is a set of actions, also a silent action $\tau$ is assumed that is not in $A$ and $\rightarrow\subseteq (A\cup\tau)\times S$ is the transition relation.
\end{definition}

We use $A_\tau$ to denote $A\cup\{\tau\}$.
By $\xRightarrow{\tau^*}$ we denote the transitive and reflexive closure of $\xrightarrow{\tau}$. For $a\in A$ we define the relation $\xRightarrow{a}$ on $S$ by $r\xRightarrow{a}s$ iff there exists $r^\prime, s^\prime\in S$ such that $r \xRightarrow{\tau^*} r^\prime \xrightarrow{a} s^\prime \xRightarrow{\tau^*} s$.

\begin{definition}
  A {\bf run} of a labeled transition system is a finite, non-empty alternating sequence $\rho = s_0a_0s_1a_1\ldots s_{n-1}a_{n-1}s_n$ of states and actions, beginning and ending with a state such that for $0\leq i<n: s_i\xrightarrow{a_i}s_{i+1}$.
\end{definition}

We also say that $\rho$ is a run from $s_0$. We denote $first(\rho)=s_0$ and $last(\rho)=s_n$.

\begin{definition}
  A relation $R\subseteq S\times S$ is called a {\bf branching bisimulation} if it is symmetric and satisfies the following transfer property: if $rRs$ and $r\xrightarrow{a} r^\prime$ then either $a=\tau$ and $r^\prime Rs$ of there exist $s^\prime,s^{\prime\prime}\in S$ such that $s \xRightarrow{\tau^*} s^\prime \xrightarrow{a} s^{\prime\prime}$, $rRs^\prime$ and $r^\prime Rs^{\prime\prime}$.
\end{definition}

\begin{figure}
  \centering
  \begin{minipage}{.5\textwidth}
    \begin{tikzpicture}[node distance=8mm,-triangle 45]
      \tikzstyle{every node} = [draw=none]
      \node (r1) {$r$};
      \node [right= of r1] (r2) {$r^\prime$};
      \node [below= of r1] (s1) {$s$};
      \node [right=16mm of s1] (or) {or};
      \node [right= of or] (s2) {$s$};
      \node [right= of s2] (s3) {$s^\prime$};
      \node [right= of s3] (s4) {$s^{\prime\prime}$};
      \node [above= of s3] (r3) {$r$};
      \node [right= of r3] (r4) {$r^\prime$};
      \draw (r1) edge node [label,above] {$\tau$} (r2);
      \draw (r1) edge[-] (s1);
      \draw (r2) edge[-,dashed] (s1);
      \draw (r3) edge node [label,above] {$a$} (r4);
      \draw (s2) edge[double,->] node [label,above] {$\tau^*$} (s3);
      \draw (s3) edge node [label,above] {$a$} (s4);
      \draw (r3) edge[-] (s2);
      \draw (r3) edge[-,dashed] (s3);
      \draw (r4) edge[-,dashed] (s4);
    \end{tikzpicture}
    \captionof{figure}{Transfer diagram for branching bisimulation}
    \label{fig:transfer diagram for branching bisimulation}
  \end{minipage}
  \hspace{.08\textwidth}
  \begin{minipage}{.3\textwidth}
    \begin{tikzpicture}[node distance=8mm]
      \tikzstyle{every node} = [draw=none]
      \node (r1) {$r$};
      \node [right= of r1] (r2) {$r^\prime$};
      \node [below= of r1] (s1) {$s$};
      \node [right= of s1] (s2) {$s^\prime$};
      \draw (r1) edge[double,->] node [label,above] {$a$} (r2);
      \draw (s1) edge[double,->] node [label,above] {$a$} (s2);
      \draw (r1) edge[-] (s1);
      \draw (r2) edge[-,dashed] (s2);
    \end{tikzpicture}
    \captionof{figure}{Transfer diagram for weak bisimulation}
    \label{fig:transfer diagram for weak bisimulation}
  \end{minipage}
\end{figure}

\begin{example}
  The diagrams shown in the Figure \ref{fig:transfer diagram for branching bisimulation} summarize the main transfer properties of branching bisimulation. We have used the dashed lines to represent the relations that have to be established in order to conclude that the two states connected by the plain line are equivalent.
\end{example}

We could have strengthened the above definition of branching bisimulation by requiring all intermediate states in $s\xRightarrow{\tau^*}s^\prime$ to be related with $r$. However, that would lead to the same equivalence relation \cite{DeNicola95Bisimulations}. 

\begin{definition}
  A relation $R\subseteq S\times S$ is called a {\bf weak bisimulation} if it is symmetric and satisfies the following transfer property: if $rRs$ and $r\xRightarrow{a} r^\prime$ then there exist $s^\prime\in S$ such that $s \xRightarrow{a} s^\prime$ and $r^\prime Rs^{\prime}$.
\end{definition}

This means that two states are considered equivalent if they lead via the same sequences of visible actions to equivalent states. The intermediate states are not questioned. The diagram in the Figure \ref{fig:transfer diagram for weak bisimulation} summarizes the transfer property for the weak bisimulation. We have used the same notational conventions of the Figure \ref{fig:transfer diagram for branching bisimulation}.

\begin{figure}
  \centering
  \begin{minipage}{.4\textwidth}
    \centering
    \begin{tikzpicture}[level distance=2cm,sibling distance=3cm,-triangle 45]
      \tikzstyle{every node} = [circle,draw]
      \node (Root) {a}
        child {
          node {}
          child {
            node {}
            edge from parent node [above=-5mm,sloped,draw=none] {win car}
          }
          child {
            node {}
            edge from parent node [above=-7mm,sloped,draw=none] {win flowers}
          }
          edge from parent node [left,draw=none] {open door}
        };
    \end{tikzpicture}  
    \captionof{figure}{Example run of a labeled transition system}
    \label{fig:example run of a labeled transition system}
  \end{minipage}
  \hspace{.08\textwidth}
  \begin{minipage}{.5\textwidth}
    \centering
    \begin{tikzpicture}[level distance=2cm,sibling distance=3cm,-triangle 45]
      \tikzstyle{every node} = [circle,draw]
      \node (Root) {}
        child {
          node {}
          child {
            node {}
            edge from parent node [left,draw=none] {win car}
          }
          edge from parent node [left,draw=none] {open door}
        }
        child {
          node {}
          child {
            node {}
            edge from parent node [left,draw=none] {win flowers}
          }
          edge from parent node [right,draw=none] {open door}
        };
    \end{tikzpicture}  
    \captionof{figure}{Example run of a labeled transition system with hidden branching before the first action}
    \label{fig:example run of a labeled transition system with hidden branching before the first action}
  \end{minipage}
\end{figure}

\begin{example}
  In the Figure \ref{fig:example run of a labeled transition system} a participant in a contest opens a door and then it is decided if he wins a car or flowers. In the Figure \ref{fig:example run of a labeled transition system with hidden branching before the first action} the price is decided beforehand so when a participant opens a door, he can directly see his outcome. We can model these events (open door, win car, win flowers) as actions of a labeled transition system.

  The contestant is not interested in the inner states of the system, he only sees these events. For him, the two systems are bisimular in terms of the weak bisimulation.

  In the state right after opening the door and before the price is given there is no way for him knowing the outcome. However, in the system from the Figure \ref{fig:example run of a labeled transition system with hidden branching before the first action} some people who organize the contest already know what is the outcome. For them, the system is not bisimilar in terms of the branching bisimulations.
\end{example}