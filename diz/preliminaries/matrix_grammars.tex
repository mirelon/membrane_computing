% !TEX root = ../diz.tex
Context-free grammars are well-studied and well-behaved in terms of decidability, but many real-world problems cannot be described with context-free grammars. Grammars with regulated rewriting are grammars with mechanisms to regulate the application of rules, so that certain derivations are avoided. Thus, with context-free rules and regulated rewriting mechanisms, one can often generate languages that are not context-free.

One of these is a matrix grammar \cite{Dassow12RegulatedRewriting}, in which instead of single productions, productions are grouped together into finite sequences. A production cannot be applied separately, it must be applied in sequence.

\begin{definition}
A \index{Grammar!matrix} {\bf matrix grammar} is a tuple $G = (N,T,M,\sigma)$, where:
\begin{itemize}
  \item $N, T$ are disjoint alphabets of non-terminal and terminal symbols,
  \item $\sigma\in N$ is the initial non-terminal,
  \item $M$ is a finite set of matrices, which are sequences of context-free rules of the form $u\rightarrow v$, where $u\in N$ and $v\in (N\cup T)^*$.
\end{itemize}
\end{definition}

\begin{definition}
A {\bf rewriting step} $x\Rightarrow y$ holds only if there is a matrix $(u_1\rightarrow v_1, u_2\rightarrow v_2, \ldots, u_n\rightarrow v_n) \in M$ such that for each $1\leq i\leq n$ the following holds: $x_i = x_i^{\prime}u_ix_i^{\prime\prime}$ and $x_{i+1} = x_i^{\prime}v_ix_i^{\prime\prime}$, where $x_i, x_i^{\prime}, x_i^{\prime\prime} \in (N\cup T)^*$ and $x_1 = x$ and $x_{n+1} = y$.
\end{definition}

\begin{example}
Consider the following matrix grammar: $$G=(\{\sigma, X,Y\}, \{ a,b,c\}, M, \sigma),$$ where $M$ contains three matrices: $$[S\rightarrow XY], [X\rightarrow aXb, Y\rightarrow cY], [X\rightarrow ab, Y\rightarrow c].$$ There are only context-free rules, yet the grammar generate the context-sensitive language $\{a^nb^nc^n|n\geq 1\}$.
\end{example}

The family of matrix grammars is denoted $MAT$. It is known that $CF \subset MAT \subset RE$. Interestingly, $MAT \cap {a}^* \subset R$ (see \cite{Besozzi:PhD:2004}).
