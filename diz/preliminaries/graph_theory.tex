% !TEX root = ../diz.tex
\begin{definition}
  A {\em graph} is a pair $G = (V,E)$ of sets such that $E\subseteq V\times V$ and $V\cap E = \emptyset$. The elements of $V$ are the vertices (or nodes) of the graph $G$, the elements of $E$ are its edges.
\end{definition}

The vertex set of a graph $G$ is denoted by $V(G)$, its edge set as $E(G)$.

\begin{definition}
  The {\em order of the graph} $G$ is the number of its vertices, denoted by $|G|$. Graphs are (in)finite iff their order is (in)finite.
\end{definition}

\begin{definition}
  A vertex $v\in V$ is {\em incident} with an edge $e\in E$ iff $v\in e$, i.e. either $e=(v,y)$, where $y\in V$ or $e=(x,v)$, where $x\in V$.
\end{definition}

\begin{definition}
  Two vertices $x,y\in V(G)$ are {\em adjacent} iff $(x,y)\in E(G)$.
\end{definition}

\begin{definition}
  A {\em path} is a non-empty graph $P=(V,E)$, where $V=\{x_0, x_1, \ldots, x_k\}$ and $E=\{(x_i,x_{i+1})|0\leq i < k\}$ and the $x_i$ are all distinct.
\end{definition}

\begin{definition}
  If $P=(V,E)$ is a path where $V=\{x_0, x_1, \ldots, x_k\}$ and $|V|\geq 3$, then the graph $C = (V,E\cup\{(x_k,x_0)\})$ is called a {\em cycle}. 
\end{definition}

\begin{definition}
  A graph $G=(V,E)$ is a {\em subgraph} of a graph $G^\prime = (V^\prime, E^\prime)$ iff $V\subseteq V^\prime$ and $E\subseteq E^\prime$. We denote it by $G\subseteq G^\prime$ and also say that $G^\prime$ contains $G$.
\end{definition}

\begin{definition}
  A non-empty graph $G$ is called {\em connected} iff any two of its vertices are linked by a path in $G$.
\end{definition}

\begin{definition}
  A graph not containing any cycle is called a {\em forest}.
\end{definition}

\begin{definition}
  A connected forest is called a {\em tree}.
\end{definition}

Sometimes it is convenient to consider one vertex of a tree as special. Such a vertex is then called the root of this tree.

\begin{definition}
  A tree $T$ with a fixed root is called a {\em rooted tree}. The root node of $T$ is denoted by $r_T$.
\end{definition}

\begin{definition}
  Let $d$ be a node of a non-root tree $T$, i.e. $d\in V(T)\setminus \{r_T\}$. As $T$ is a tree, there is a unique path from $d$ to $r_T$ \cite{Diestel97Graphs}. The node adjacent to $d$ on that path is also unique and is called a {\em parent node} of $d$ and is denoted by $parent_T(d)$.
\end{definition}

\begin{definition}
  Let $T_1, T_2$ be two rooted trees. A bijection $f: V(T_1)\rightarrow V(T_2)$ is an {\em isomorphism} iff $\forall x,y\in V(T): (x,y)\in E(T_1)\Leftrightarrow (f(x), f(y))\in E(T_2)$.
\end{definition}

Rooted trees $T_1$ and $T_2$ are called isomorphic if there exists an isomorphism on their nodes.
