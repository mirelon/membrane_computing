Vector addition systems were introduced by Karp and Miller \cite{Karp69ParallelProgramSchemata}, and were later shown by Hack \cite{Hack74PetriVAS} to be equivalent to Petri nets.

\begin{definition}
  A {\bf vector addition system} (VAS) is a pair $G = (x, W)$, where $x\in \mathbb N^n$ is an initial vector and $W\subseteq \mathbb Z^n$ is a finite set of vectors, where $n>0$ is called the dimension of VAS.
\end{definition}

The initial vector is seen as the initial values of multiple counters and the vectors in $W$ are seen as actions that update the counters. These counters may never drop below zero. 

\begin{definition}
  The {\bf reachability set} of the VAS $G = (x,W)$ is the set \linebreak $R(G) = \{z | \exists v_1,\ldots,v_j\in W: z=x+v_1+\ldots+v_j \wedge \forall 1\leq i\leq j: x+v_1+\ldots+v_i\geq 0\}$.
\end{definition}

\begin{definition}
  A {\bf vector addition system with states} (VASS) is a tuple $G = (x, W, Q, T, p_0)$, where:
  \begin{itemize}
    \item $(x, W)$ is a vector addition system,
    \item $Q$ is a finite set of states,
    \item $T$ is a finite set of transitions of the form $p\rightarrow(q,v)$, where $v\in W$ and $p,q\in Q$ are states,
    \item $p_0\in Q$ is the starting state.
  \end{itemize}
\end{definition}

The transition $p\rightarrow(q,v)$ can be applied at vector $y$ in state $p$ and yields the vector $y+v$ in state $q$, provided that $y+v\geq 0$.

\begin{example}
  For the Petri net in the Figure \ref{fig:example petri net}, the corresponding VAS $(x,W)$ is:
  \begin{itemize}
    \item $x=(3,0,0,0)$,
    \item $W=\{(-1,1,1,1),(1,0,0,-1)\}$.
  \end{itemize}
\end{example}

It is known \cite{Hack74PetriVAS} that Petri nets, VAS and VASS are computationally equivalent.
