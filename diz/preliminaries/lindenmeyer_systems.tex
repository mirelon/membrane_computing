% !TEX root = ../diz.tex
In 1968, a Hungarian botanist and theoretical biologist Aristid Lindenmayer introduced \cite{Lindenmayer68} a new string rewriting algorithm named Lindenmayer systems (or \index{L-systems} L-systems for short). They are used by biologists and theoretical computer scientists to mathematically model growth processes of living organisms, especially plants. The difference with Chomsky grammars is that rewriting is parallel, not sequential.

The simplest version of L-systems assumes that the development of a cell is free of influence of other cells.
This type of L-systems is called $0L$ systems, where ``0'' stands for zero-sided communication between cells.

\begin{definition}
A $0L$ system is a triple $(\Sigma, P, \omega)$, where $\Sigma$ is an alphabet, $\omega$ is a word over $\Sigma$ and $P$ is a finite set of rewriting rules of the form $a\rightarrow x$, where $a\in\Sigma, x\in\Sigma^*$.
\end{definition}

It is assumed there is at least one rewriting rule for each letter of $\Sigma$. $0L$ system works in parallel way, so all the symbols are rewritten in each step.

\begin{example}
Consider a $0L$ system with alphabet $\Sigma = \{a,b\}$, initial word $\omega = a$ and rewriting rules $P = \{a\rightarrow b, b\rightarrow ab\}$.
Since in this system there is exactly one rule for every letter of the alphabet, the rewriting is thus deterministic and the generated words will be $\{a, b, ab, bab, abbab, \ldots \}$. 
\end{example}

L-systems with tables ($T$) have several sets of rewriting rules instead of just one set. At one step of the rewriting process, rules belonging to the same set have to be applied. The biological motivation for introducing tables is that one may want different rules to take care of different environmental conditions (heat, light, etc.) or of different stages of development.

\begin{definition}
An extended \index{ET0L} ($E0L$) system is a pair $G_1 = (G, \Sigma_T)$, where $G = (\Sigma, P, \omega)$ is an $0L$ system, where $\Sigma_T \subseteq \Sigma$, referred to as the terminal alphabet. The language generated by $G_1$ is defined by $L(G_1) = L(G)\cap \Sigma_T^*$.
\end{definition}

Such languages are called $E0L$ languages. $E0L$ languages with tables are called $ET0L$ languages.

It is known that $CF \subset E0L \subset ET0L \subset CS$ (see section \ref{sec:chomsky_hierarchy} for definitions of $CF$ and $CS$).
