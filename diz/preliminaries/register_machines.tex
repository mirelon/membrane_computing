% !TEX root = ../diz.tex
\begin{definition}
  A {\bf $n$-register machine} is a tuple $M = (n,P,i,h)$, where:
  \begin{itemize}
    \item $n$ is the number of registers,
    \item $P$ is a set of labeled instructions of the form $j : (op(r),k,l)$, where $op(r)$ is an operation on register $r$ of $M$, and $j$, $k$, $l$ are labels from the set $Lab(M)$ (which numbers the instructions in a one-to-one manner),
    \item $i$ is the initial label, and
    \item $h$ is the final label.
  \end{itemize}
\end{definition}

The machine is capable of the following instructions:
\begin{itemize}
  \item $(add(r),k,l)$ : Add one to the contents of register $r$ and proceed to instruction $k$ or to instruction $l$; in the deterministic variants usually considered in the literature we demand $k = l$.
  \item $(sub(r),k,l)$ : If register $r$ is not empty, then subtract one from its contents and go to instruction $k$, otherwise proceed to instruction $l$.
  \item $halt$ : This instruction stops the machine. This additional instruction can only be assigned to the final label $h$.
\end{itemize}

A deterministic $m$-register machine can analyze an input $(n_1,\dots,n_m)\in N_0^m$ in registers 1 to $m$, which is recognized if the register machine finally stops by the halt instruction with all its registers being empty (this last requirement is not necessary). If the machine does not halt, the analysis was not successful.