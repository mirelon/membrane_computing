% !TEX root = ../diz.tex
Consider any kind of software testing problem. Its description will typically start as ``For a given program decide whether the function it computes is\ldots''. In the setting of Turing machines, we often encounter natural problems of the form ``Decide if the language recognized by a given Turing machine\ldots''. Rice's theorem \cite{Rice53Theorem} proves in one clean sweep that all these problems are undecidable. That is, whenever we have a decision problem in which we are given a Turing machine and we are asked to determine a property of the language recognized by the machine, that decision problem is always undecidable. The only exceptions will be the trivial properties that are always true or always false.

Note that Rice's theorem is applicable only for Turing complete models. It means that the properties analyzed in section \ref{sub:analysis_of_behavioral_properties} can still be perfectly decidable as they are studied on Petri nets, which are not Turing complete.

We will mention Rice's theorem again in section \ref{sec:active_membranes}.