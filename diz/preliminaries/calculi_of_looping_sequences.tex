Another tool for describing biological membranes is the formalism Calculus of Looping Sequences (CLS) \cite{Barbuti07CLS}.

In the last few years many formalisms originally developed by computer scientists to model systems of interacting components have been applied to biology. Here, we can mention Petri nets (see Section \ref{sec:petri_nets}). Others, such as P systems (see Chapter \ref{cha:p_systems}), have been proposed as biologically inspired computational models and have been later applied to the description of biological systems.

Many of these models either offer only very low-level interaction primitives or they are specialized to the description of some particular kinds of phenomena such as membrane interactions or protein interactions. Finally, P Systems have a simple notation and are not specialized to the description of a particular class of systems, but they are still not completely general. For instance, it is possible to describe biological membranes and the movement of molecules across membranes, and there are some variants able to describe also more complex membrane activities. However, the formalism is not so flexible to allow describing easily new activities observed on membranes without extending the formalism to model such activities.

For these reasons there was a new formalism called Calculus of Looping Sequences introduced.

CLS is a formalism based on term rewriting with some features, such as a commutative parallel composition operator, and some semantic means, such as bisimulations, which are common in process calculi. This permits to combine the simplicity of notation of rewriting systems with the advantage of a form of compositionality.

CLS does not offer an easy representation for membranes whose nature is fluid and for proteins which consequently move freely on membrane surfaces. For this reason, a variant called CLS+ was introduced in \cite{Milazzo07CLS}.