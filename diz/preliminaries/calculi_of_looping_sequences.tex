Another tool for describing biological membranes is the formalism Calculus of Looping Sequences (CLS) \cite{Barbuti07CLS}.

In the last few years many formalisms originally developed by computer scientists to model systems of interacting components have been applied to biology. Here, we can mention Petri nets (see Section \ref{sec:petri_nets}). Others, such as P systems (see Chapter \ref{cha:p_systems}), have been proposed as biologically inspired computational models and have been later applied to the description of biological systems.

Many of these models either offer only very low-level interaction primitives or they are specialized to the description of some particular kinds of phenomena such as membrane interactions or protein interactions. Finally, P Systems have a simple notation and are not specialized to the description of a particular class of systems, but they are still not completely general. For instance, it is possible to describe biological membranes and the movement of molecules across membranes, and there are some variants able to describe also more complex membrane activities. However, the formalism is not so flexible to allow describing easily new activities observed on membranes without extending the formalism to model such activities.

For these reasons there was a new formalism called Calculus of Looping Sequences introduced.

CLS is a formalism based on term rewriting with some features, such as a commutative parallel composition operator, and some semantic means, such as bisimulations, which are common in process calculi. This permits to combine the simplicity of notation of rewriting systems with the advantage of a form of compositionality.

A CLS model consists of a term and a set of rewrite rules. The term is intended to represent the structure of the modeled system and the rewrite rules to represent the events that may cause the system to evolve.

We start with defining the syntax of terms. We assume a possibly infinite alphabet $\Sigma$ of symbols.

\begin{definition}
  Terms $T$ and sequences $S$ of CLS are given by the following grammar:
  \begin{align*}
    T ::= S \bigpipe (S)^L\rfloor T \bigpipe T|T\\
    S ::= \eps \bigpipe a \bigpipe S\cdot S
  \end{align*}
  where $a\in \Sigma$ and $\eps$ represents the empty sequence. We denote with $\mathcal T$ the infinite set of terms and with $\mathcal S$ the infinite set of sequences.
\end{definition}

\begin{figure}
  \centering
  \begin{tikzpicture}[node distance=16mm,-triangle 45]
    \node(a){a};
    \node[right=12mm of a](f){f};
    \node[right= of f](g){g};
    \node[right=12mm of g](c){c};
    \node[below=10mm of f](d){d};
    \node[right= of d](e){e};
    \node[below right=16mm and 8mm of d.center](b){b};
    \node[below=10mm of b](description){A representation of the term $(a\cdot b\cdot c)^L\rfloor ((d\cdot e)^L | f\cdot g)$.};
    \draw (a) edge[bend right=60] (b);
    \draw (b) edge[bend right=60] (c);
    \draw (c) edge[bend right=60] (a);
    \draw (d) edge[bend right=60] (e);
    \draw (e) edge[bend right=60] (d);
    \draw (f) edge (g);
  \end{tikzpicture}
  \caption{Example CLS}
  \label{fig:example cls}
\end{figure}

In CLS we have a sequencing operator $\_\cdot\_$, a looping operator $(\_)^L$, a parallel composition operator $\_|\_$ and a containment operator $\_\rfloor\_$. Sequencing can be used to concatenate elements of the alphabet $\Sigma$. The empty sequence $\eps$ denotes the concatenation of zero symbols. A term can be either a sequence or a looping sequence (that is the application of the looping operator to a sequence) containing another term, or the parallel composition of two terms. By definition, looping and containment are always applied together, hence we can consider them as a single binary operator $(\_)^L\rfloor\_$ which applies to one sequence and one term.

\begin{example}
  In the Figure \ref{fig:example cls} we show an example of CLS and its visual representation. The same structure may be represented by syntactically different terms, e.g. $(b\cdot c\cdot a)^L\rfloor (f\cdot g | (e\cdot d)^L)$. We introduce a structural congruence relation to identify such terms.
\end{example}

\begin{definition}
  The structural congruence relations $\equiv_S$ and $\equiv_T$ are the least congruence relations on sequences and on terms, respectively, satisfying the following rules:
  \begin{itemize}
    \item $S_1\cdot(S_2\cdot S_3)\equiv_S (S_1\cdot S_2)\cdot S_3$
    \item $S\cdot\eps\equiv_S \eps\cdot S\equiv_S S$
    \item $S_1\equiv_S S_2$ implies $S_1\equiv_T S_2$ and $(S_1)^L\rfloor T\equiv_T (S_2)^L\rfloor T$
    \item $T_1|T_2\equiv_T T_2|T_1$
    \item $T_1|(T_2|T_3)\equiv_T (T_1|T_2)|T_3$
    \item $T|\eps\equiv_T T$
    \item $(\eps)^L\rfloor\eps\equiv_T\eps$
    \item $(S_1\cdot S_2)^L\rfloor T\equiv_T (S_2\cdot S_1)^L\rfloor T$
  \end{itemize}
\end{definition}

Note that the last rule does not introduce the commutativity of sequences, but only says that looping sequences can rotate.

What could look strange in CLS is the use of looping sequences for the description of membranes, as sequencing is not a commutative operation and this do not correspond to the usual fluid representation of membrane surface in which objects can move freely. What one would expect is to have a multiset or a parallel composition of objects on a membrane. For this reason, a variant called CLS+ was introduced in \cite{Milazzo07CLS}, in which the looping operator can be applied
to a parallel composition of sequences.

\begin{definition}
  Terms $T$, branes $B$ and sequences $S$ of CLS+ are given by the following grammar:
  \begin{align*}
    T ::= S \bigpipe (B)^L\rfloor T \bigpipe T|T\\
    B ::= S \bigpipe B|B\\
    S ::= \eps \bigpipe a \bigpipe S\cdot S
  \end{align*}
\end{definition}

The structural congruence relation of CLS+ is a trivial extension of the one of CLS. The only difference is that commutativity of branes replaces rotation of
looping sequences. CLS+ models can be translated into CLS models, while preserving the semantics of the model \cite{Barbuti07CLS}. Moreover, in the same paper, a traslation of maximal parallel P system to CLS is shown.