% !TEX root = ../diz.tex
We are now ready to define the basic type of computation model inspired by the study of real languages - formal grammars. Realizing the parts of sentences could be labeled to indicate their function, it seems natural that one might notice that there appear to be rules dictating how these parts of speech are used.
\begin{definition}
A \index{Grammar!formal} {\bf formal grammar} is a tuple $G = (N,T,P,\sigma)$, where
\begin{itemize}
  \item $N, T$ are disjoint alphabets of non-terminal and terminal symbols,
  \item $\sigma\in N$ is the initial non-terminal,
  \item $P$ is a finite set of rewriting rules of the form $u\rightarrow v$, with $u\in (N\cup T)^*N(N\cup T)^*$ and $v\in (N\cup T)^*$.
\end{itemize}
\end{definition}

\begin{definition}
A {\bf rewriting step} in the grammar $G$ is a binary relation $\Rightarrow$ on $(N\cup T)^*$, where $x\Rightarrow y$ iff $\exists w_1, w_2\in (N\cup T)^*$ and a rule $u\rightarrow v \in P$ such that $x=w_1uw_2$ and $y=w_1vw_2$.
\end{definition}

\begin{definition}
\index{Language} Language defined by a grammar $G$ is a set $L(G)=\{w\in T^*|\sigma\Rightarrow w\}$.
\end{definition}

Set of languages that can be defined by a formal grammar are called \index{Language!recursively enumerable} recursively enumerable languages $RE$.
