This chapter is about some basic notions of computer science which will be used through the work. We start by defining formal languages and basic models (grammars, machines) that define language families and end by defining multiset languages.

\section{Formal languages} % (fold)
\label{sec:formal_languages}
Our study is based on the classical theory of formal languages. We will recall some definitions:

\begin{definition}
An {\bf alphabet} is a finite nonempty set of symbols.
\end{definition}

\begin{definition}
A {\bf string} over an alphabet is a finite sequence of symbols from alphabet.
\end{definition}

The length of the string $s$ is denoted by $|s|$. We denote by $V^*$ the set of all strings over an alphabet $V$. By $V^+$ = $V^* - \{\eps\}$ we denote the set of all nonempty strings over V.

\begin{definition}
A {\bf language} over the alphabet $V$ is any subset of $V^*$.
\end{definition}

\begin{definition}
A {\bf family of languages} is a set of languages.
\end{definition}


% section formal_languages (end)

\section{Formal grammars} % (fold)
\label{sec:formal_grammars}
% !TEX root = ../diz.tex
We are now ready to define the basic type of computation model inspired by the study of real languages - formal grammars. Realizing the parts of sentences could be labeled to indicate their function, it seems natural that one might notice that there appear to be rules dictating how these parts of speech are used.
\begin{definition}
A {\bf formal grammar} is a tuple $G = (N,T,P,\sigma)$, where
\begin{itemize}
  \item $N, T$ are disjoint alphabets of non-terminal and terminal symbols,
  \item $\sigma\in N$ is the initial non-terminal,
  \item $P$ is a finite set of rewriting rules of the form $u\rightarrow v$, with $u\in (N\cup T)^*N(N\cup T)^*$ and $v\in (N\cup T)^*$.
\end{itemize}
\end{definition}

\begin{definition}
A {\bf rewriting step} in the grammar $G$ is a binary relation $\Rightarrow$ on $(N\cup T)^*$, where $x\Rightarrow y$ only if $\exists w_1, w_2\in (N\cup T)^+$ and a rule $u\rightarrow v \in P$ such that $x=w_1uw_2$ and $y=w_1vw_2$.
\end{definition}

\begin{definition}
Language defined by a grammar $G$ is a set $L(G)=\{w\in T^*|\sigma\Rightarrow w\}$.
\end{definition}

Languages that can be generated by a formal grammar are the recursively enumerable languages $RE$.


% section formal_grammars (end)

\section{Chomsky hierarchy} % (fold)
\label{sec:chomsky_hierarchy}
In this section we introduce several well-known families of languages.

\begin{definition}
A {\bf regular grammar} is a formal grammar, where the rewriting rules are of the form $u\rightarrow v$, where $u\in N$ and $v\in T^*(N\cup \{\eps\})$.
\end{definition}

\begin{definition}
A {\bf regular language} is a language generated by a regular grammar. The family of regular languages is denoted $R$.
\end{definition}

\begin{definition}
A {\bf context-free grammar} is a formal grammar, where rewriting rules are of the form $u\rightarrow v$, where $u\in N$ and $v\in (N\cup T)^*$.
\end{definition}

\begin{definition}
A {\bf context-free language} is a language generated by a context-free grammar. The family of context-free languages is denoted $CF$.
\end{definition}

\begin{definition}
A {\bf context-sensitive grammar} is a formal grammar, where rewriting rules are of the form $u\rightarrow v$, where $u\in (N\cup T)^*N(N\cup T)^*$, $v\in (N\cup T)^*$ and $|u| < |v|$.
\end{definition}

\begin{definition}
A {\bf context-sensitive language} is a language generated by a context-sensitive grammar. The family of context-sensitive languages is denoted $CS$.
\end{definition}

These families of languages forms the Chomsky hierarchy by means of inclusions: $R \subset CF \subset CS \subset RE$.


% section chomsky_hierarchy (end)

\section{Matrix grammars} % (fold)
\label{sec:matrix_grammars}
% !TEX root = ../diz.tex
Context-free grammars are well-studied and well-behaved in terms of decidability, but many real-world problems cannot be described with context-free grammars. Grammars with regulated rewriting are grammars with mechanisms to regulate the application of rules, so that certain derivations are avoided. Thus, with context-free rules and regulated rewriting mechanisms, one can often generate languages that are not context-free.

One of these is a matrix grammar \cite{Dassow12RegulatedRewriting}, in which instead of single productions, productions are grouped together into finite sequences. A production cannot be applied separately, it must be applied in sequence.

\begin{definition}
A \index{Grammar!matrix} {\bf matrix grammar} is a tuple $G = (N,T,M,\sigma)$, where:
\begin{itemize}
  \item $N, T$ are disjoint alphabets of non-terminal and terminal symbols,
  \item $\sigma\in N$ is the initial non-terminal,
  \item $M$ is a finite set of matrices, which are sequences of context-free rules of the form $u\rightarrow v$, where $u\in N$ and $v\in (N\cup T)^*$.
\end{itemize}
\end{definition}

\begin{definition}
A {\bf rewriting step} $x\Rightarrow y$ holds only if there is a matrix $(u_1\rightarrow v_1, u_2\rightarrow v_2, \ldots, u_n\rightarrow v_n) \in M$ such that for each $1\leq i\leq n$ the following holds: $x_i = x_i^{\prime}u_ix_i^{\prime\prime}$ and $x_{i+1} = x_i^{\prime}v_ix_i^{\prime\prime}$, where $x_i, x_i^{\prime}, x_i^{\prime\prime} \in (N\cup T)^*$ and $x_1 = x$ and $x_{n+1} = y$.
\end{definition}

\begin{example}
Consider the following matrix grammar:
$$G=(\{\sigma, X,Y\}, \{ a,b,c\}, M, \sigma),$$
where $M$ contains three matrices:
$$[\sigma\rightarrow XY], [X\rightarrow aXb, Y\rightarrow cY], [X\rightarrow ab, Y\rightarrow c].$$
The following example computation of $G$ uses the first matrix, then three times the second matrix and finally the last matrix:
$$S\Rightarrow XY\Rightarrow aXbcY\Rightarrow aaXbbccY\Rightarrow aaaXbbbcccY\Rightarrow aaabbbccc$$
Although there are only context-free rules, yet the grammar generate the context-sensitive language $\{a^nb^nc^n|n\geq 1\}$.
\end{example}

The family of matrix grammars is denoted $MAT$. It is known that $CF \subset MAT \subset RE$. Interestingly, $MAT \cap {a}^* \subset R$ (see \cite{Besozzi:PhD:2004}).


% section matrix_grammars (end)

\section{Register machines} % (fold)
\label{sec:register_machines}
% !TEX root = ../diz.tex
\begin{definition}
  A {\em $n$-register machine} is a tuple $M = (n,P,i,h)$, where:
  \begin{itemize}
    \item $n$ is the number of registers,
    \item $P$ is a set of labeled instructions of the form $j : (op(r),k,l)$, where $op(r)$ is an operation on register $r$ of $M$, and $j$, $k$, $l$ are labels from the set $Lab(M)$ (which numbers the instructions in a one-to-one manner),
    \item $i$ is the initial label, and
    \item $h$ is the final label.
  \end{itemize}
\end{definition}

The machine is capable of the following instructions:
\begin{itemize}
  \item $(add(r),k,l)$ : Add one to the contents of register $r$ and proceed to instruction $k$ or to instruction $l$; in the deterministic variants usually considered in the literature we demand $k = l$.
  \item $(sub(r),k,l)$ : If register $r$ is not empty, then subtract one from its contents and go to instruction $k$, otherwise proceed to instruction $l$.
  \item $halt$ : This instruction stops the machine. This additional instruction can only be assigned to the final label $h$.
\end{itemize}

A deterministic $m$-register machine can analyze an input $(n_1,\dots,n_m)\in N_0^m$ in registers 1 to $m$, which is recognized if the register machine finally stops by the halt instruction with all its registers being empty (this last requirement is not necessary). If the machine does not halt, the analysis was not successful.

% section register_machines (end)

\section{Lindenmayer systems} % (fold)
\label{sec:lindenmayer_systems}
In 1968, a Hungarian botanist and theoretical biologist Aristid Lindenmayer introduced \cite{Lindenmayer68} a new string rewriting algorithm named Lindenmayer systems (or L-systems for short). They are used by biologists and theoretical computer scientists to mathematically model growth processes of living organisms, especially plants. The difference with Chomsky grammars is that rewriting is parallel, not sequential.

The simplest version of L-systems assumes that the development of a cell is free of influence of other cells.
This type of L-systems is called $0L$ systems, where ``0'' stands for zero-sided communication between cells.

\begin{definition}
A $0L$ system is a triple $(\Sigma, P, \omega)$, where $\Sigma$ is an alphabet, $\omega$ is a word over $\Sigma$ and $P$ is a finite set of rewriting rules of the form $a\rightarrow x$, where $a\in\Sigma, x\in\Sigma^*$.
\end{definition}

It is assumed there is at least one rewriting rule for each letter of $\Sigma$. $0L$ system works in parallel way, so all the symbols are rewritten in each step.

\begin{example}
Consider a $0L$ system with alphabet $\Sigma = \{a,b\}$, initial word $\omega = a$ and rewriting rules $P = \{a\rightarrow b, b\rightarrow ab\}$.
Since in this system there is exactly one rule for every letter of the alphabet, the rewriting is thus deterministic and the generated words will be $\{a, b, ab, bab, abbab, \ldots \}$. 
\end{example}

$1L$ systems allows the rewriting rules to include context of size 1, so it allows for rules of type $yaz\rightarrow x$.

L-systems with tables ($T$) have several sets of rewriting rules instead of just one set. At one step of the rewriting process, rules belonging to the same set have to be applied. The biological motivation for introducing tables is that one may want different rules to take care of different environmental conditions (heat, light, etc.) or of different stages of development.

\begin{definition}
An extended ($E0L$) system is a pair $G_1 = (G, \Sigma_T)$, where $G = (\Sigma, P, \omega)$ is an $0L$ system, where $\Sigma_T \subseteq \Sigma$, referred to as the terminal alphabet. The language generated by $G_1$ is defined by $L(G_1) = L(G)\cap \Sigma_T^*$.
\end{definition}

Such languages are called $E0L$ languages. $E0L$ languages with tables are called $ET0L$ languages.

It is known that $CF \subset E0L \subset ET0L \subset CS$ (see section \ref{sec:chomsky_hierarchy} for definitions of $CF$ and $CS$).


% section lindenmayer_systems (end)

\section{Multisets} % (fold)
\label{sec:multisets}
% !TEX root = ../diz.tex
\begin{definition}
A multiset over a set $X$ is a mapping $M: X\rightarrow \mathbb N$.
\end{definition}

We denote by $M(x), x\in X$ the multiplicity of $x$ in the multiset $M$.

\begin{definition}
The {\em support} of a multiset $M$ is the set $supp(M)=\{x\in X|M(x)\geq 1\}$.
\end{definition}

It is the set of items with at least one occurrence.

\begin{definition}
A multiset is {\em empty} when its support is empty.
\end{definition}

A multiset $M$ with finite support $X = \{x_1, x_2, \ldots, x_n\}$ can be represented by the string $x_1^{M(x_1)}x_2^{M(x_2)}\ldots x_n^{M(x_n)}$.
As elements of a multiset can also be strings, we separate them with the pipe symbol, e.g. $$element|element|other\_element$$.

\begin{definition}
Multiset inclusion. We say that multiset $M_1$ is included in multiset $M_2$ if $\forall x \in X: M_1(x)\leq M_2(x)$. We denote it by $M_1\subseteq M_2$.
\end{definition}

\begin{definition}
The {\em union} of two multisets $M_1\cup M_2$ is a multiset where $\forall x \in X: (M_1\cup M_2)(x)=M_1(x)+M_2(x)$.
\end{definition}

\begin{definition}
The {\em difference} of two multisets $M_1-M_2$ is a multiset where $\forall x \in X: (M_1-M_2)(x)=M_1(x)-M_2(x)$.
\end{definition}

\begin{definition}
Product of multiset $M$ with natural number $n\in \mathbb N$ is a multiset where $\forall x \in X: (n\cdot M)(x)=n\cdot M(x)$.  
\end{definition}


% section multisets (end)

\section{Semilinear sets} % (fold)
\label{sec:semilinear_sets}
% !TEX root = ../diz.tex
\begin{definition}
  A {\em linear set} $L(c,p_1,\ldots,p_r)$ is a subset $\{c+\sum\limits_{i=1}^r|k_i\in\mathbb N\}$ of $\mathbb N^n$, where $c,p_1,\ldots,p_r\in \mathbb N^n$.
\end{definition}

We call $c$ the constant and $p_1,\ldots,p_r$ the periods of the linear set.

\begin{example}
  For $n=1$ the linear set is a subset of $\mathbb N$. For $n=1$ and $r=1$ we get an arithmetic progression.
\end{example}

\begin{example}
  $L((0,0),(0,1),(1,0))$ contains all pairs with one zero element and one non-negative element: $$\{(0,0), (1,0), (2,0), \ldots, (x,0), \ldots, (0,1), (0,2), \ldots, (0,y), \ldots\}$$.
\end{example}

\begin{definition}
  A subset of $\mathbb N^n$ is called {\em semilinear} if it is a finite union of linear sets.
\end{definition}

\subsection{Parikh's mapping} % (fold)
\label{sec:parikh_s_mapping}

In this subsection we show how semilinear sets and multisets relate to the formal language theory. We will start with several basic definitions.

The number of occurrences of a given symbol $a\in \Sigma$ in the string $w\in \Sigma^*$ is denoted by $|w|_a$.

\begin{definition}
$\Psi_\Sigma(w)=(|w|_{a_1},|w|_{a_2},\ldots,|w|_{a_n})$ is called a {\em Parikh image of the string} $w\in \Sigma^*$, where $\Sigma=\{a_1,a_2,\ldots a_n\}$.
\end{definition}

When referring to the Parikh mapping for a language, this should be taken to mean the mapping applied to all the words on the language. This idea is expressed in the next definition. 

\begin{definition}
For a language $L\subseteq \Sigma^*$, $\Psi_\Sigma(L)=\{\Psi_\Sigma(w)|w\in L\}$ is the {\em Parikh image of the language} $L$.
\end{definition}

\begin{definition}
If $FL$ is a family of languages, by $PsFL$ we denote the family of Parikh images of languages in $FL$, e.g. $PsCF, PsRE$.
\end{definition}

\begin{example}
Consider an alphabet $V=\{a,b\}$ and a language $L=\{a, ab, ba\}$.
$\Psi_\Sigma(L)=\{(1,0), (1,1)\}$. Notice that Parikh image of $L$ has only 2 element while $L$ has 3 elements.
\end{example}

\begin{example}
  Consider the context-free grammar $G = (N,T,P,\sigma)$, where $N=\{A,B\}$, $T=\{a,b\}$ and with rules $P=\{\sigma\rightarrow A\sigma B|B\sigma A|ab, A\rightarrow a, B\rightarrow b\}$.

  This means that the language generated by $G$ contains strings where $a$s and $b$s can occur intermixed. But, as is clear from the language definition, the same number of $a$s and $b$s will always be present. Moreover, there will always be at least one of each letters.

  Consider the words $w_1 = aaaabbbb \in L(G)$ and $w_2 = babababa \in L(G)$. Both of these words have the same Parikh image: $\Psi_\Sigma(w_i) = (4,4)$. It is interesting to note that most information embedded in a word generated with a context-free grammar is thrown away by the Parikh mapping.
\end{example}

This loss of information is expressed in the Parikh's theorem \cite{Parikh66}, stating that the Parikh image of a context-free language is semilinear. The biggest implication of the theorem though, is that it shows that if the order of the symbols is ignored, then it is impossible to distinguish between a regular set and a context-free language \cite{Kozen97Automata}.

Another interesting result appeared in \cite{Ito69Semilinear} that every semilinear set is a finite union of disjoint linear sets.

% subsection parikh_s_mapping (end)


% section semilinear_sets (end)

\section{Petri nets} % (fold)
\label{sec:petri_nets}
% !TEX root = ../diz.tex
Another well-studied formalism often referred to (e.g. in \cite{Dang04Sequential,Freund:2004:Async}) are Petri nets. Petri nets \cite{Petri62,Yen06PetriNets} were introduced by Carl Adam Petri in 1962 in his PhD thesis. A Petri net is a graphical and mathematical tool for the modeling of concurrent processes and analysis of system behavior. A Petri net is usually drawn as a directed bipartite graph with two kind of nodes. Places are represented by circles within which each small black dot denotes a token. Transitions are represented by bars. Each edge is either from a place to a transition or vice versa.

\begin{definition}
  A {\bf Petri net} is a tuple $(P, T, \varphi)$, where:
  \begin{itemize}
    \item $P$ is a finite set of places,
    \item $T$ is a finite set of transitions,
    \item $\varphi: (P\times T)\cup(T\times P)\rightarrow \mathbb N$ is a flow function.
  \end{itemize}
\end{definition}

The edges of the bipartite graph are annotated by either $\varphi(p,t)$ or $\varphi(t,p)$, where $p\in P$ and $t\in T$ are two endpoints of the arc. If $\varphi(p,t)=1$ or $\varphi(t,p)=1$, we usually omit the label.

\begin{definition}
  A {\bf marking} is a mapping $\mu: P\rightarrow \mathbb N$.
\end{definition}

The mapping $\mu$ assigns certain number of tokens to each place of the net.

\begin{definition}
  A marking $\mu_1$ {\bf covers} marking $\mu_2$, when $\forall p\in P: \mu_1(p)\geq\mu_2(p)$ - in each place there is no less tokens in $\mu_1$ than in $\mu_2$. We denote it by $\mu_1\geq\mu_2$.
\end{definition} 

\begin{definition}
  A transition $t\in T$ is {\bf enabled} at a marking $\mu$ iff $\forall p\in P, \varphi(p,t)\leq\mu(p)$.
\end{definition}

If a transition $t$ is enabled, it may fire by removing $\varphi(p,t)$ tokens from each input place $p$ and putting $\varphi(t,p^\prime)$ tokens in each output place $p^\prime$. We then write $\mu\xrightarrow{t} \mu^\prime$, where $\forall p\in P: \mu^\prime(p) = \mu(p)-\varphi(p,t)+\varphi(t,p)$.

\begin{example}
  In the Figure \ref{fig:example petri net} the Petri net has four places and two transitions. At the current marking the transition $t_1$ is enabled and the transition $t_2$ is not enabled. Firing the transition $t_1$ takes one token from the place $p_1$ and produces one token to places $p_2, p_3$ and $p_4$. In the resulting marking both transitions $t_1$ and $t_2$ are enabled.
\end{example}

\begin{definition}
  A {\bf marked Petri net} is a tuple $(P,T,\varphi,\mu_0)$, where $(P,T,\varphi)$ is a Petri net and $\mu_0$ is called the initial marking.
\end{definition}

\begin{definition}
  A sequence of transitions $\sigma = t_1\ldots t_n$ is a {\bf firing sequence} from $\mu_0$ iff $\mu_0\xrightarrow{t_1}\mu_1\xrightarrow{t_2}\ldots\xrightarrow{t_n}\mu_n$ for some markings $\mu_1,\ldots,\mu_n$. We also write $\mu_0\xrightarrow{\sigma}\mu_n$.
\end{definition}

We write $\mu_0\xrightarrow{\sigma}$ to denote that $\sigma$ is enabled and can be fired from $\mu_0$, i.e., $\mu_0\xrightarrow{\sigma}$ iff there exists a marking $\mu$ such that $\mu_0\xrightarrow{\sigma}\mu$.
The notation $\mu_0\xrightarrow{*}\mu$ is used to denote the existence of a firing sequence $\sigma$ such that $\mu_0\xrightarrow{\sigma}\mu$.

\begin{definition}
  A marking $\mu$ is reachable for a marked Petri net $\mathcal P = (P,T,\varphi,\mu_0)$ iff $\mu_0\xrightarrow{*}\mu$.
\end{definition}

\begin{definition}
  Let $\mathcal P = (P,T,\varphi,\mu_0)$ be a marked Petri net. The {\bf reachability set} of $\mathcal P$ is $R(\mathcal(P)) = \{\mu|\mu_0\xrightarrow{*}\mu\}$.
\end{definition}

A notion of reachability graph is helpful for analyzing the behavior of a Petri net as a tool for visualisation of the structure of the reachability set.

\begin{definition}
  Let $\mathcal P = (P,T,\varphi,\mu_0)$ be a marked Petri net. The {\bf reachability graph} of $\mathcal P$ is a labelled graph whose nodes are the reachable markings and edge from $\mu_1$ to $\mu_2$ is labeled with a transition $t\in T$ iff $\mu_1\xrightarrow{t}\mu_2$.
\end{definition}

\begin{figure}
  \centering
  \begin{minipage}{.4\textwidth}
    \begin{tikzpicture}
      \tikzstyle{transition}=[rectangle,thick,fill=black,minimum height=8mm]
      \node [place,tokens=3,label=above:$p_1$] (p1) {};
      \node [transition,label=above:$t_1$] (t1) [right of=p1] {}
        edge [pre] (p1);
      \node [place,tokens=0,label=right:$p_3$] (p3) [right of=t1] {}
        edge [pre] (t1);
      \node [place,tokens=0,label=right:$p_2$] (p2) [above of=p3] {}
        edge [pre] (t1);
      \node [place,tokens=0,label=right:$p_4$] (p4) [below of=p3] {}
        edge [pre] (t1);
      \node [transition,label=below:$t_2$] (t2) [below of=t1] {}
        edge [pre] (p4)
        edge [post] (p1);
    \end{tikzpicture}
    \captionof{figure}{An example Petri net}
    \label{fig:example petri net}
  \end{minipage}
  \hspace{.08\textwidth}
  \begin{minipage}{.4\textwidth}
    \begin{tikzpicture}[node distance=8mm,-triangle 45]
      \tikzstyle{every node} = [rectangle,draw]
      \tikzstyle{label} = [draw=none]
      \node (1) {3,0,0,0};
      \node [below= of 1] (2) {2,1,1,1};
      \node [below= of 2] (3) {1,2,2,2};
      \node [below= of 3] (4) {0,3,3,3};
      \node [right= of 2] (5) {3,1,1,0};
      \node [below= of 5] (6) {2,2,2,1};
      \node [below= of 6] (7) {1,3,3,2};
      \node [below= of 7] (8) {1,4,4,3};
      \node [draw=none,right= of 6] (9) {$\ldots$};
      \node [draw=none,right= of 7] (10) {$\ldots$};
      \node [draw=none,right= of 8] (11) {$\ldots$};
      \draw (1) edge node [label,right] {$t_1$} (2);
      \draw (2) edge node [label,right] {$t_1$} (3);
      \draw (3) edge node [label,right] {$t_1$} (4);
      \draw (2) edge node [label,above] {$t_2$} (5);
      \draw (3) edge node [label,above] {$t_2$} (6);
      \draw (4) edge node [label,above] {$t_2$} (7);
      \draw (5) edge node [label,right] {$t_1$} (6);
      \draw (6) edge node [label,right] {$t_1$} (7);
      \draw (7) edge node [label,right] {$t_1$} (8);
      \draw (6) edge node [label,above] {$t_2$} (9);
      \draw (7) edge node [label,above] {$t_2$} (10);
      \draw (8) edge node [label,above] {$t_2$} (11);
    \end{tikzpicture}
    \captionof{figure}{An example reachability graph}
    \label{fig:example reachability graph}
  \end{minipage}
\end{figure}

\begin{example}
  Consider a Petri net $\mathcal P$ from the Figure \ref{fig:example petri net}. Its reachability graph is in the Figure \ref{fig:example reachability graph}. $\mathcal P$ is not bounded because by alternately firing transitions $t_1$ and $t_2$ we can reach infinitely many different markings. We can also easily see that it is live, because in every marking for every transition $t\in\{t_1, t_2\}$ there is a firing sequence ending with $t$.
\end{example}

In spite of its simplicity, the applicability of the technique of reachability graph analysis is rather limited in the sense that it suffers from the state explosion phenomenon as the sizes of the reachability sets grow beyond any primitive recursive function in the worst case \cite{Yen06PetriNets}.

Coverability graph analysis offers an alternative to the techinque of reachability graph analysis by abstracting out certain details to make the graph finite. To understand the intuition behind coverability graphs, consider the Figure \ref{fig:example reachability graph} which shows a part of the reachability graph of the Petri net in the Figure \ref{fig:example petri net}. Consider the path $(3,0,0,0)\xrightarrow{t_1}(2,1,1,1)\xrightarrow{t_2}(3,1,1,0)$ along which the places $p_2$ and $p_3$ both gain an extra token in the end, i.e. $(3,0,0,0) > (3,1,1,0)$. Clearly they can be made to contain arbitrary large number of tokens by repeating the firing sequence $t_1t_2$ for a sufficient number of times, as $(3,0,0,0)\xrightarrow{t_1t_2}(3,1,1,0)\xrightarrow{t_1t_2}(3,2,2,0)\xrightarrow{t_1t_2}\ldots\xrightarrow{t_1t_2}(3,n,n,0)$, for arbitrary $n$. In order to capture the notion of a place being unbounded, we short-circuit the above infinite sequence of computation as $(3,0,0,0)\xrightarrow{t_1}(2,1,1,1)\xrightarrow{t_2}(3,\omega,\omega,0)$, where $\omega$ is a symbol denoting something being arbitrarily large. As it turns out, the coverability graph of a Petri net is always finite \cite{Karp69ParallelProgramSchemata}. The corresponding coverability graph of the example Petri net in the Figure \ref{fig:example petri net} is in the Figure \ref{fig:example coverability graph}. The algorithm for generating the coverability graph of a Petri net \cite{Yen06PetriNets} is shown \vpageref[below]{alg:coverability_graph}.

\begin{figure}
  \centering
  \begin{tikzpicture}[node distance=8mm,-triangle 45]
    \tikzstyle{every node} = [rectangle,draw]
    \tikzstyle{label} = [draw=none]
    \node (1) {3,0,0,0};
    \node [below= of 1] (2) {2,1,1,1};
    \node [below= of 2] (3) {1,2,2,2};
    \node [below= of 3] (4) {0,3,3,3};
    \node [right= of 2] (5) {$3,\omega,\omega,0$};
    \node [below= of 5] (6) {$2,\omega,\omega,1$};
    \node [below= of 6] (7) {$1,\omega,\omega,2$};
    \node [below= of 7] (8) {$1,\omega,\omega,3$};
    \draw (1) edge node [label,right] {$t_1$} (2);
    \draw (2) edge node [label,right] {$t_1$} (3);
    \draw (3) edge node [label,right] {$t_1$} (4);
    \draw (2) edge node [label,above] {$t_2$} (5);
    \draw (3) edge node [label,above] {$t_2$} (6);
    \draw (4) edge node [label,above] {$t_2$} (7);
    \draw (5) edge [bend right=30] node [label,left] {$t_1$} (6);
    \draw (6) edge [bend right=30] node [label,left] {$t_1$} (7);
    \draw (7) edge [bend right=30] node [label,left] {$t_1$} (8);
    \draw (6) edge [bend right=30] node [label,right] {$t_2$} (5);
    \draw (7) edge [bend right=30] node [label,right] {$t_2$} (6);
    \draw (8) edge [bend right=30] node [label,right] {$t_2$} (7);
  \end{tikzpicture}
  \caption{An example coverability graph}
  \label{fig:example coverability graph}
\end{figure}

\begin{algorithm}
  \caption{Coverability graph algorithm}\label{alg:coverability_graph}
  \begin{algorithmic}[1]
    \Procedure{CoverabilityGraph}{marked Petri net $\mathcal P = (P, T, \varphi, \mu_0)$}
      \State $\text{create a node $\mu_{init}$ such that $\mu_{init} = \mu_0$ and mark it as `new'}$
      \While{$\text{there is a `new' node $\mu$}$}
        \For{$\text{each transition $t$ enabled at $\mu$}$}
          \If{$\text{there is a node $\mu^\prime=\mu+\Delta t$}$}
            \State $\text{add an edge $\mu\xrightarrow{t}\mu^\prime$}$
          \ElsIf{$\text{there is a path $\mu_{init}\xrightarrow{*}\mu^{\prime\prime}\xrightarrow{*}\mu$ such that $\mu^{\prime\prime}<\mu+\Delta t$}$}
            \State $\text{add a `new' node $x$ with}$
            \State \hspace{\algorithmicindent}$\text{$x(p)=\omega$ if $\mu^{\prime\prime}(p)<(\mu+\Delta t)(p)$}$
            \State \hspace{\algorithmicindent}$\text{$x(p)=\mu^{\prime\prime}(p)$ otherwise}$
            \State $\text{add an edge $\mu\xrightarrow{t}x$}$
          \Else
            \State $\text{add a `new' node $x$ with $x=\mu+\Delta t$ and an edge $\mu\xrightarrow{t}x$}$
          \EndIf
        \EndFor
        \State $\text{mark $\mu$ with `old'}$
      \EndWhile
    \EndProcedure
  \end{algorithmic}
\end{algorithm}

\subsection{Analysis of behavioral properties} % (fold)
\label{sub:analysis_of_behavioral_properties}

Analysis of several behavioral properties is studied and following decidability problems are of special importance:

\subsubsection{The boundedness problem} % (fold)
\label{ssub:the_boundedness_problem}
  The boundedness problem is, given a marked Petri net $\mathcal P$, deciding whether $|R(\mathcal P)|$ is finite. This problem was first considered by Karp and Miller \cite{Karp69ParallelProgramSchemata}, where it was shown to be decidable using the technique of coverability graph analysis. A Petri net is unbounded iff an $\omega$ occurs in the corresponding coverability graph. The algorithm presented there was basically an unbounded search and consequently no complexity analysis was shown. Subsequently, a lower bound of $O(2^{cm})$ space was shown by Lipton in \cite{Lipton76Reachability}, where $m$ is the number of places in the Petri net and $c$ is a constant. Finally, an upper bound of $O(2^{cn\log{n}})$ space was given by Rackoff in \cite{Rackoff78Reachability}. Here, however, $n$ represents the size or number of bits in the problem instance and $c$ is a constant.
% subsubsection the_boundedness_problem (end)

\subsubsection{The covering problem} % (fold)
\label{ssub:the_covering_problem}
  The covering problem is, given a marked Petri net $\mathcal P$ and a marking $\mu$, deciding whether there exists $\mu^\prime\in R(\mathcal P)$ such that $\mu^\prime\geq\mu$. The complexity (both upper and lower bounds) of the covering problem can be derived along a similar line of that of the boundedness problem \cite{Rackoff78Reachability}.
% subsubsection the_covering_problem (end)

\subsubsection{The reachability problem} % (fold)
\label{ssub:the_reachability_problem}
  The reachability problem is, given a marked Petri net $\mathcal P$ and a marking $\mu$, deciding whether $\mu\in R(\mathcal P)$. This problem has attracted the most attention in the Petri net community. One reason is that the problem has many real-world applications; furthermore, it is the key to the solutions of several other Petri net problems. Before the decidability question of the reachability problem for general Petri nets was proven by Mayr in 1981 \cite{Mayr81PetriNetReachability}, a number of attempts had been made to investigate the problem for restricted classes of Petri nets, in hope of gaining more insights and developing new tools in order to conquer the general Petri net reachability problem. It should be noted that the technique of the coverability graph analysis does not answer the reachability problem as $\omega$ abstracts out the exact number of tokens that a place can accumulate, should the place be potentially unbounded.
% subsubsection the_reachability_problem (end)

\subsubsection{The containment problem} % (fold)
\label{ssub:the_containment_problem}
  The containment problem is, given two marked Petri nets $\mathcal P_1$ and $\mathcal P_2$, deciding whether $R(\mathcal P_1)\subseteq R(\mathcal P_2)$. In the late 1960's, Rabin first showed the containment problem for Petri nets to be undecidable. Even though the original work of Rabin have never been published, a new proof based on Hilbert's Tenth Problem \cite{Davis73Hilbert} was presented at MIT in 1972 \cite{Baker73PetriNetContainment}.
% subsubsection the_containment_problem (end)

\subsubsection{The equivalence problem} % (fold)
\label{ssub:the_equivalence_problem}
  The equivalence problem: given two marked Petri nets $\mathcal P_1$ and $\mathcal P_2$, deciding whether $R(\mathcal P_1) = R(\mathcal P_2)$. In 1975, Hack \cite{Hack1976PetriNetEquivalence} extended Rabin's result of the containment problem by showing the equivalence problem to be undecidable as well. The proof was also based on Hilbert's Tenth Problem.
% subsubsection the_equivalence_problem (end)

\subsubsection{The liveness problem} % (fold)
\label{ssub:the_liveness_problem}
  The liveness problem: given a marked Petri net $\mathcal P$, deciding whether for every $t\in T, \mu\in R(\mathcal P)$ there exists a sequence of transitions $\sigma$ such that $\mu\xrightarrow{\sigma t}$, i.e. $t$ is enabled after firing $\sigma$ from $\mu$. In \cite{Hack74PetriNetLiveness}, several variants of the reachability problem were shown to be recursively equivalent. Among them is the single-place zero reachability problem, i.e. the problem of determining whether a marking with no tokens in a designated place can be reached. Hack also showed the single-place zero reachability problem to be recursively equivalent to the liveness problem, which is then as well decidable.
% subsubsection the_liveness_problem (end)

% subsection analysis_of_behavioral_properties (end)

\subsection{Backwards-compatible extensions} % (fold)
\label{sub:backwards_compatible_extensions}

Conventional Petri nets have several limitations such as the inability to test for zero tokens in a place. Models often tend to become large, with no support for structuring. To remedy these weaknesses, a number of extended Petri net models have been proposed in the literature. Some of these extensions only increase the modeling convenience by making the model more expressive while they do not actually increase the power of the basic model and are completely backwards-compatible, e.g. colored Petri nets.

\subsubsection{Colored Petri nets} % (fold)
\label{ssub:colored_petri_nets}

A colored Petri net (CPN) has each token attached with a color, indicating the identity of the token. Moreover, each place and each transition has attached a set of colors. A transition can fire with respect to each of its colors. By firing a transition, tokens are removed from the input places and added to the output places in the same way as that in original Petri nets, except that a functional dependency is specified between the color of the transition firing and the colors of the involved tokens. The color attached to a token may be changed by a transition firing and it often represents a complex data-value. CPNs lead to compact net models by using of the concept of colors.

% subsubsection colored_petri_nets (end)

% subsection backwards_compatible_extensions (end)

\subsection{Turing complete extensions} % (fold)
\label{sub:turing_complete_extensions}

The computational power of conventional or backwards-compatible Petri nets such as CPNs is strictly weaker than that of Turing machines \cite{Yen06PetriNets}. From a theoretical viewpoint, the limitation of Petri nets is precisely due to the lack of abilities to test potentially unbounded places for zero and then act accordingly. With zero-testing capabilities, it is fairly easy to show the equivalence to register machines, which are Turing equivalent. We will mention several extensions, which add properties that cannot be modeled in the original Petri net.

\subsubsection{Petri nets with inhibitor arcs} % (fold)
\label{ssub:petri_nets_with_inhibitor_arcs}

The inhibitor arc connects an input place to a transition. The presence of an inhibitor arc connecting an input place to a transition changes the transition enabling conditions. In the presence of the inhibitor arc, a transition is regarded as enabled if each input place, connected to the transition by a normal arc, contains at least the number of tokens equal to the weight of the arc, and no tokens are present on each input place connected to the transition by the
inhibitor arc. The transition firing rule is the same for normally connected places. The firing, however, does not change the marking in the inhibitor arc connected places. 

% subsubsection petri_nets_with_inhibitor_arcs (end)

\subsubsection{Time petri nets} % (fold)
\label{ssub:time_petri_nets}

Time Petri nets (TPN) are Petri nets in which each transition $t$ has attached two times $a\leq b$. Assuming that $t$ was last enabled at time $c$, then $t$ may fire only during the interval $[c+a, c+b]$ and must fire at time $c+b$ at the latest unless it is disabled before by the firing of another transition. Firing a transition takes no time.

% subsubsection time_petri_nets (end)

\subsubsection{Timed Petri nets} % (fold)
\label{ssub:timed_petri_nets}

Timed Petri nets (TdPN) are obtained from Petri nets by associating a firing time (delay) to each transition of the net. Moreover, each token has an age property, so the marking of a place $p$ is a finite multiset of ages. The precondition of a transition with input place $p$ is an interval. A transition is enabled if for every input place there exists an appropriate token, i.e. its age is included in the interval.

% subsubsection timed_petri_nets (end)

\subsubsection{Prioritized Petri nets} % (fold)
\label{ssub:prioritized_petri_nets}

Transactions in prioritized Petri nets have input arcs of two types: normal and prioritized. A place with a token and several transitions enabled from this place will fire the transition with a priority arc first. If there are more than one priority arc outgoing from a place which causes that more than one transition is enabled, then the firing choice is nondeterministic.

% subsubsection prioritized_petri_nets (end)

\subsubsection{Maximal parallel Petri nets} % (fold)
\label{ssub:maximal_parallel_petri_nets}

Under the maximal parallel semantics, maximal sets of simultaneously enabled rules are fired. They are of interest due to their close connections with the model of P systems.

% subsubsection maximal_parallel_petri_nets (end)

% subsection turing_complete_extensions (end)

\subsection{Other extensions} % (fold)
\label{sub:other_extensions}

While the above extended Petri nets are powerful enough to simulate Turing machines, all nontrivial behavioral properties for such Petri nets become undecidable. A natural and interesting question to ask is: are there Petri nets whose powers lie between conventional Petri nets and Turing machines? As it turns out, the quest for such weaker extensions has attracted considerable attention in recent years.

\subsubsection{Reset nets} % (fold)
\label{ssub:reset_nets}

A reset arc does not impose a precondition on firing, and empties the place when the transition fires. This makes reachability and boundedness undecidable, while some other properties, such as termination, coverability remain decidable.

% subsubsection reset_nets (end)

\subsubsection{Transfer nets} % (fold)
\label{ssub:transfer_nets}

Petri nets with transfer arcs (transfer nets) can contain transitions that can also consume all the tokens present in one place regardless of the actual number of tokens and move them to another place. The termination and coverability remain decidable and reachability undecidable, on the other hand, in this case the boundedness is decidable \cite{Dufourd98Reset}.

% subsubsection transfer_nets (end)

% subsection other_extensions (end)


% section petri_nets (end)

\section{Vector addition systems} % (fold)
\label{sec:vector_addition_systems}
Vector addition systems were introduced by Karp and Miller \cite{Karp69ParallelProgramSchemata}, and were later shown by Hack \cite{Hack74PetriVAS} to be equivalent to Petri nets.

\begin{definition}
  A {\bf vector addition system} (VAS) is a pair $G = (x, W)$, where $x\in \mathbb N^n$ is an initial vector and $W\subseteq \mathbb Z^n$ is a finite set of vectors, where $n>0$ is called the dimension of VAS.
\end{definition}

The initial vector is seen as the initial values of multiple counters and the vectors in $W$ are seen as actions that update the counters. These counters may never drop below zero. 

\begin{definition}
  The {\bf reachability set} of the VAS $G = (x,W)$ is the set \linebreak $R(G) = \{z | \exists v_1,\ldots,v_j\in W: z=x+v_1+\ldots+v_j \wedge \forall 1\leq i\leq j: x+v_1+\ldots+v_i\geq 0\}$.
\end{definition}

\begin{definition}
  A {\bf vector addition system with states} (VASS) is a tuple $G = (x, W, Q, T, p_0)$, where:
  \begin{itemize}
    \item $(x, W)$ is a vector addition system,
    \item $Q$ is a finite set of states,
    \item $T$ is a finite set of transitions of the form $p\rightarrow(q,v)$, where $v\in W$ and $p,q\in Q$ are states,
    \item $p_0\in Q$ is the starting state.
  \end{itemize}
\end{definition}

The transition $p\rightarrow(q,v)$ can be applied at vector $y$ in state $p$ and yields the vector $y+v$ in state $q$, provided that $y+v\geq 0$.

\begin{example}
  For the Petri net in the Figure \ref{fig:example petri net}, the corresponding VAS $(x,W)$ is:
  \begin{itemize}
    \item $x=(3,0,0,0)$,
    \item $W=\{(-1,1,1,1),(1,0,0,-1)\}$.
  \end{itemize}
\end{example}

It is known \cite{Hack74PetriVAS} that Petri nets, VAS and VASS are computationally equivalent.


% section vector_addition_systems (end)

\section{Büchi automaton} % (fold)
\label{sec:buchi_automaton}
No need to include Buchi yet.
% section buchi_automaton (end)

\section{Calculi of looping sequences} % (fold)
\label{sec:calculi_of_looping_sequences}
% !TEX root = ../diz.tex
Another tool for describing biological membranes is the formalism Calculus of Looping Sequences (CLS) \cite{Barbuti07CLS}.

In the last few years many formalisms originally developed by computer scientists to model systems of interacting components have been applied to biology. Here, we can mention Petri nets (see Section \ref{sec:petri_nets}). Others, such as P systems (see Chapter \ref{cha:p_systems}), have been proposed as biologically inspired computational models and have been later applied to the description of biological systems.

Many of these models either offer only very low-level interaction primitives or they are specialized to the description of some particular kinds of phenomena such as membrane interactions or protein interactions. Finally, P Systems have a simple notation and are not specialized to the description of a particular class of systems, but they are still not completely general. For instance, it is possible to describe biological membranes and the movement of molecules across membranes, and there are some variants able to describe also more complex membrane activities. However, the formalism is not so flexible to allow describing easily new activities observed on membranes without extending the formalism to model such activities.

For these reasons there was a new formalism called Calculus of Looping Sequences introduced.

CLS is a formalism based on term rewriting with some features, such as a commutative parallel composition operator, and some semantic means, such as bisimulations, which are common in process calculi. This permits to combine the simplicity of notation of rewriting systems with the advantage of a form of compositionality.

A CLS model consists of a term and a set of rewrite rules. The term is intended to represent the structure of the modeled system and the rewrite rules to represent the events that may cause the system to evolve.

We start with defining the syntax of terms. We assume a possibly infinite alphabet $\Sigma$ of symbols.

\begin{definition}
  Terms $T$ and sequences $S$ of CLS are given by the following grammar:
  \begin{align*}
    T ::= S \bigpipe (S)^L\rfloor T \bigpipe T|T\\
    S ::= \eps \bigpipe a \bigpipe S\cdot S
  \end{align*}
  where $a\in \Sigma$ and $\eps$ represents the empty sequence. We denote with $\mathcal T$ the infinite set of terms and with $\mathcal S$ the infinite set of sequences.
\end{definition}

\begin{figure}
  \centering
  \begin{tikzpicture}[node distance=16mm,-triangle 45]
    \node(a){a};
    \node[right=12mm of a](f){f};
    \node[right= of f](g){g};
    \node[right=12mm of g](c){c};
    \node[below=10mm of f](d){d};
    \node[right= of d](e){e};
    \node[below right=16mm and 8mm of d.center](b){b};
    \node[below=10mm of b](description){A representation of the term $(a\cdot b\cdot c)^L\rfloor ((d\cdot e)^L | f\cdot g)$.};
    \draw (a) edge[bend right=60] (b);
    \draw (b) edge[bend right=60] (c);
    \draw (c) edge[bend right=60] (a);
    \draw (d) edge[bend right=60] (e);
    \draw (e) edge[bend right=60] (d);
    \draw (f) edge (g);
  \end{tikzpicture}
  \caption{Example CLS}
  \label{fig:example cls}
\end{figure}

In CLS we have a sequencing operator $\_\cdot\_$, a looping operator $(\_)^L$, a parallel composition operator $\_|\_$ and a containment operator $\_\rfloor\_$. Sequencing can be used to concatenate elements of the alphabet $\Sigma$. The empty sequence $\eps$ denotes the concatenation of zero symbols. A term can be either a sequence or a looping sequence (that is the application of the looping operator to a sequence) containing another term, or the parallel composition of two terms. By definition, looping and containment are always applied together, hence we can consider them as a single binary operator $(\_)^L\rfloor\_$ which applies to one sequence and one term.

\begin{example}
  In the Figure \ref{fig:example cls} we show an example of CLS and its visual representation. The same structure may be represented by syntactically different terms, e.g. $(b\cdot c\cdot a)^L\rfloor (f\cdot g | (e\cdot d)^L)$. We introduce a structural congruence relation to identify such terms.
\end{example}

\begin{definition}
  The structural congruence relations $\equiv_S$ and $\equiv_T$ are the least congruence relations on sequences and on terms, respectively, satisfying the following rules:
  \begin{itemize}
    \item $S_1\cdot(S_2\cdot S_3)\equiv_S (S_1\cdot S_2)\cdot S_3$
    \item $S\cdot\eps\equiv_S \eps\cdot S\equiv_S S$
    \item $S_1\equiv_S S_2$ implies $S_1\equiv_T S_2$ and $(S_1)^L\rfloor T\equiv_T (S_2)^L\rfloor T$
    \item $T_1|T_2\equiv_T T_2|T_1$
    \item $T_1|(T_2|T_3)\equiv_T (T_1|T_2)|T_3$
    \item $T|\eps\equiv_T T$
    \item $(\eps)^L\rfloor\eps\equiv_T\eps$
    \item $(S_1\cdot S_2)^L\rfloor T\equiv_T (S_2\cdot S_1)^L\rfloor T$
  \end{itemize}
\end{definition}

Note that the last rule does not introduce the commutativity of sequences, but only says that looping sequences can rotate.

What could look strange in CLS is the use of looping sequences for the description of membranes, as sequencing is not a commutative operation and this do not correspond to the usual fluid representation of membrane surface in which objects can move freely. What one would expect is to have a multiset or a parallel composition of objects on a membrane. For this reason, a variant called CLS+ was introduced in \cite{Milazzo07CLS}, in which the looping operator can be applied
to a parallel composition of sequences.

\begin{definition}
  Terms $T$, branes $B$ and sequences $S$ of CLS+ are given by the following grammar:
  \begin{align*}
    T ::= S \bigpipe (B)^L\rfloor T \bigpipe T|T\\
    B ::= S \bigpipe B|B\\
    S ::= \eps \bigpipe a \bigpipe S\cdot S
  \end{align*}
\end{definition}

The structural congruence relation of CLS+ is a trivial extension of the one of CLS. The only difference is that commutativity of branes replaces rotation of
looping sequences. CLS+ models can be translated into CLS models, while preserving the semantics of the model \cite{Barbuti07CLS}. Moreover, in the same paper, a traslation of maximal parallel P system to CLS is shown.

% section calculi_of_looping_sequences (end)

\section{Graph theory} % (fold)
\label{sec:graph_theory}
% !TEX root = ../diz.tex
\begin{definition}
  A {\em graph} is a pair $G = (V,E)$ of sets such that $E\subseteq V\times V$ and $V\cap E = \emptyset$. The elements of $V$ are the vertices (or nodes) of the graph $G$, the elements of $E$ are its edges.
\end{definition}

The vertex set of a graph $G$ is denoted by $V(G)$, its edge set as $E(G)$.

\begin{definition}
  The {\em order of the graph} $G$ is the number of its vertices, denoted by $|G|$. Graphs are (in)finite iff their order is (in)finite.
\end{definition}

\begin{definition}
  A vertex $v\in V$ is {\em incident} with an edge $e\in E$ iff $v\in e$, i.e. either $e=(v,y)$, where $y\in V$ or $e=(x,v)$, where $x\in V$.
\end{definition}

\begin{definition}
  Two vertices $x,y\in V(G)$ are {\em adjacent} iff $(x,y)\in E(G)$.
\end{definition}

\begin{definition}
  A {\em path} is a non-empty graph $P=(V,E)$, where $V=\{x_0, x_1, \ldots, x_k\}$ and $E=\{(x_i,x_{i+1})|0\leq i < k\}$ and the $x_i$ are all distinct.
\end{definition}

\begin{definition}
  If $P=(V,E)$ is a path where $V=\{x_0, x_1, \ldots, x_k\}$ and $|V|\geq 3$, then the graph $C = (V,E\cup\{(x_k,x_0)\})$ is called a {\em cycle}. 
\end{definition}

\begin{definition}
  A graph $G=(V,E)$ is a {\em subgraph} of a graph $G^\prime = (V^\prime, E^\prime)$ iff $V\subseteq V^\prime$ and $E\subseteq E^\prime$. We denote it by $G\subseteq G^\prime$ and also say that $G^\prime$ contains $G$.
\end{definition}

\begin{definition}
  A non-empty graph $G$ is called {\em connected} iff any two of its vertices are linked by a path in $G$.
\end{definition}

\begin{definition}
  A graph not containing any cycle is called a {\em forest}.
\end{definition}

\begin{definition}
  A connected forest is called a {\em tree}.
\end{definition}

Sometimes it is convenient to consider one vertex of a tree as special. Such a vertex is then called the root of this tree.

\begin{definition}
  A tree $T$ with a fixed root is called a {\em rooted tree}. The root node of $T$ is denoted by $r_T$.
\end{definition}

\begin{definition}
  Let $d$ be a node of a non-root tree $T$, i.e. $d\in V(T)\setminus \{r_T\}$. As $T$ is a tree, there is a unique path from $d$ to $r_T$ \cite{Diestel97Graphs}. The node adjacent to $d$ on that path is also unique and is called a {\em parent node} of $d$ and is denoted by $parent_T(d)$.
\end{definition}

\begin{definition}
  Let $T_1, T_2$ be two rooted trees. A bijection $f: V(T_1)\rightarrow V(T_2)$ is an {\em isomorphism} iff $\forall x,y\in V(T): (x,y)\in E(T_1)\Leftrightarrow (f(x), f(y))\in E(T_2)$.
\end{definition}

Rooted trees $T_1$ and $T_2$ are called isomorphic if there exists an isomorphism on their nodes.


% section graph_theory (end)

\section{Bisimulations} % (fold)
\label{sec:bisimulations}
% !TEX root = ../diz.tex
For proving equivalence between two computation models, a notion of bisimulation is essential. Specifically, there are multiple notions that represent various equivalences. Definitions in this section are taken from \cite{DeNicola95Bisimulations}.
\begin{definition}
  A {\bf state transition system} is a pair $(S, \rightarrow)$, where $S$ is a set of states and $\rightarrow\subseteq S\times S$ is a binary transition relation over $S$.
\end{definition}
  If $p,q\in S$, then $(p,q)\in \rightarrow$ is usually written as $p\rightarrow q$. This represents the fact that there is a transition from state $p$ to state $q$.

\begin{definition}
  A {\bf labeled state transition system} (LTS) is a tuple $(S, A, \rightarrow)$, where $S$ is a set of states, $A$ is a set of labels and $\rightarrow\subseteq S\times A\times S$ is a ternary transition relation.
\end{definition}

If $p,q\in S$ and $a\in A$, then $(p,a,q)\in \rightarrow$ is usually written as $p\xrightarrow{a} q$. This represents the fact that there is a transition from state $p$ to state $q$ with a label $a$.

\begin{definition}
  Let $(S_1, A, \rightarrow)$ and $(S_2, A, \rightarrow)$ be two labeled transition systems.
  A {\bf simulation} is a binary relation $R\subseteq S_1\times S_2$ such that if $(s_1,s_2)\in R$ then for each $s_1\xrightarrow{a} t_1$ there is some $s_2\xrightarrow{a} t_2$ such that $(t_1, t_2)\in R$.
\end{definition}

\begin{definition}
  Let $(S_1, A, \rightarrow)$ and $(S_2, A, \rightarrow)$ be two labeled transition systems.
  A {\bf bisimulation} is a binary relation $R\subseteq S_1\times S_2$ such that if $(s_1,s_2)\in R$ then:
  \begin{enumerate}
    \item for each $s_1\xrightarrow{a} t_1$ there is some $s_2\xrightarrow{a} t_2$ such that $(t_1, t_2)\in R$,
    \item for each $s_2\xrightarrow{a} t_2$ there is some $s_1\xrightarrow{a} t_1$ such that $(t_1, t_2)\in R$.
  \end{enumerate}
\end{definition}

Sometimes, we want to allows systems to perform internal (silent) steps, of which the impact is considered unobservable. There are multiple notions of the bisimulation. The definition above is called a strong bisimulation.

\begin{definition}
  A {\bf labeled transition system with silent actions} is a triple $(S, A, \rightarrow)$, where $S$ is a set of states, $A$ is a set of actions, also a silent action $\tau$ is assumed that is not in $A$ and $\rightarrow\subseteq (A\cup\tau)\times S$ is the transition relation.
\end{definition}

We use $A_\tau$ to denote $A\cup\{\tau\}$.
By $\xRightarrow{\tau^*}$ we denote the transitive and reflexive closure of $\xrightarrow{\tau}$. For $a\in A$ we define the relation $\xRightarrow{a}$ on $S$ by $r\xRightarrow{a}s$ iff there exists $r^\prime, s^\prime\in S$ such that $r \xRightarrow{\tau^*} r^\prime \xrightarrow{a} s^\prime \xRightarrow{\tau^*} s$.

\begin{definition}
  A {\bf run} of a labeled transition system is a finite, non-empty alternating sequence $\rho = s_0a_0s_1a_1\ldots s_{n-1}a_{n-1}s_n$ of states and actions, beginning and ending with a state such that for $0\leq i<n: s_i\xrightarrow{a_i}s_{i+1}$.
\end{definition}

We also say that $\rho$ is a run from $s_0$. We denote $first(\rho)=s_0$ and $last(\rho)=s_n$.

\begin{definition}
  A relation $R\subseteq S\times S$ is called a {\bf branching bisimulation} if it is symmetric and satisfies the following transfer property: if $rRs$ and $r\xrightarrow{a} r^\prime$ then either $a=\tau$ and $r^\prime Rs$ of there exist $s^\prime,s^{\prime\prime}\in S$ such that $s \xRightarrow{\tau^*} s^\prime \xrightarrow{a} s^{\prime\prime}$, $rRs^\prime$ and $r^\prime Rs^{\prime\prime}$.
\end{definition}

\begin{figure}
  \centering
  \begin{minipage}{.5\textwidth}
    \begin{tikzpicture}[node distance=8mm,-triangle 45]
      \tikzstyle{every node} = [draw=none]
      \node (r1) {$r$};
      \node [right= of r1] (r2) {$r^\prime$};
      \node [below= of r1] (s1) {$s$};
      \node [right=16mm of s1] (or) {or};
      \node [right= of or] (s2) {$s$};
      \node [right= of s2] (s3) {$s^\prime$};
      \node [right= of s3] (s4) {$s^{\prime\prime}$};
      \node [above= of s3] (r3) {$r$};
      \node [right= of r3] (r4) {$r^\prime$};
      \draw (r1) edge node [label,above] {$\tau$} (r2);
      \draw (r1) edge[-] (s1);
      \draw (r2) edge[-,dashed] (s1);
      \draw (r3) edge node [label,above] {$a$} (r4);
      \draw (s2) edge[double,->] node [label,above] {$\tau^*$} (s3);
      \draw (s3) edge node [label,above] {$a$} (s4);
      \draw (r3) edge[-] (s2);
      \draw (r3) edge[-,dashed] (s3);
      \draw (r4) edge[-,dashed] (s4);
    \end{tikzpicture}
    \captionof{figure}{Transfer diagram for branching bisimulation}
    \label{fig:transfer diagram for branching bisimulation}
  \end{minipage}
  \hspace{.08\textwidth}
  \begin{minipage}{.3\textwidth}
    \begin{tikzpicture}[node distance=8mm]
      \tikzstyle{every node} = [draw=none]
      \node (r1) {$r$};
      \node [right= of r1] (r2) {$r^\prime$};
      \node [below= of r1] (s1) {$s$};
      \node [right= of s1] (s2) {$s^\prime$};
      \draw (r1) edge[double,->] node [label,above] {$a$} (r2);
      \draw (s1) edge[double,->] node [label,above] {$a$} (s2);
      \draw (r1) edge[-] (s1);
      \draw (r2) edge[-,dashed] (s2);
    \end{tikzpicture}
    \captionof{figure}{Transfer diagram for weak bisimulation}
    \label{fig:transfer diagram for weak bisimulation}
  \end{minipage}
\end{figure}

\begin{example}
  The diagrams shown in the Figure \ref{fig:transfer diagram for branching bisimulation} summarize the main transfer properties of branching bisimulation. We have used the dashed lines to represent the relations that have to be established in order to conclude that the two states connected by the plain line are equivalent.
\end{example}

We could have strengthened the above definition of branching bisimulation by requiring all intermediate states in $s\xRightarrow{\tau^*}s^\prime$ to be related with $r$. However, that would lead to the same equivalence relation \cite{DeNicola95Bisimulations}. 

\begin{definition}
  A relation $R\subseteq S\times S$ is called a {\bf weak bisimulation} if it is symmetric and satisfies the following transfer property: if $rRs$ and $r\xRightarrow{a} r^\prime$ then there exist $s^\prime\in S$ such that $s \xRightarrow{a} s^\prime$ and $r^\prime Rs^{\prime}$.
\end{definition}

This means that two states are considered equivalent if they lead via the same sequences of visible actions to equivalent states. The intermediate states are not questioned. The diagram in the Figure \ref{fig:transfer diagram for weak bisimulation} summarizes the transfer property for the weak bisimulation. We have used the same notational conventions of the Figure \ref{fig:transfer diagram for branching bisimulation}.

\begin{figure}
  \centering
  \begin{minipage}{.4\textwidth}
    \centering
    \begin{tikzpicture}[level distance=2cm,sibling distance=3cm,-triangle 45]
      \tikzstyle{every node} = [circle,draw]
      \node (Root) {a}
        child {
          node {}
          child {
            node {}
            edge from parent node [above=-5mm,sloped,draw=none] {win car}
          }
          child {
            node {}
            edge from parent node [above=-7mm,sloped,draw=none] {win flowers}
          }
          edge from parent node [left,draw=none] {open door}
        };
    \end{tikzpicture}  
    \captionof{figure}{Example run of a labeled transition system}
    \label{fig:example run of a labeled transition system}
  \end{minipage}
  \hspace{.08\textwidth}
  \begin{minipage}{.5\textwidth}
    \centering
    \begin{tikzpicture}[level distance=2cm,sibling distance=3cm,-triangle 45]
      \tikzstyle{every node} = [circle,draw]
      \node (Root) {}
        child {
          node {}
          child {
            node {}
            edge from parent node [left,draw=none] {win car}
          }
          edge from parent node [left,draw=none] {open door}
        }
        child {
          node {}
          child {
            node {}
            edge from parent node [left,draw=none] {win flowers}
          }
          edge from parent node [right,draw=none] {open door}
        };
    \end{tikzpicture}  
    \captionof{figure}{Example run of a labeled transition system with hidden branching before the first action}
    \label{fig:example run of a labeled transition system with hidden branching before the first action}
  \end{minipage}
\end{figure}

\begin{example}
  In the Figure \ref{fig:example run of a labeled transition system} a participant in a contest opens a door and then it is decided if he wins a car or flowers. In the Figure \ref{fig:example run of a labeled transition system with hidden branching before the first action} the price is decided beforehand so when a participant opens a door, he can directly see his outcome. We can model these events (open door, win car, win flowers) as actions of a labeled transition system.

  The contestant is not interested in the inner states of the system, he only sees these events. For him, the two systems are bisimular in terms of the weak bisimulation.

  In the state right after opening the door and before the price is given there is no way for him knowing the outcome. However, in the system from the Figure \ref{fig:example run of a labeled transition system with hidden branching before the first action} some people who organize the contest already know what is the outcome. For them, the system is not bisimilar in terms of the branching bisimulations.
\end{example}
% section bisimulations (end)
