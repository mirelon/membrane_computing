% !TEX root = diz.tex
\chapter*{Abstract}
\begin{description} \itemsep1pt \parskip0pt \parsep0pt
  \item[Author:] \mfauthor
  \item[Title:] \mftitle
  \item[University:] Comenius University in Bratislava
  \item[Faculty:] Faculty of Mathematics, Physics and Informatics
  \item[Department:] Department of Applied Informatics
  \item[Supervisor:] \mfadvisor
\end{description}

This work discusses the research in the membrane systems, an emerging field of natural computing. Many variants of membranes systems have already been studied, most of them uses parallel rewriting and are computationally complete. Various sequential models have been proposed, however, in many cases they are weaker than their parallel variant.

We propose some other sequential models, provide universality proofs for them and suggest further research.

\begin{description}
  \item[Keywords:] Computation models inspired by biology, Membrane systems, P systems
\end{description}

\chapter*{Abstrakt}
\begin{description} \itemsep1pt \parskip0pt \parsep0pt
  \item[Autor:] \mfauthor
  \item[Názov dizertačnej práce:] \mftitle
  \item[Škola:] Univerzita Komenského v Bratislave
  \item[Fakulta:] Fakulta matematiky, fyziky a informatiky
  \item[Katedra:] Katedra aplikovanej informatiky
  \item[Vedúci dizertačnej práce:] \mfadvisor
  \item \mfplacedate
\end{description}

V tejto práci sa zaoberáme výskumom v oblasti membránových systémov. Veľa variantov už bolo preskúmaných, väčšinou pri výpočte používajú paralelizmus a sú Turingovsky úplné. Navrhlo sa aj veľa sekvenčných modelov, ale väčšina z nich má slabšiu výpočtovú silu ako ich paralelný variant.

V druhej časti predkladáme niektoré sekvenčné modely, u ktorých dokazujeme Turingovu úplnosť a navrhujeme modely, ktoré plánujeme preskúmať v rámci dizertačnej práce.

\begin{description}
  \item[Kľúčové slová:] Výpočtové modely inšpirované biológiou, Membránové systémy, P systémy
\end{description}