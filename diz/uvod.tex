\addcontentsline{toc}{chapter}{Introduction}

There are a lot of areas in the theoretical computer science that are motivated by other science fields. Computation models motivated by biology forms a large group of them. They include neural networks, computational models based on DNA evolutionary algorithms, which have already found their use in computer science and proved that it is worth to be inspired by biology. L-systems are specialized for describing the growth of plants, but they have also found the applications in computer graphics, especially in fractal geometry. Other emerging areas are still awaiting for their more significant uses.

One of them is the membrane computing. It is relatively young field of natural computing - in comparison: neural networks have been researched since 1943 and membrane systems since 1998 \cite{Paun98}.

Membrane systems (P systems) are distributed parallel computing devices inspired by the structure and functionality of cells. Recently, many P system variants have been developed in order to simulate the cells more realistically or just to improve the computational power.

In chapter \ref{cha:preliminaries} we recall some computer science basics we will use through the work.
Natural computing including models inspired by nature will be introduced in chapter \ref{cha:natural_computing}. One of these models, P systems, with the current state of the research in their variants, is presented in chapter \ref{cha:p_systems}. Part \ref{prt:project_proposal} consists of two chapters, chapter \ref{cha:project_proposal} about dissertation project, overview of the current research and future plans and chapter \ref{cha:current_status_of_the_research} with details about our results.