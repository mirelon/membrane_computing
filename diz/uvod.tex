\addcontentsline{toc}{chapter}{Introduction}

About computational models inspired by biology. Neural networks, evolution algorithms, membrane systems.

V teoretickej informatike je veľa oblastí, ktoré sú motivované inými vednými
disciplínami. Veľkú skupinu tvoria modely motivované biológiou. Patria sem
napríklad neurónové siete, výpočtové modely založené na DNA, evolučné algoritmy, ktoré
si už našli svoje významné uplatnenie v informatike a dokázali, že sa oplatí
inšpirovať biológiou. L-systémy sú špecializované na popisovanie rastu rastlín, ale našli si uplatnenie aj v počítačovej grafike, konkrétne vo fraktálnej geometrii. Ďalšie rozvíjajúce sa oblasti ešte čakajú na svoje významnejšie uplatnenie.

Jednou z nich sú membránové systémy. Je pomerne mladá oblasť - prvý článok
bol publikovaný v roku 1998 (see \cite{Paun98})