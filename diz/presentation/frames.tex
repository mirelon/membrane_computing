\usepackage[utf8]{inputenc}
\usepackage{booktabs}% for rules in equations = example of computation step
\usepackage{slovak}
\usetheme{Warsaw}
\title{Biologicky motivované \\výpočtové modely}
\author{Michal Kováč}
\institute{FMFI UK}
\date{24.6.2013}
\begin{document}

\begin{frame}[t]
\titlepage
\end{frame}
\note{Vážená komisia, \dots}

\section*{Outline}
\begin{frame}
\tableofcontents
\end{frame}
\note{Prezentáciu začnem prehľadom existujúcich modelov, ktoré sú inšpirované biológiou.
Potom budem hovoriť o P systémoch, pretože im som sa najviac venoval.
Existuje množstvo variantov, o ktorých niečo poviem v ďalšej časti.
Prezentáciu zavŕšim predostretím plánov na dizertačnú prácu.}

\section{Prehľad problematiky} % (fold)
\label{sec:prehlad_problematiky}

\subsection{Prehľad modelov} % (fold)
\label{sub:prehlad_modelov}

\begin{frame}[t]\frametitle{Biologicky motivované \\výpočtové modely}
Modely vznikajú s dvoma účelmi:
\begin{itemize}
  \item simulácia biologických javov
  \item zdokonalenie informatických riešení
\end{itemize}
\end{frame}
\note{Tieto modely majú dvojaké uplatnenie.
Jednak v rámci biológie môžu slúžiť ako reálne modely správania sa živých systémov, na ktorých si možnosťami simulácie či verifikácie môžeme overovať správnosť nášho chápania ich biologickej činnosti, robiť virtuálne biologické experimenty.

Na druhej strane môžu slúžiť ako nové inšpiratívne výpočtové modely otvárajúce rad teoretických informatických otázok (napr. výpočtová sila) alebo ako modely na popis aj iných ako biologických systémov.}

\begin{frame}[t]\frametitle{Biologicky motivované \\výpočtové modely}
\begin{itemize}
  \item Neurónové siete (od 1943)
  \item Celulárne automaty (od 1948)
  \item Evolučné algoritmy (od 1954)
  \item L systémy (od 1968)
  \item P systémy (od 1998) \cite{Paun98}
  \item \dots
\end{itemize}
\end{frame}
\note{Dlho skúmané modely ako neurónové siete, celulárne automaty, evolučné algoritmy, či L systémy, si už našli svoje uplatnenie v praxi, kým membránové systémy sú ešte len v začiatkoch svojho vývoja.}

% subsection prehlad_modelov (end)

\subsection{P systémy} % (fold)
\label{sub:p_systemy}

\begin{frame}[t]\frametitle{Membránová štruktúra}
\includegraphics[width=0.7\textwidth]{membrane_structure.png}
\hfill
\includegraphics[width=0.3\textwidth]{membrane_tree.png}
\end{frame}
\note{Membránové systémy sú inšpirované bunkami. Základom je preto membránová štruktúra, ktorá pozostáva z regiónov, ktoré sú oddelené membránami. Tvorí to hierarchickú štruktúru, ktorá sa dá zobraziť ako strom.}

\begin{frame}[t]\frametitle{Obsah membrány}
\begin{itemize}
  \item multimnožina objektov
  \begin{itemize}
    \item $a\ |\ b\ |\ b$
  \end{itemize}
  \item prepisovacie pravidlá
  \begin{itemize}
    \item $a\ |\ b\ |\ b\rightarrow a\ |\ a\downarrow\ |\ a\uparrow\ |\ b\downarrow_6$
    \item $b \rightarrow a\ |\ \delta$
  \end{itemize}
\end{itemize}
\end{frame}
\note{V každej membráne je multimnožina objektov, ktorú zapisujeme takto.

Každá membrána má aj množinu prepisovacích pravidiel. Ľavá aj pravá strana pravidla pozostáva z multimnožiny objektov, pričom ľavá strana nesmie byť prázdna.

Posielanie objektov cez membránu sa uskutočňuje tak, že na pravej strane môžu mať objekty špecifikované, či ostanú v aktuálnom regióne, alebo pôjdu do rodiča, alebo do syna. Môže sa poslať do konkrétneho syna, alebo všetkým.}

\begin{frame}[t]\frametitle{P systém}

P systém definujeme ako $\Pi = (V, \mu, w_1, w_2,\dots , w_m, R_1, R_2, \dots , R_m)$, kde:
\begin{itemize}
  \item $V$ je abeceda objektov
  \item $\mu$ je membránová štruktúra
  \item $w_1, w_2, \dots w_m$ sú počiatočné multimnožiny v membránach $1\dots m$, $w_i\subseteq \mathbb{N}^V$
  \item $R_1, R_2, \dots , R_m$ sú množiny prepisovacích pravidiel v membránach $1\dots m$, pričom $$R_i\subseteq(\mathbb{N}^V\setminus 0^V)\times\mathbb{N}^{V\times(\{here,in,out\}\cup\{in_1,\dots in_m\})}$$.
\end{itemize}

\end{frame}
\note{}

\begin{frame}[t]\frametitle{Konfigurácia a krok výpočtu}
\begin{itemize}
  \item konfigurácia = membránová štruktúra + obsahy membrán
  \item krok výpočtu: maximálny paralelizmus
\end{itemize}

\begin{align*}
  a\ |\ b\ |\ b&\rightarrow c\\
  b &\rightarrow c\ |\ c\\
  a\ |\ a\ |\ b\ |\ b\\\midrule
  \onslide<2->{a\ |\ c\\\midrule
  a\ |\ a\ |\ c\ |\ c}
\end{align*}
\end{frame}
\note{}

\begin{frame}[t]\frametitle{Jazyk}
\begin{itemize}
  \item Parikhovo zobrazenie
  \item alternatíva: worm objects \cite{Mate02}
  \begin{itemize}
    \item namiesto multimnožín objektov sú v membránach multimnožiny stringov
    \item inšpirované DNA
  \end{itemize}

  \item generatívny mód
  \item akceptačný mód
\end{itemize}
\end{frame}
\note{}

% subsection p_systemy (end)

\subsection{Varianty} % (fold)
\label{sub:varianty}

\begin{frame}[t]\frametitle{Varianty pravidiel}
\begin{itemize}
  \item kontextové (PsRE)
  \item kooperatívne (PsRE)
  \item katalytické
  \begin{itemize}
    \item s 2 katalyzátormi (PsRE) \cite{Freund2005TwoCatalysts}
    \item s 1 katalyzátorom (otvorený problem)
    \item s 1 katalyzátorom a inhibítormi (PsRE) \cite{Ionescu:jucs_10_5:on_p_systems_with}
  \end{itemize}
  \item bezkontextové (PsCF) \cite{Sburlan05dragos}
  \item bezkontextové s inhibítormi (PsET0L) \cite{Ionescu:jucs_10_5:on_p_systems_with}
\end{itemize}
\end{frame}
\note{}

\begin{frame}[t]\frametitle{Varianty kroku výpočtu}
\begin{itemize}
  \item maximálny paralelizmus (PsRE)
  \item maximálny paralelizmus bez priorít (PsRE) \cite{Sosik:2002:WithoutPriorities}
  \item sekvenčný (vieme simulovať pomocou VASS, \cite{Dang:2005:Sequential})
  \item sekvenčný s prioritami (TODO)
  \item asynchrónny (TODO)
  \item minimálny paralelizmus (PsRE) \cite{Ciobanu:2007:MinimalParallelism}
  \item n-paralelizmus, max-n-paralelizmus, \dots
\end{itemize}
\end{frame}
\note{}

% subsection varianty (end)

% section prehlad_problematiky (end)

\section{Plány na dizertačnú prácu} % (fold)
\label{sec:plany_na_dizertacnu_pracu}

\subsection{Aktuálne riešené problémy} % (fold)
\label{sub:aktualne_riesene_problemy}

\begin{frame}[t]\frametitle{Sekvenčné P systémy}
\begin{itemize}
  \item nie sú univerzálne
  \item na univerzalitu treba:
  \begin{itemize}
    \item povoliť neobmedzené vytváranie membrán \cite{Dang:2005:Sequential}
    \item inhibítory
      \begin{itemize}
        \item publikuje sa
        \item Inhibiting the parallelism in P systems
        \item 2nd International Workshop on Hybrid Systems and Biology
      \end{itemize}
    \item iné rozšírenia (vacuum, ...)
    \item inšpirácie z výsledkov iných formalizmov
  \end{itemize}
\end{itemize}
\end{frame}
\note{}

% subsection aktualne_riesene_problemy (end)

\subsection{Ďalšie plány} % (fold)
\label{sub:dalsie_plany}

\begin{frame}[t]\frametitle{Ďalšie plány}
\begin{itemize}
  \item Preskúmať možnosti kombinovania ďalších variantov P systémov z hľadiska výpočtovej sily
  \begin{itemize}
    \item priestorové P systémy
    \item rozpadajúce sa objekty
    \item energie
  \end{itemize}
  \item Porovnať s inými formalizmami, napríklad Petriho siete / reaction systems / CLS / ...
  \item Nájsť nové varianty
\end{itemize}
\end{frame}
\note{}


\begin{frame}[t]\frametitle{Inšpirácie z výsledkov iných formalizmov}
\begin{itemize}
  \item Petriho siete
  \begin{itemize}
    \item nie sú univerzálne
    \item s inhibítormi áno
    \item ake iné varianty Petriho sietí ešte nikto nevyskúšal aplikovať v P systémoch?
  \end{itemize}
  \item CLS (Calculi of Looping Sequences)
  \begin{itemize}
    \item sekvenčný model, vie simulovať P systémy \cite{Barbuti07CLS}
  \end{itemize}
  \item Reakčné (alebo reaktívne?) systémy
\end{itemize}
\end{frame}
\note{}

\begin{frame}[t]\frametitle{Nové varianty}
Besozzi \cite{Besozzi:PhD:2004}: Dobrý variant by mal byť:
\begin{itemize}
  \item realistický
  \item univerzálny
  \item iredundantný
\end{itemize}
\end{frame}
\note{}

% subsection dalsie_plany (end)

% section plany_na_dizertacnu_pracu (end)

\begin{frame}[allowframebreaks]{Literatúra}
\bibliographystyle{apalike}
\bibliography{presentation.bib}
\end{frame}

\begin{frame}[plain]
\begin{center}
  Ďakujem za pozornosť
\end{center}
\end{frame}

\end{document}