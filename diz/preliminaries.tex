\chapter{Preliminaries} % (fold)
\label{cha:preliminaries}

\section{Formal languages theory} % (fold)
\label{sec:formal_languages_theory}

Our study is based on the classical theory of formal languages. We will recall some definitions:

\begin{definition}
An {\bf alphabet} is a finite nonempty set of symbols. Usually it is denoted by $\Sigma$ or $V$.
\end{definition}

\begin{definition}
A {\bf string} over an alphabet is a finite sequence of symbols from alphabet.
\end{definition}

We denote by $V^*$ the set of all strings over an alphabet $V$. By $V^+$ = $V^* - \{\eps\}$ we denote the set of all nonempty strings over V.

\begin{definition}
A {\bf language} over the alphabet $V$ is any subset of $V^*$.
\end{definition}

% section formal_languages_theory (end)

\section{Register machines} % (fold)
\label{sec:register_machines}

% We will use the notion of register machine as defined in our article

\begin{definition}
  A {\bf $n$-register machine} is a tuple $M = (n,P,i,h)$, where:
  \begin{itemize}
    \item $n$ is the number of registers,
    \item $P$ is a set of labeled instructions of the form $j : (op(r),k,l)$, where $op(r)$ is an operation on register $r$ of $M$, and $j$, $k$, $l$ are labels from the set $Lab(M)$ (which numbers the instructions in a one-to-one manner),
    \item $i$ is the initial label, and
    \item $h$ is the final label.
  \end{itemize}
\end{definition}

The machine is capable of the following instructions:
\begin{itemize}
  \item $(add(r),k,l)$ : Add one to the contents of register $r$ and proceed to instruction $k$ or to instruction $l$; in the deterministic variants usually considered in the literature we demand $k = l$.
  \item $(sub(r),k,l)$ : If register $r$ is not empty, then subtract one from its contents and go to instruction $k$, otherwise proceed to instruction $l$.
  \item $halt$ : This instruction stops the machine. This additional instruction can only be assigned to the final label $h$.
\end{itemize}

A deterministic $m$-register machine can analyze an input $(n_1,\dots,n_m)\in N_0^m$ in registers 1 to $m$, which is recognized if the register machine finally stops by the halt instruction with all its registers being empty (this last requirement is not necessary). If the machine does not halt, the analysis was not successful.

% section register_machines (end)

\section{Multisets} % (fold)
\label{sec:multisets}

\begin{definition}
A multiset over a set $X$ is a mapping $M: X\rightarrow \mathbb N$.
\end{definition}

We denote by $M(x), x\in X$ the multiplicity of $x$ in the multiset $M$.

\begin{definition}
The {\bf support} of a multiset $M$ is the set $supp(M)=\{x\in X|M(x)\geq 1\}$.
\end{definition}

It is the set of items with at least one occurence.

\begin{definition}
A multiset is {\bf empty} when it's support is empty.
\end{definition}

A multiset $M$ with finite support $X = \{x_1, x_2, \dots, x_n\}$ can be represented by the string $x_1^{M(x_1)}x_2^{M(x_2)}\dots x_n^{M(x_n)}$.

\begin{definition}
Multiset inclusion. We say that mutiset $M_1$ is included in multiset $M_2$ if $M_1(x)\leq M_2(x)\forall x \in X$. We denote it by $M_1\subseteq M_2$.
\end{definition}

\begin{definition}
The {\bf union} of two multisets $M_1\cup M_2 : X\rightarrow \mathbb N$ is defined as $(M_1\cup M_2)(x)=M_1(x)+M_2(x)$.
\end{definition}

\begin{definition}
The {\bf difference} of two multisets $M_1-M_2 : X\rightarrow \mathbb N$ is defined as $(M_1-M_2)(x)=M_1(x)-M_2(x)$.
\end{definition}

\begin{definition}
Product of multiset $M$ with natural number $n\in \mathbb N$ is $(n\cdot M)(x)=n\cdot M(x)$.	
\end{definition}

% section multisets (end)

\section{Multiset languages} % (fold)
\label{sec:multiset_languages}

The number of occurences of a given symbol $a\in V$ in the string $w\in V^*$ is denoted by $|w|_a$.

\begin{definition}
$\Psi_V(w)=(|w|_{a_1},|w|_{a_2},\dots,|w|_{a_n})$ is called a Parikh vector associated with the string $w\in V^*$, where $V=\{a_1,a_2,\dots a_n\}$.
\end{definition}

\begin{definition}
For a language $L\subseteq V^*$, $\Psi_V(L)=\{\Psi_V(w)|w\in L\}$ is the Parikh mapping associated with $V$.
\end{definition}

\begin{example}
Consider an alphabet $V=\{a,b\}$ and a language $L=\{a, ab, ba\}$.
$\Psi_V(L)=\{(1,0), (1,1)\}$. Notice that Parikh image of $L$ has only 2 element while $L$ has 3 elements.
\end{example}

\begin{definition}
If FL is a family of languages, by PsFL is denoted the family of Parikh images of languages in FL.
\end{definition}


% section multiset_languages (end)

% chapter preliminaries (end)