This chapter is about some basic notions of computer science which will be used through the work. We start by defining formal languages and basic models (grammars, machines) that define language families and end by defining multiset languages.

\section{Formal languages} % (fold)
\label{sec:formal_languages}

Our study is based on the classical theory of formal languages. We will recall some definitions:

\begin{definition}
An {\bf alphabet} is a finite nonempty set of symbols. Usually it is denoted by $\Sigma$ or $V$.
\end{definition}

\begin{definition}
A {\bf string} over an alphabet is a finite sequence of symbols from alphabet.
\end{definition}

We denote by $V^*$ the set of all strings over an alphabet $V$. By $V^+$ = $V^* - \{\eps\}$ we denote the set of all nonempty strings over V.

\begin{definition}
A {\bf language} over the alphabet $V$ is any subset of $V^*$.
\end{definition}

\begin{definition}
A {\bf family languages} is a set of languages.
\end{definition}


\section{Formal grammars} % (fold)
\label{sec:formal_grammars}

\begin{definition}
A {\bf formal grammar} is a tuple $G = (N,T,P,\sigma)$, where
\begin{itemize}
  \item $N, T$ are disjoint alphabets of non-terminal and terminal symbols,
  \item $\sigma\in N$ is the initial non-terminal,
  \item $P$ is a finite set of rewriting rules of the form $u\rightarrow v$, with $u\in (N\cup T)^*N(N\cup T)^*$ and $v\in (N\cup T)^*$.
\end{itemize}
\end{definition}

\begin{definition}
A {\bf rewriting step} in the grammar $G$ is a binary relation $\Rightarrow$ on $(N\cup T)^*$, where $x\Rightarrow y$ only if $\exists w_1, w_2\in (N\cup T)^+$ and a rule $u\rightarrow v \in P$ such that $x=w_1uw_2$ and $y=w_1vw_2$.
\end{definition}

\begin{definition}
Language defined by a grammar $G$ is a set $L(G)=\{w\in T^*|\sigma\Rightarrow w\}$.
\end{definition}

Languages that can be generated by a formal grammar are the recursively enumerable languages $RE$.

% section formal_languages (end)

% section formal_grammars (end)

\section{Chomsky hierarchy} % (fold)
\label{sec:chomsky_hierarchy}

In this section we introduce several well-known families of languages.

\begin{definition}
{\bf Regular grammar} is formal grammar, where rewriting rules are of the form $u\rightarrow v$, where $u\in N$ and $v\in T^*(N\cup \{\eps\})$.
\end{definition}

\begin{definition}
{\bf Regular languages} are languages generated by regular grammars. They are denoted $R$.
\end{definition}

\begin{definition}
{\bf Context-free grammar} is formal grammar, where rewriting rules are of the form $u\rightarrow v$, where $u\in N$ and $v\in (N\cup T)^*$.
\end{definition}

\begin{definition}
{\bf Context-free languages} are languages generated by context-free grammars. They are denoted $CF$.
\end{definition}

\begin{definition}
{\bf Context-sensitive grammar} is formal grammar, where rewriting rules are of the form $u\rightarrow v$, where $u\in N$ and $v\in (N\cup T)^*$.
\end{definition}

\begin{definition}
{\bf Context-sensitive languages} are languages generated by context-sensitive grammars. They are denoted $CS$.
\end{definition}

These families of languages forms the Chomsky hierarchy by means of inclusions: $R \subset CF \subset CS \subset RE$.

% section chomsky_hierarchy (end)

\section{Matrix grammars} % (fold)
\label{sec:matrix_grammars}

\begin{definition}
A {\bf matrix grammar} is a tuple $G = (N,T,M,\sigma)$, where:
\begin{itemize}
  \item $N, T$ are disjoint alphabets of non-terminal and terminal symbols,
  \item $\sigma\in N$ is the initial non-terminal,
  \item $M$ is a finite set of matrices, which are sequences of context-free rules of the form rewriting rules of the form $u\rightarrow v$, where $u\in N$ and $v\in (N\cup T)^*$.
\end{itemize}
\end{definition}

\begin{definition}
A {\bf rewriting step} $x\Rightarrow y$ holds only if there is a matrix $(u_1\rightarrow v_1, u_2\rightarrow v_2, \dots, u_n\rightarrow v_n) \in M$ such that for each $1\leq i\leq n$ the following holds: $x_i = x_i^{\prime}u_ix_i^{\prime\prime}$ and $x_{i+1} = x_i^{\prime}v_ix_i^{\prime\prime}$, where $x_i, x_i^{\prime}, x_i^{\prime\prime} \in (N\cup T)^*$ and $x_1 = x$ and $x_{n+1} = y$.
\end{definition}

\begin{example}
Consider the matrix grammar $G=(\{\sigma, X,Y\}, \{ a,b,c\}, M, \sigma)$, where $M$ contains three matrices: $[S\rightarrow XY], [X\rightarrow aXb, Y\rightarrow cY], [X\rightarrow ab, Y\rightarrow c]$. There are only context-free rules, yet the grammar generate the context-sensitive language $\{a^nb^nc^n|n\geq 1\}$.
\end{example}

The family of matrix grammars is denoted $MAT$.

It is known that $CF \subset MAT \subset RE$. Interestingly, $MAT \cap {a}^* \subset R$ (see \cite{Besozzi:PhD:2004}).

% section matrix_grammars (end)

\section{Register machines} % (fold)
\label{sec:register_machines}

% We will use the notion of register machine as defined in our article

\begin{definition}
  A {\bf $n$-register machine} is a tuple $M = (n,P,i,h)$, where:
  \begin{itemize}
    \item $n$ is the number of registers,
    \item $P$ is a set of labeled instructions of the form $j : (op(r),k,l)$, where $op(r)$ is an operation on register $r$ of $M$, and $j$, $k$, $l$ are labels from the set $Lab(M)$ (which numbers the instructions in a one-to-one manner),
    \item $i$ is the initial label, and
    \item $h$ is the final label.
  \end{itemize}
\end{definition}

The machine is capable of the following instructions:
\begin{itemize}
\item $(add(r),k,l)$ : Add one to the contents of register $r$ and proceed to instruction $k$ or to instruction $l$; in the deterministic variants usually considered in the literature we demand $k = l$.
\item $(sub(r),k,l)$ : If register $r$ is not empty, then subtract one from its contents and go to instruction $k$, otherwise proceed to instruction $l$.
\item $halt$ : This instruction stops the machine. This additional instruction can only be assigned to the final label $h$.
\end{itemize}

A deterministic $m$-register machine can analyze an input $(n_1,\dots,n_m)\in N_0^m$ in registers 1 to $m$, which is recognized if the register machine finally stops by the halt instruction with all its registers being empty (this last requirement is not necessary). If the machine does not halt, the analysis was not successful.

% section register_machines (end)

\section{Lindenmeyer systems} % (fold)
\label{sec:lindenmeyer_systems}

In 1968, a Hungarian botanist and theoretical biologist Aristid Lindenmeyer introduced \cite{Lindenmeyer68} a new string rewriting algorithm named Lindenmeyer systems (or L-systems for short). They are used by biologists and theoretical computer scientists to mathematically model growth processes of living organisms, especially plants. The difference with Chomsky grammars is that rewriting is parallel, not sequential.

The simplest version of L-systems assumes that the development of a cell is free of influence of other cells.
This type of L-systems is called $0L$ systems, where ``0'' stands for zero-sided communication between cells.

\begin{definition}
$0L$ system is a triple $(\Sigma, P, \omega)$, where $\Sigma$ is an alphabet, $\omega$ is a word over $\Sigma$ and $P$ is a finite set of rewriting rules of the form $a\rightarrow x$, where $a\in\Sigma, x\in\Sigma^*$.
\end{definition}

It is assumed there is at least one rewriting rule for each letter of $\Sigma$. $0L$ system works in parallel way, so all the symbols are rewritten in each step.

\begin{example}
Consider the $0L$ system with alphabet $\Sigma = \{a,b\}$, initial word $\omega = a$ and rewriting rules $P = \{a\rightarrow b, b\rightarrow ab\}$.
Since in this system there is exactly one rule for every letter of the alphabet, the rewriting is deterministic and the generated words will be $\{a, b, ab, bab, abbab, \dots \}$. 
\end{example}

$1L$ systems allows the rewriting rules to include context of size 1, so it allows for rules of type $yaz\rightarrow x$.

L-systems with tables ($T$) have several sets of rewriting rules instead of just one set. At one step of the rewriting process, rules belonging to the same set have to be applied. The biological motivation for introducing tables is that one may want different rules to take care of different environmental conditions (heat, light, etc.) or of different stages of development.

\begin{definition}
An extended ($E0L$) system is a pair $G_1 = (G, \Sigma_T)$, where $G = (\Sigma, P, \omega)$ is an $0L$ system, where $\Sigma_T \subseteq \Sigma$, referred to as the terminal alphabet. The language generated by $G_1$ is defined by $L(G_1) = L(G)\cap \Sigma_T^*$.
\end{definition}

Such languages are called $E0L$ languages. $E0L$ languages with tables are called $ET0L$ languages.

It is known that $CF \subset E0L \subset ET0L \subset CS$ (see section \ref{sec:chomsky_hierarchy} for definitions of $CF$ and $CS$).
% section lindenmeyer_systems (end)



\section{Multisets} % (fold)
\label{sec:multisets}

\begin{definition}
A multiset over a set $X$ is a mapping $M: X\rightarrow \mathbb N$.
\end{definition}

We denote by $M(x), x\in X$ the multiplicity of $x$ in the multiset $M$.

\begin{definition}
The {\bf support} of a multiset $M$ is the set $supp(M)=\{x\in X|M(x)\geq 1\}$.
\end{definition}

It is the set of items with at least one occurence.

\begin{definition}
A multiset is {\bf empty} when it's support is empty.
\end{definition}

A multiset $M$ with finite support $X = \{x_1, x_2, \dots, x_n\}$ can be represented by the string $x_1^{M(x_1)}x_2^{M(x_2)}\dots x_n^{M(x_n)}$.

\begin{definition}
Multiset inclusion. We say that mutiset $M_1$ is included in multiset $M_2$ if $M_1(x)\leq M_2(x)\forall x \in X$. We denote it by $M_1\subseteq M_2$.
\end{definition}

\begin{definition}
The {\bf union} of two multisets $M_1\cup M_2 : X\rightarrow \mathbb N$ is defined as $(M_1\cup M_2)(x)=M_1(x)+M_2(x)$.
\end{definition}

\begin{definition}
The {\bf difference} of two multisets $M_1-M_2 : X\rightarrow \mathbb N$ is defined as $(M_1-M_2)(x)=M_1(x)-M_2(x)$.
\end{definition}

\begin{definition}
Product of multiset $M$ with natural number $n\in \mathbb N$ is $(n\cdot M)(x)=n\cdot M(x)$.  
\end{definition}

% section multisets (end)

\section{Multiset languages} % (fold)
\label{sec:multiset_languages}

The number of occurences of a given symbol $a\in V$ in the string $w\in V^*$ is denoted by $|w|_a$.

\begin{definition}
$\Psi_V(w)=(|w|_{a_1},|w|_{a_2},\dots,|w|_{a_n})$ is called a Parikh vector associated with the string $w\in V^*$, where $V=\{a_1,a_2,\dots a_n\}$.
\end{definition}

\begin{definition}
For a language $L\subseteq V^*$, $\Psi_V(L)=\{\Psi_V(w)|w\in L\}$ is the Parikh mapping associated with $V$.
\end{definition}

\begin{example}
Consider an alphabet $V=\{a,b\}$ and a language $L=\{a, ab, ba\}$.
$\Psi_V(L)=\{(1,0), (1,1)\}$. Notice that Parikh image of $L$ has only 2 element while $L$ has 3 elements.
\end{example}

\begin{definition}
If $FL$ is a family of languages, by $PsFL$ is denoted the family of Parikh images of languages in $FL$.
\end{definition}

% section multiset_languages (end)