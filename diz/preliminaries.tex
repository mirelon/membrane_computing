This chapter is about some basic notions of computer science which will be used through the work. We start by defining formal languages and basic models (grammars, machines) that define language families and end by defining multiset languages.

\section{Formal languages} % (fold)
\label{sec:formal_languages}

Our study is based on the classical theory of formal languages. We will recall some definitions:

\begin{definition}
An {\bf alphabet} is a finite nonempty set of symbols.
\end{definition}

\begin{definition}
A {\bf string} over an alphabet is a finite sequence of symbols from alphabet.
\end{definition}

The length of the string $s$ is denoted by $|s|$. We denote by $V^*$ the set of all strings over an alphabet $V$. By $V^+$ = $V^* - \{\eps\}$ we denote the set of all nonempty strings over V.

\begin{definition}
A {\bf language} over the alphabet $V$ is any subset of $V^*$.
\end{definition}

\begin{definition}
A {\bf family of languages} is a set of languages.
\end{definition}


\section{Formal grammars} % (fold)
\label{sec:formal_grammars}

\begin{definition}
A {\bf formal grammar} is a tuple $G = (N,T,P,\sigma)$, where
\begin{itemize}
  \item $N, T$ are disjoint alphabets of non-terminal and terminal symbols,
  \item $\sigma\in N$ is the initial non-terminal,
  \item $P$ is a finite set of rewriting rules of the form $u\rightarrow v$, with $u\in (N\cup T)^*N(N\cup T)^*$ and $v\in (N\cup T)^*$.
\end{itemize}
\end{definition}

\begin{definition}
A {\bf rewriting step} in the grammar $G$ is a binary relation $\Rightarrow$ on $(N\cup T)^*$, where $x\Rightarrow y$ only if $\exists w_1, w_2\in (N\cup T)^+$ and a rule $u\rightarrow v \in P$ such that $x=w_1uw_2$ and $y=w_1vw_2$.
\end{definition}

\begin{definition}
Language defined by a grammar $G$ is a set $L(G)=\{w\in T^*|\sigma\Rightarrow w\}$.
\end{definition}

Languages that can be generated by a formal grammar are the recursively enumerable languages $RE$.

% section formal_languages (end)

% section formal_grammars (end)

\section{Chomsky hierarchy} % (fold)
\label{sec:chomsky_hierarchy}

In this section we introduce several well-known families of languages.

\begin{definition}
A {\bf regular grammar} is a formal grammar, where the rewriting rules are of the form $u\rightarrow v$, where $u\in N$ and $v\in T^*(N\cup \{\eps\})$.
\end{definition}

\begin{definition}
A {\bf regular language} is a language generated by a regular grammar. The family of regular languages is denoted $R$.
\end{definition}

\begin{definition}
A {\bf context-free grammar} is a formal grammar, where rewriting rules are of the form $u\rightarrow v$, where $u\in N$ and $v\in (N\cup T)^*$.
\end{definition}

\begin{definition}
A {\bf context-free language} is a language generated by a context-free grammar. The family of context-free languages is denoted $CF$.
\end{definition}

\begin{definition}
A {\bf context-sensitive grammar} is a formal grammar, where rewriting rules are of the form $u\rightarrow v$, where $u\in (N\cup T)^*N(N\cup T)^*$, $v\in (N\cup T)^*$ and $|u| < |v|$.
\end{definition}

\begin{definition}
A {\bf context-sensitive language} is a language generated by a context-sensitive grammar. The family of context-sensitive languages is denoted $CS$.
\end{definition}

These families of languages forms the Chomsky hierarchy by means of inclusions: $R \subset CF \subset CS \subset RE$.

% section chomsky_hierarchy (end)

\section{Matrix grammars} % (fold)
\label{sec:matrix_grammars}

\begin{definition}
A {\bf matrix grammar} is a tuple $G = (N,T,M,\sigma)$, where:
\begin{itemize}
  \item $N, T$ are disjoint alphabets of non-terminal and terminal symbols,
  \item $\sigma\in N$ is the initial non-terminal,
  \item $M$ is a finite set of matrices, which are sequences of context-free rules of the form $u\rightarrow v$, where $u\in N$ and $v\in (N\cup T)^*$.
\end{itemize}
\end{definition}

\begin{definition}
A {\bf rewriting step} $x\Rightarrow y$ holds only if there is a matrix $(u_1\rightarrow v_1, u_2\rightarrow v_2, \ldots, u_n\rightarrow v_n) \in M$ such that for each $1\leq i\leq n$ the following holds: $x_i = x_i^{\prime}u_ix_i^{\prime\prime}$ and $x_{i+1} = x_i^{\prime}v_ix_i^{\prime\prime}$, where $x_i, x_i^{\prime}, x_i^{\prime\prime} \in (N\cup T)^*$ and $x_1 = x$ and $x_{n+1} = y$.
\end{definition}

\begin{example}
Consider the matrix grammar $G=(\{\sigma, X,Y\}, \{ a,b,c\}, M, \sigma)$, where $M$ contains three matrices: $[S\rightarrow XY], [X\rightarrow aXb, Y\rightarrow cY], [X\rightarrow ab, Y\rightarrow c]$. There are only context-free rules, yet the grammar generate the context-sensitive language $\{a^nb^nc^n|n\geq 1\}$.
\end{example}

The family of matrix grammars is denoted $MAT$.

It is known that $CF \subset MAT \subset RE$. Interestingly, $MAT \cap {a}^* \subset R$ (see \cite{Besozzi:PhD:2004}).

% section matrix_grammars (end)

\section{Register machines} % (fold)
\label{sec:register_machines}

% We will use the notion of register machine as defined in our article

\begin{definition}
  A {\bf $n$-register machine} is a tuple $M = (n,P,i,h)$, where:
  \begin{itemize}
    \item $n$ is the number of registers,
    \item $P$ is a set of labeled instructions of the form $j : (op(r),k,l)$, where $op(r)$ is an operation on register $r$ of $M$, and $j$, $k$, $l$ are labels from the set $Lab(M)$ (which numbers the instructions in a one-to-one manner),
    \item $i$ is the initial label, and
    \item $h$ is the final label.
  \end{itemize}
\end{definition}

The machine is capable of the following instructions:
\begin{itemize}
  \item $(add(r),k,l)$ : Add one to the contents of register $r$ and proceed to instruction $k$ or to instruction $l$; in the deterministic variants usually considered in the literature we demand $k = l$.
  \item $(sub(r),k,l)$ : If register $r$ is not empty, then subtract one from its contents and go to instruction $k$, otherwise proceed to instruction $l$.
  \item $halt$ : This instruction stops the machine. This additional instruction can only be assigned to the final label $h$.
\end{itemize}

A deterministic $m$-register machine can analyze an input $(n_1,\dots,n_m)\in N_0^m$ in registers 1 to $m$, which is recognized if the register machine finally stops by the halt instruction with all its registers being empty (this last requirement is not necessary). If the machine does not halt, the analysis was not successful.

% section register_machines (end)

\section{Lindenmayer systems} % (fold)
\label{sec:lindenmayer_systems}

In 1968, a Hungarian botanist and theoretical biologist Aristid Lindenmayer introduced \cite{Lindenmayer68} a new string rewriting algorithm named Lindenmayer systems (or L-systems for short). They are used by biologists and theoretical computer scientists to mathematically model growth processes of living organisms, especially plants. The difference with Chomsky grammars is that rewriting is parallel, not sequential.

The simplest version of L-systems assumes that the development of a cell is free of influence of other cells.
This type of L-systems is called $0L$ systems, where ``0'' stands for zero-sided communication between cells.

\begin{definition}
A $0L$ system is a triple $(\Sigma, P, \omega)$, where $\Sigma$ is an alphabet, $\omega$ is a word over $\Sigma$ and $P$ is a finite set of rewriting rules of the form $a\rightarrow x$, where $a\in\Sigma, x\in\Sigma^*$.
\end{definition}

It is assumed there is at least one rewriting rule for each letter of $\Sigma$. $0L$ system works in parallel way, so all the symbols are rewritten in each step.

\begin{example}
Consider a $0L$ system with alphabet $\Sigma = \{a,b\}$, initial word $\omega = a$ and rewriting rules $P = \{a\rightarrow b, b\rightarrow ab\}$.
Since in this system there is exactly one rule for every letter of the alphabet, the rewriting is thus deterministic and the generated words will be $\{a, b, ab, bab, abbab, \ldots \}$. 
\end{example}

$1L$ systems allows the rewriting rules to include context of size 1, so it allows for rules of type $yaz\rightarrow x$.

L-systems with tables ($T$) have several sets of rewriting rules instead of just one set. At one step of the rewriting process, rules belonging to the same set have to be applied. The biological motivation for introducing tables is that one may want different rules to take care of different environmental conditions (heat, light, etc.) or of different stages of development.

\begin{definition}
An extended ($E0L$) system is a pair $G_1 = (G, \Sigma_T)$, where $G = (\Sigma, P, \omega)$ is an $0L$ system, where $\Sigma_T \subseteq \Sigma$, referred to as the terminal alphabet. The language generated by $G_1$ is defined by $L(G_1) = L(G)\cap \Sigma_T^*$.
\end{definition}

Such languages are called $E0L$ languages. $E0L$ languages with tables are called $ET0L$ languages.

It is known that $CF \subset E0L \subset ET0L \subset CS$ (see section \ref{sec:chomsky_hierarchy} for definitions of $CF$ and $CS$).
% section lindenmayer_systems (end)

\section{Semilinear sets} % (fold)
\label{sec:semilinear_sets}

% section semilinear_sets (end)

\section{Petri nets} % (fold)
\label{sec:petri_nets}

Petri nets \cite{Petri62,Yen06PetriNets} were introduced by Carl Adam Petri in 1962 in his PhD thesis. A Petri net is a graphical and mathematical tool for the modeling of concurrent processes and analysis of system behavior. A Petri net is usually drawn as a directed bipartite graph with two kind of nodes. Places are represented by circles within which each small black dot denotes a token. Transitions are represented by bars. Each edge is either from a place to a transition or vice versa.

\begin{definition}
  A {\bf Petri net} is a tuple $(P, T, \varphi)$, where:
  \begin{itemize}
    \item $P$ is a finite set of places,
    \item $T$ is a finite set of transitions,
    \item $\varphi: (P\times T)\cup(T\times P)\rightarrow \mathbb N$ is a flow function.
  \end{itemize}
\end{definition}

\begin{definition}
  A {\bf marking} is a mapping $\mu: P\rightarrow \mathbb N$.
\end{definition}

The mapping $\mu$ assigns tokens to each place of the net. The edges are annotated by either $\varphi(p,t)$ or $\varphi(t,p)$, where $p\in P$ and $t\in T$ are two endpoints of the arc. If $\varphi(p,t)=1$ or $\varphi(t,p)=1$, we usually omit the label.

\begin{definition}
  A transition $t\in T$ is {\bf enabled} at a marking $\mu$ iff $\forall p\in P, \varphi(p,t)\leq\mu(p)$.
\end{definition}

If a transition $t$ is enabled, it may fire by removing $\varphi(p,t)$ tokens from each input place $p$ and putting $\varphi(t,p^\prime)$ tokens in each output place $p^\prime$. We then write $\mu\xrightarrow{t} \mu^\prime$, where $\forall p\in P: \mu^\prime(p) = \mu(p)-\varphi(p,t)+\varphi(t,p)$.

\begin{figure}
  \centering
  \begin{tikzpicture}
    \tikzstyle{transition}=[rectangle,thick,fill=black,minimum height=8mm]
    \node [place,tokens=3,label=above:$p_1$] (p1) {};
    \node [transition,label=above:$t_1$] (t1) [right of=p1] {}
      edge [pre] (p1);
    \node [place,tokens=0,label=right:$p_3$] (p3) [right of=t1] {}
      edge [pre] (t1);
    \node [place,tokens=0,label=right:$p_2$] (p2) [above of=p3] {}
      edge [pre] (t1);
    \node [place,tokens=0,label=right:$p_4$] (p4) [below of=p3] {}
      edge [pre] (t1);
    \node [transition,label=below:$t_2$] (t2) [below of=t1] {}
      edge [pre] (p4)
      edge [post] (p1);
  \end{tikzpicture}
  \caption{Example Petri net}
  \label{fig:example petri net}
\end{figure}

\begin{example}
  In the Figure \ref{fig:example petri net} the Petri net has four places and two transitions. At the current marking the transition $t_1$ is enabled and the transition $t_2$ is not enabled. Firing the transition $t_1$ takes one token from the place $p_1$ and produces one token to places $p_2, p_3$ and $p_4$. In the resulting marking both transitions $t_1$ and $t_2$ are enabled.
\end{example}

\begin{definition}
  A sequence of transitions $\sigma = t_1\ldots t_n$ is a {\bf firing sequence} from $\mu_0$ iff $\mu_0\xrightarrow{t_1}\mu_1\xrightarrow{t_2}\ldots\xrightarrow{t_n}\mu_n$ for some markings $\mu_1,\ldots,\mu_n$. We also write $\mu_0\xrightarrow{\sigma}\mu_n$.
\end{definition}

We write $\mu_0\xrightarrow{\sigma}$ to denote that $\sigma$ is enabled and can be fired from $\mu_0$, i.e., $\mu_0\xrightarrow{\sigma}$ iff there exists a marking $\mu$ such that $\mu_0\xrightarrow{\sigma}\mu$.

The notation $\mu_0\xrightarrow{*}\mu$ is used to denote the existence of a firing sequence $\sigma$ such that $\mu_0\xrightarrow{\sigma}\mu$.

\begin{definition}
  A {\bf marked Petri net} is a tuple $(P,T,\varphi,\mu_0)$, where $(P,T,\varphi)$ is a Petri net and $\mu_0$ is called the initial marking.
\end{definition}

\begin{definition}
  Let $\mathcal P = (P,T,\varphi,\mu_0)$ be a marked Petri net. The {\bf reachability set} of $\mathcal P$ je $R(\mathcal(P)) = \{\mu|\mu_0\xrightarrow{*}\mu\}$.
\end{definition}

% section petri_nets (end)

\section{Vector addition systems} % (fold)
\label{sec:vector_addition_systems}

Vector addition systems were introduced by Karp and Miller \cite{Karp69ParallelProgramSchemata}, and were later shown by Hack \cite{Hack74PetriVAS} to be equivalent to Petri nets.

\begin{definition}
  A {\bf vector addition system} (VAS) is a pair $G = (x, W)$, where $x\in \mathbb N^n$ is an initial vector and $W\subseteq \mathbb Z^n$ is a finite set of vectors, where $n>0$ is called the dimension of VAS.
\end{definition}

The initial vector is seen as the initial values of multiple counters and the vectors in $W$ are seen as actions that update the counters. These counters may never drop below zero. 

\begin{definition}
  The {\bf reachability set} of the VAS $G = (x,W)$ is the set \linebreak $R(G) = \{z | \exists v_1,\ldots,v_j\in W: z=x+v_1+\ldots+v_j \wedge \forall 1\leq i\leq j: x+v_1+\ldots+v_i\geq 0\}$.
\end{definition}

\begin{definition}
  A {\bf vector addition system with states} (VASS) is a tuple $G = (x, W, Q, T, p_0)$, where:
  \begin{itemize}
    \item $(x, W)$ is a vector addition system,
    \item $Q$ is a finite set of states,
    \item $T$ is a finite set of transitions of the form $p\rightarrow(q,v)$, where $v\in W$ and $p,q\in Q$ are states,
    \item $p_0\in Q$ is the starting state.
  \end{itemize}
\end{definition}

The transition $p\rightarrow(q,v)$ can be applied at vector $y$ in state $p$ and yields the vector $y+v$ in state $q$, provided that $y+v\geq 0$.

\begin{example}
  For the Petri net in the Figure \ref{fig:example petri net}, the corresponding VAS $(x,W)$ is:
  \begin{itemize}
    \item $x=(3,0,0,0)$,
    \item $W=\{(-1,1,1,1),(1,0,0,-1)\}$.
  \end{itemize}
\end{example}

It is known \cite{Hack74PetriVAS} that Petri nets, VAS and VASS are computationally equivalent.

% section vector_addition_systems (end)

\section{Büchi automaton} % (fold)
\label{sec:buchi_automaton}

% section buchi_automaton (end)

\section{Calculi of looping sequences} % (fold)
\label{sec:calculi_of_looping_sequences}

% section calculi_of_looping_sequences (end)

\section{Graph theory} % (fold)
\label{sec:graph_theory}

% section graph_theory (end)

\section{Multisets} % (fold)
\label{sec:multisets}

\begin{definition}
A multiset over a set $X$ is a mapping $M: X\rightarrow \mathbb N$.
\end{definition}

We denote by $M(x), x\in X$ the multiplicity of $x$ in the multiset $M$.

\begin{definition}
The {\bf support} of a multiset $M$ is the set $supp(M)=\{x\in X|M(x)\geq 1\}$.
\end{definition}

It is the set of items with at least one occurrence.

\begin{definition}
A multiset is {\bf empty} when its support is empty.
\end{definition}

A multiset $M$ with finite support $X = \{x_1, x_2, \ldots, x_n\}$ can be represented by the string $x_1^{M(x_1)}x_2^{M(x_2)}\ldots x_n^{M(x_n)}$.
As elements of a multiset can also be strings, we separate them with the pipe symbol, e.g. $element|element|other\_element$.

\begin{definition}
Multiset inclusion. We say that multiset $M_1$ is included in multiset $M_2$ if $\forall x \in X: M_1(x)\leq M_2(x)$. We denote it by $M_1\subseteq M_2$.
\end{definition}

\begin{definition}
The {\bf union} of two multisets $M_1\cup M_2$ is a multiset where $\forall x \in X: (M_1\cup M_2)(x)=M_1(x)+M_2(x)$.
\end{definition}

\begin{definition}
The {\bf difference} of two multisets $M_1-M_2$ is a multiset where $\forall x \in X: (M_1-M_2)(x)=M_1(x)-M_2(x)$.
\end{definition}

\begin{definition}
Product of multiset $M$ with natural number $n\in \mathbb N$ is a multiset where $\forall x \in X: (n\cdot M)(x)=n\cdot M(x)$.  
\end{definition}

% section multisets (end)

\section{Multiset languages} % (fold)
\label{sec:multiset_languages}

The number of occurrences of a given symbol $a\in V$ in the string $w\in V^*$ is denoted by $|w|_a$.

\begin{definition}
$\Psi_V(w)=(|w|_{a_1},|w|_{a_2},\ldots,|w|_{a_n})$ is called a Parikh vector associated with the string $w\in V^*$, where $V=\{a_1,a_2,\ldots a_n\}$.
\end{definition}

\begin{definition}
For a language $L\subseteq V^*$, $\Psi_V(L)=\{\Psi_V(w)|w\in L\}$ is the Parikh mapping associated with $V$.
\end{definition}

\begin{example}
Consider an alphabet $V=\{a,b\}$ and a language $L=\{a, ab, ba\}$.
$\Psi_V(L)=\{(1,0), (1,1)\}$. Notice that Parikh image of $L$ has only 2 element while $L$ has 3 elements.
\end{example}

\begin{definition}
If $FL$ is a family of languages, by $PsFL$ we denote the family of Parikh images of languages in $FL$.
\end{definition}

% section multiset_languages (end)

\section{Bisimulations} % (fold)
\label{sec:bisimulations}
\begin{definition}
  A {\bf state transition system} is a pair $(S, \rightarrow)$, where $S$ is a set of states and $\rightarrow\subseteq S\times S$ is a binary transition relation over $S$.
\end{definition}
  If $p,q\in S$, then $(p,q)\in \rightarrow$ is usually written as $p\rightarrow q$. This represents the fact that there is a transition from state $p$ to state $q$.

\begin{definition}
  A {\bf labelled state transition system} (LTS) is a tuple $(S, A, \rightarrow)$, where $S$ is a set of states, $A$ is a set of labels and $\rightarrow\subseteq S\times A\times S$ is a ternary transition relation.
\end{definition}
  If $p,q\in S$ and $a\in A$, then $(p,a,q)\in \rightarrow$ is usually written as $p\xrightarrow{a} q$. This represents the fact that there is a transition from state $p$ to state $q$ with a label $a$.

\begin{definition}
  Let $(S_1, A, \rightarrow)$ and $(S_2, A, \rightarrow)$ be two labelled transition systems.
  A {\bf simulation} is a binary relation $R\subseteq S_1\times S_2$ such that if $(s_1,s_2)\in R$ then for each $s_1\xrightarrow{a} t_1$ there is some $s_2\xrightarrow{a} t_2$ such that $(t_1, t_2)\in R$.
\end{definition}

\begin{definition}
  Let $(S_1, A, \rightarrow)$ and $(S_2, A, \rightarrow)$ be two labelled transition systems.
  A {\bf bisimulation} is a binary relation $R\subseteq S_1\times S_2$ such that if $(s_1,s_2)\in R$ then:
  \begin{enumerate}
    \item for each $s_1\xrightarrow{a} t_1$ there is some $s_2\xrightarrow{a} t_2$ such that $(t_1, t_2)\in R$,
    \item for each $s_2\xrightarrow{a} t_2$ there is some $s_1\xrightarrow{a} t_1$ such that $(t_1, t_2)\in R$.
  \end{enumerate}
\end{definition}
% section bisimulations (end)