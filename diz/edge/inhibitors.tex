% !TEX root = ../diz.tex
\index{P systems!sequential} A sequential variant without priorities and with cooperative rules is not universal (see \cite{Ibarra04dang}). They have tried modifying the variant to increase the computational power and showed that with rules for membrane creation with unbounded number of membranes it became universal.

We have tried another approach using rules with inhibitors. We show that this variant is computationally complete in both generating and accepting case. For the generative case we present a proof by a simulation of maximal parallel P system and in the accepting case we prove it can simulate a register machine. These results were published in \cite{Kovac14Inhibitors}.

In the literature there are two variants of \index{P systems!with inhibitors} rules with inhibitors.
\begin{enumerate}
  \item Some of them (\cite{Ionescu:jucs_10_5:on_p_systems_with}, \cite{Sburlan05dragos}) allow to use only one inhibitor per rule, e.g. $u\rightarrow v|_{\neg i}$.
  \item Others (\cite{Agrigoroaiei:2010:Dissolution}, \cite{Sburlan:2006:FurtherResultsPromotersInhibitors}) are more expressive allowing to use a set of inhibitors per rule, e.g. $u\rightarrow v|_{\neg B}$, where $B$ is a set of objects. Such a rule can be applied only if no element of $B$ is present in the region, where the rule is applied.
\end{enumerate}

Our proof uses variant of rules with a set of inhibitors, but they could also be implemented with rules with a single inhibitor, while the complexity of the simulation would be increased.
Dissolution could be also solved with increased complexity by introducing additional phases of the simulation. But because P systems without dissolution are still computationally complete \cite{Agrigoroaiei:2010:Dissolution}, and for the sake of simplicity, we simulate only rules without dissolution 

\begin{definition}
\label{def:evolution_rule_without_dissolution}
  An {\bf evolution rule without dissolution} over an alphabet $\Sigma$ in a P system with $m$ membranes is a pair $(u,v)$, which is usually written as $u\rightarrow v$, where $u$ is a string over $\Sigma$ and $v$ is a string over $\Sigma\times(\{here, out\}\cup\{in_j|1\leq j\leq m\})$.
\end{definition}

\begin{definition}
  An {\bf evolution rule with inhibitors without dissolution} is a triple $(u,v,B)$, which is usually written as $u\rightarrow v|_{\neg B}$, where $(u,v)$ is an evolution rule without dissolution by definition \ref{def:evolution_rule_without_dissolution} and $B\subseteq\Sigma$ is a set of objects called inhibitors of the rule.
\end{definition}

\definecolor{run}{rgb}{1,0.5,0}
\definecolor{restore}{rgb}{0,0.5,0}
\definecolor{synchronize}{rgb}{0,0,1}
\definecolor{senddown}{rgb}{1,0,0}

\subsection{Simulation of maximal parallel P system} % (fold)
\label{sub:simulation_of_maximal_parallel_p_system}

The last lemmas (\ref{lemma:context_rules} and \ref{lemma:inhibitor_step}) will simplify the proof of the following theorem.

\begin{veta}
% TODO: rozdelit na viac liem
  The sequential P system with inhibitors defines the same Parikh image of language as P system with maximal parallelism.
\end{veta}

\begin{dokaz}
  We will prove the theorem by simulation.
  The proof is quite technical with some workarounds.
  For every maximal parallel P system $\Pi$ we will find a sequential P system with inhibitors $\overline{\Pi}$ such that for every computation of $\Pi$ there exists a computation of $\overline{\Pi}$ with the same result.
  The computation of $\Pi$ is a sequence of maximal parallel steps.
  In each step of $\Pi$, a maximal multiset of rewriting rules are applied in each membrane.
  We simulate this maximal parallel step with several steps of sequential system $\overline{\Pi}$.

  We must prevent rule application from the next maximal parallel step as that would be an incorrect simulation.
  Preventing this will be done with synchronization.
  Every membrane will have a state, represented as an object.
  The most important states are $RUN$ and $SYNCHRONIZE$.
  The simulation starts with every membrane containing an object representing $RUN$ state.
  In the $RUN$ state, rewriting rules of $\Pi$ are applied until there is no rule to be applied. Then, the membrane proceeds to the $SYNCHRONIZE$ state.
  All the membranes in the $SYNCHRONIZE$ state are waiting for other membranes to reach $SYNCHRONIZE$ state, so they all can proceed to the $RUN$ state of the next maximal parallel step of $\Pi$.

  Other states are just technical - we need to implement sending objects between membranes and preparing for the next maximal parallel step by clearing temporary objects.

  More detailed description of states:
  

  \begin{itemize}
    \item $RUN$: Rewriting occurs.
    To prevent double rewriting, we mark the products of rules with $a^{\prime}$ instead of $a$.
    Objects that are to be sent to the parent membrane are directly sent because the parent membrane is already in $RUN$ or $SYNCHRONIZE$ state (see Figure~\ref{fig:possible_pairs_of_states_of_parent_and_child_membrane}), so the $a^{\prime}$ symbols that are sent don't break anything.
    
    But objects that are to be sent down, can't be sent immediately because child membranes can be in the $RESTORE$ state restoring symbols from the previous maximal parallel step. Current symbols could interfere with them and they could be rewritten twice in this step. Such objects are only marked as ``to be sent down'': $a^{\downarrow\prime}$.

    When $RUN$ phase ends (in the membrane $i$), the $SYNCTOKEN_i$ is sent upwards to notify the skin membrane that the membrane $i$ is ready to be synchronized.

    \item $SYNCHRONIZE$: Rewriting has ended and membrane is now waiting to receive signal $SYNCED$ from the skin membrane to continue to the next state.

    \item $SENDDOWN$: Signal $SYNCED$ has been caught and now all descendant membranes are in the $SYNCHRONIZE$ state. So objects $a^{\downarrow\prime}$ can be sent down.

    \item $RESTORE$: All $a^{\prime}$ symbols are restored to $a$ and other temporary symbols are cleared, so the next maximal parallel step can take place.
  \end{itemize}

  \begin{figure}
    \def\svgwidth{\textwidth}
    \input{possible_pairs_of_states_of_parent_and_child_membrane.pdf_tex}
    % \includegraphics[width=\textwidth]{possible_pairs_of_states_of_parent_and_child_membrane}
    \caption{Possible pairs of states of parent and child membrane}
    \label{fig:possible_pairs_of_states_of_parent_and_child_membrane}
  \end{figure}

  \begin{figure}
    \def\svgwidth{\textwidth}
    \input{snapshot_of_all_membrane_states_while_simulating.pdf_tex}
    % \includegraphics[width=\textwidth]{snapshot_of_all_membrane_states_while_simulating}
    \caption{Snapshot of all membrane states while simulating}
    \label{fig:snapshot_of_all_membrane_states_while_simulating}
  \end{figure}

  In the figure \ref{fig:possible_pairs_of_states_of_parent_and_child_membrane} the pairs of states of parent and child membrane are joined with possible transitions between two consecutive global synchronizations - after the maximal parallel steps $i$ and $i+1$.

  In the figure \ref{fig:snapshot_of_all_membrane_states_while_simulating} the membrane structure is presented as hierarchical structure. Every region is in one of four states. It is obvious that the sending of the objects is performed in such states that the receiving region is in $RUN$ or $SYNCHRONIZE$ state, so the received objects (marked $a^{\prime}$) cannot interfere with rewriting.

  It is important to note that when a region is in $SENDDOWN$ state and objects are sent through the child membrane, the receiving region is in the $SYNCHRONIZE$ state waiting for the $SYNCED$ signal, which will be sent to it when $SENDDOWN$ and $RESTORE$ phases finished.
  

  % Rewriting rules

  The rewriting rules in $\overline{\Pi}$ fire in a sequential manner.
  They must have at most two objects at the left side. If a rule has more than two objects at the left side, we recall the lemma \ref{lemma:context_rules} and discuss the preconditions.
  They may contain inhibitors. Every time it contains more than one inhibitor, we recall the lemma \ref{lemma:inhibitor_step} and discuss the preconditions. 

  Here follows the list of rules in $\overline{\Pi}$.

  \begin{itemize}
    \item For every rule $r_i\in R$ such that $r_i = a_1^{M(a_1)}a_2^{M(a_2)}\dots a_n^{M(a_n)} \rightarrow a_1^{N(a_1)}a_2^{N(a_2)}\dots a_n^{N(a_n)}$ we will have the following rules:
  
    $a_1^{M(a_1)-m_1}\dot{a}_1^{m_1}a_2^{M(a_2)-m_2}\dot{a}_2^{m_2}\dots a_n^{M(a_n)-m_n}\dot{a}_n^{m_n}|RUN \rightarrow a_1^{\prime N(a_1)}a_2^{\prime N(a_2)}\dots a_n^{\prime N(a_n)}|RUN$

    There will be such rule for each $0\leq m_j\leq M(a_j)$.
    Right side of the rules contains symbols $a^\prime$, that prevents the symbols to be rewritten again.
    Usage of symbols $\dot{a}$ in these rules represents the idea that they can be used in rewriting in the same way as $a$. Their purpose will be explained later in the proof.

    \item For every symbol $a\in V$ we will have the following rules:

    $a|RUN \rightarrow \dot{a}|RUN|_{\neg \dot{a}}$

    There will be max one occurrence of $\dot{a}$.

    \item For every rule $r_i\in R$ there will be a rule that detects if the rule $r_i$ is not usable. According to left side of the rule $r_i$, symbol $UNUSABLE_i$ will be created when there is not enough objects to fire the rule $r_i$. It means that left side of rule $r_i$ requires more instances of some object than are present in membrane.

    If the left side is of type:
    \begin{itemize}
      \item $a$: It is a context free rule. The rule can't be used if there is no occurrence of $a$ nor $\dot{a}$.

      $RUN \rightarrow UNUSABLE_i|RUN|_{\neg\{UNUSABLE_i, a, \dot{a}\}}$

      \item $ab$: It is a cooperative rule with two distinct objects on the left side. The rule can't be used if there is one of them missing.

      $RUN \rightarrow UNUSABLE_i|RUN|_{\neg\{UNUSABLE_i, a, \dot{a}\}}$

      $RUN \rightarrow UNUSABLE_i|RUN|_{\neg\{UNUSABLE_i, b, \dot{b}\}}$

      \item $a^2$: It is a cooperative rule with two same objects. The rule can't be used if there is at most one occurrence of the symbol. That happens if there is no occurrence of $a$. There can still be $\dot{a}$, but at most one occurrence.

      $RUN \rightarrow UNUSABLE_i|RUN|_{\neg\{UNUSABLE_i, a\}}$
    \end{itemize}

    

    \item For every membrane with label $i$ there will be rule:

    $UNUSABLE_1|\dots|UNUSABLE_m|RUN \rightarrow SYNCHRONIZE|SYNCTOKEN_i\uparrow$

    If no rule can be used, maximal parallel step in the region is completed so it goes to synchronization phase and sends a synchronization token to the parent membrane. The objects $UNUSABLE_i$ are not consumed in other rules, so by Lemma \ref{lemma:context_rules} this rule can be written with set of cooperative rules.

    \item For every membrane other than the skin membrane, there will be a rule:

    $SYNCHRONIZE|SYNCTOKEN_j \rightarrow SYNCHRONIZE|SYNCTOKEN_j\uparrow$

    Membrane forwards all sync token to parent membrane.

    \item In the skin membrane there is a rule which collects all the synchronization tokens from all membranes $1\dots k$ and then sends down signal that synchronization is complete. But before that, there can be some symbols that should be sent down, but they weren't, because the region below could have not started the rewriting phase that time. The result was just marked with $a^{\downarrow\prime}$.

    $SYNCTOKEN_1|\dots|SYNCTOKEN_k|SYNCHRONIZE \rightarrow SENDDOWN$

    The objects $SYNCTOKEN_i$ are not consumed in other rules in the skin membrane, so by Lemma \ref{lemma:context_rules} this rule can be written with set of cooperative rules.

    \item Every membrane other than skin membrane have to receive the signal to go to senddown phase:

    $SYNCHRONIZE|SYNCED \rightarrow SENDDOWN$

    \item Every membrane will have rules for every symbol $a\in V$ to send down all unsent object that should have been sent down:

    $SENDDOWN|a^{\downarrow\prime} \rightarrow SENDDOWN|a^{\prime}\downarrow$

    \item Every membrane will have rule for detecting when all such objects have been sent and it goes to restore phase:

    $SENDDOWN \rightarrow RESTORE|_{\neg \{a_i^{\downarrow\prime}|1\leq i\leq n\}}$

    \item In restore phase all symbols $a^{\prime}$ will be rewritten to $a$ in order to be able to be rewritten in next maximal parallel step:

    $RESTORE|a^{\prime} \rightarrow RESTORE|a$

    \item There may be some $GONE$ symbols left, which should be cleared too:

    $RESTORE|GONE_i \rightarrow RESTORE$

    \item When restore phase ends, it sends down signal that all membranes have been synchronized and next phase of rewriting has began in upper membranes:

    $RESTORE \rightarrow RUN|SYNCED\downarrow|_{\neg \{a_i^{\prime}|1\leq i\leq n\}\cup\{GONE_i|1\leq i\leq n\}}$
  \end{itemize}


  The regions are never empty, because they contain at least the object representing the state. Thus the rules with set of inhibitors can be simulated by single inhibitors due to the Lemma~\ref{lemma:inhibitor_step}.
\end{dokaz}

% subsection simulation_of_maximal_parallel_p_system (end)

We have also reached this result in the accepting case by simulation of \index{Register machine} register machines.

\begin{veta}
  Sequential P system with inhibitors can simulate register machine and thus equals $PsRE$.
\end{veta}


\begin{dokaz}
\label{proof:reg_by_inh}
  Suppose we have a $n$-register machine $M = (n,P,i,h)$. In our simulation we will have a membrane structure consisting of one membrane and the contents of register $j$ will be represented by the multiplicity of the object $a_j$.

  We will have P system $(V, \mu, w, R)$, where:
  \begin{itemize}
    \item $V$ is an alphabet consisting of symbols that represent registers $a_1,\dots a_n$ and instruction labels in $Lab(M)$,
    \item $\mu$ is a membrane structure consisting of only one membrane,
    \item $w$ is initial contents of the membrane. It contains symbols for the input for the machine $a_i^{n_i}$ where $n_i$ is initial state of register with label $i$ and initial instruction $e \in Lab(M)$.
    \item $R$ is the set of rules in the skin membrane:
    
    For all instructions of type $(e : add(j), f)$ we will have rule:
    \begin{itemize}
      \item $e \rightarrow a_j|f$.
    \end{itemize}
    
    For all instructions of type $(e : sub(j), f, z)$ we will have rules:
    \begin{itemize}
      \item $e|a_j \rightarrow f$ and
      \item $e \rightarrow z|_{\neg a_j}$.
    \end{itemize}

    And finally halting rules:
    \begin{itemize}
      \item $h|a_j \rightarrow h|\#$ for all $a\leq j\leq n$,
      \item $\# \rightarrow \#$,
    \end{itemize}
  \end{itemize}

  When the halting instruction is reached, if there is an object present in the membrane, the hash symbol $\#$ is created and it will cycle forever. If there is no object present, there is no rule to apply and computation will halt. It corresponds to the condition that all registers should be empty when halting.
\end{dokaz}

\subsection{Concluding remarks} % (fold)
\label{sub:concluding_remarks_of_inhibitors}

Although these results are not very surprising as similar results already hold for Petri nets, which are equivalent to sequential P systems, the simulation of maximal parallel P system used in the proof for generating case is valuable not only for the universality, but also can be seen as a method of conversion between P systems in sequential manner and maximally parallel manner, which may be essential for future works on P systems and other multiset rewriting systems. The simulation also shows a method how to synchronize application of multiple rules in a membrane and how to synchronize this parallel rule application across whole membrane structure. Sequential variants are promising alternative to traditional maximal parallel variants and will be good subject for the further research. Future plans include research of other more restricted variants such as omitting cooperation in the rules or restricting the power of inhibitors.

% subsection concluding_remarks (end)
