% !TEX root = ../diz.tex
{\em ``In general, it seems that any extension which does not allow zero testing will not actually increase the modeling power (or decrease the decision power) of Petri nets but merely result in another equivalent formulation of the basic Petri net model. (Modeling convenience may be increased)''} \cite{Peterson81PetriNets}, page 203.

The above quote from \cite{Peterson81PetriNets} was a fair summary of current beliefs in the Petri net community regarding extensions of the basic Petri net mode: extensions are either Turing-powerful of they are not real extensions.

It is not the case as shown in \cite{Dufourd98Reset}. They show extensions of Petri nets which do not allow zero testing but that will actually increase the computational power and decrease the decision power (e.g. boundedness becomes undecidable).

In this chapter we investigate several ``weak'' extensions of sequential P systems, which allow for zero-testing, aiming to fit in layers between mere reformulations of the basic sequential P system and universal sequential P systems with inhibitors.

We will extend the definition of evolution rules with additional decision option for objects that are being sent through a membrane to another region. Recall the original definition of the evolution rule \ref{def:evolution_rule}: $u\rightarrow v$, where $u$ is a string over $\Sigma$ and $v=v^\prime$ or $v=v^\prime\delta$, where $v^\prime$ is a string over $\Sigma\times(\{here, out\}\cup\{in_j|1\leq j\leq m\})$ and $\delta$ is a special symbol not in $\Sigma$. Recall also the algorithm \ref{alg:application_of_a_rule_in_a_p_system}, which will be extended in following subsections.

\subsection{Objects avoiding empty regions} % (fold)
\label{sub:objects_avoiding_empty_regions}

We will have a specific subset of objects, which when occur in a rule in form of $(a, out)$ or $(a, in_j)$, and the target region is empty, they are not sent and stay in the current region instead.

\begin{definition}
  P system with objects avoiding empty regions is a tuple $$\Pi = (\Sigma, \Gamma, \mu, w_1, w_2,\ldots , w_m, R_1, R_2,\ldots , R_m),$$ where $$(\Sigma, \mu, w_1, w_2,\ldots , w_m, R_1, R_2,\ldots , R_m)$$ is a P system and $\Gamma\subseteq\Sigma$ is a subset of objects avoiding empty regions.
\end{definition}

We present an algorithm for the rule application \vpageref[below]{alg:application_of_a_rule_in_a_p_system_with_objects_avoiding_empty_regions} (algorithm \ref{alg:application_of_a_rule_in_a_p_system_with_objects_avoiding_empty_regions}).

\begin{example}
  Suppose a P system with objects avoiding empty regions $(\Sigma = \{a,b\}, \Gamma = \{a\}, \mu = [_1[_2]_2]_1, w_1 = a, w_2 = \eps, \\R_1 = \{a\rightarrow a\downarrow_2b\downarrow_2\}, R_2 = \{\})$.

  The computation starts with an empty region 2 and the object $a$ in the region 1. Application of the rule sends the object $b$ into the region 2 and the object $a$ stays in the current region because it is avoiding the empty region 2. In the next step, the application of the rule sends both objects into the region 2, because it already contains the object $b$. The computation finishes with objects $abb$ in the region 2 and empty region 1.
\end{example}

\begin{algorithm}
  \caption{Application of a single rule in a P system with objects avoiding empty regions}\label{alg:application_of_a_rule_in_a_p_system_with_objects_avoiding_empty_regions}
  \begin{algorithmic}[1]
    \Procedure{RuleAplication}{applicable rule $u\rightarrow v\in R_i$, configuration $C = (\mu, w_1,w_2,\ldots w_m)$, set of objects avoiding empty regions $\Gamma\subseteq\Sigma$}
      \State $w_i := w_i - u$
      \ForAll{$(a, here)\in v$}
        \State $w_i := w_i + a$
      \EndFor
      \ForAll{$(a, out)\in v$}
        \If{$a\in\Gamma$ and $parent(i)$ is empty}
          \State $w_i := w_i + a$
        \Else
          \State $w_{parent(i)} := w_{parent(i)} + a$
        \EndIf
      \EndFor
      \ForAll{$(a, in_j)\in v$}
        \If{$a\in\Gamma$ and $j$ is empty}
          \State $w_i := w_i + a$
        \Else
          \State $w_j := w_j + a$
        \EndIf
      \EndFor
      \If{$v = v^\prime\delta$}
        \State $w_{parent(i)} := w_{parent(i)} + w_i$
        \State $w_i := \text{empty multiset}$
      \EndIf
    \EndProcedure
  \end{algorithmic}
\end{algorithm}

% subsection objects_avoiding_empty_regions (end)

\subsection{Objects altering when entering empty region} % (fold)
\label{sub:objects_altering_when_entering_empty_region}

We represent altering the objects entering empty regions with a mapping $\Phi: \Sigma\rightarrow\Sigma$. The mapping defines for each object what it will become when sent to an empty region.

\begin{definition}
  P system with objects altering when entering empty region is a tuple $$\Pi = (\Sigma, \Phi, \mu, w_1, w_2,\ldots , w_m, R_1, R_2,\ldots , R_m),$$ where $$(\Sigma, \mu, w_1, w_2,\ldots , w_m, R_1, R_2,\ldots , R_m)$$ is a P system and $\Phi: \Sigma\rightarrow\Sigma$ is a mapping of objects.
\end{definition}

We present an algorithm for this variant of the rule application \vpageref[below]{alg:application_of_a_rule_in_a_p_system_with_objects_altering_when_entering_empty_region} (algorithm \ref{alg:application_of_a_rule_in_a_p_system_with_objects_altering_when_entering_empty_region}).

\begin{example}
  Suppose a P system with objects altering when entering empty region $(\Sigma = \{a,b\}, \Phi = \{a\rightarrow b, b\rightarrow b\}, \mu = [_1[_2]_2]_1, w_1 = a, w_2 = \eps, \\R_1 = \{a\rightarrow ba\downarrow_2, b\rightarrow a\downarrow_2\}, R_2 = \{\})$.

  The computation starts with an empty region 2 and the object $a$ in the region 1. Application of the rule $a\rightarrow ba\downarrow_2$ produces an object $b$ which stays in the current region and the second product $a$ is sent into the region 2, which is empty, so the produced $a$ is altered to $b$. Now both regions contains an object $b$. In the next step, the application of the rule $b\rightarrow a\downarrow_2$ in region 1 sends the produced object $a$ into the region 2, but this time it is not altered because the region is nonempty. The computation finishes with objects $ab$ in the region 2 and empty region 1.
\end{example}

\begin{algorithm}
  \caption{Application of a single rule in a P system with objects altering when entering empty region}\label{alg:application_of_a_rule_in_a_p_system_with_objects_altering_when_entering_empty_region}
  \begin{algorithmic}[1]
    \Procedure{RuleAplication}{applicable rule $u\rightarrow v\in R_i$, configuration $C = (\mu, w_1,w_2,\ldots w_m)$}
      \State $w_i := w_i - u$
      \ForAll{$(a, here)\in v$}
        \State $w_i := w_i + a$
      \EndFor
      \ForAll{$(a, out)\in v$}
        \If{$parent(i)$ is empty}
          \State $w_{parent(i)} := w_{parent(i)} + \Phi(a)$
        \Else
          \State $w_{parent(i)} := w_{parent(i)} + a$
        \EndIf
      \EndFor
      \ForAll{$(a, in_j)\in v$}
        \If{$j$ is empty}
          \State $w_j := w_j + \Phi(a)$
        \Else
          \State $w_j := w_j + a$
        \EndIf
      \EndFor
      \If{$v = v^\prime\delta$}
        \State $w_{parent(i)} := w_{parent(i)} + w_i$
        \State $w_i := \text{empty multiset}$
      \EndIf
    \EndProcedure
  \end{algorithmic}
\end{algorithm}

% subsection objects_altering_when_entering_empty_region (end)

\subsection{Vacuum objects} % (fold)
\label{sub:vacuum_objects}
(from project proposal):

We propose a new variant of P system with vacuum in accordance with the criteria formulated in \cite{Besozzi:PhD:2004} (see beginning of the section \ref{sec:p_system_variants}). In the common sense, vacuum represents a state of space with no or a little matter in it. Using vacuum in modelling frameworks can help express certain phenomena more easily. We define a new P system variant, which creates a special vacuum object in a region as soon as the region becomes empty. The vacuum is removed whenever some object interacts with it. After the interaction, there is vacuum no longer. This removal process is realized by allowing the vacuum object to be used only on the left side of rules. If we made the vacuum to be removed automatically when an object enters the region, there would be no difference with the variant without vacuum objects because of no interactions with it.

We are interested in how the variant with the vacuum improves the computation power of a sequential P system in comparison to the variant without using the vacuum. This case have been shown to be universal.

In the future we will research other restriction for this variant such as non-cooperative rules, decaying objects, deterministic steps and will use techniques to show even non-universality results.

\begin{veta}
  The sequential P system with Vacuum is universal.
\end{veta}

\begin{dokaz}
  We can simulate the variant of P system where the only cooperative rule is of type $a|a \rightarrow b$. According to \cite{Ibarra04dang} the variant, where the only cooperative rule is when both objects are the same, is universal. If there is no rule $a \rightarrow b$, we can rewrite $a$ to $a^{\prime}$ so we can mark all present symbols. $a^{\prime}$ symbols are kept in special membrane so the Vacuum can be created in main membrane and we can synchronize.
  
  But we won't do this madness again.
  
  Instead, we will try to prove universality by simulating the register machine. We need to detect when the current register is empty. If there was a symbol for every register as in the proof~\ref{proof:reg_by_inh}, the Vacuum would be created only if all registers are empty. But the $sub()$ instruction need to detect when one concrete register is empty.
  
  We will have a membrane for each register. That membrane will be contained in the skin membrane. The number of objects in membrane $i$ will correspond to the value of register $i$.
  
  The alphabet will consist of instruction labels and register counter $a$. The skin membrane will only have an instruction label. It is sent to corresponding membrane where the instruction is executed. Then, the following instruction is sent back to the skin membrane.
  
  We will have following rules in the skin membrane:
  
  \begin{itemize}
  \item $e \rightarrow e^{\downarrow_j}$ for an instruction of type $e : add(j), f$ or $e : sub(j), f, z$ and
  \item $h \rightarrow h^{\downarrow}$ for a halting instruction h.
  \end{itemize}
  
  And in non-skin membranes:
  
  \begin{itemize}
  \item $e \rightarrow a|f^{\uparrow}$ instructions of type $e : add(j), f$,
  \item $e|a \rightarrow f^{\uparrow}$ for instructions of type $e : sub(j), f, z$,
  \item $e|VACUUM \rightarrow z^{\uparrow}$ for instructions of type $e : sub(j), f, z$, and
  \item $h|a \rightarrow h|a$
  \end{itemize}

  When halting, if there is an nonempty register, it will cycle forever with the last rule. However, if all registers are empty, the halting instruction label will stay in all membranes and the computation will halt.
  
\end{dokaz}
% subsection vacuum_objects (end)

\subsection{Concluding remarks} % (fold)
\label{sub:concluding_remarks_of_emptyness_detection}

The above presented variants show possible implementations of zero-check by the feature of emptyness detection which is a specific notion for membrane systems. The aim of the research was to find a variant with zero-check that will not lead to computational completeness and will possibly reveal some unexpected connection with other models of computation. Sequential P systems with objects altering when entering empty region (subsection \ref{sub:objects_altering_when_entering_empty_region}) and sequential P systems with vacuum objects (subsection \ref{sub:vacuum_objects}) are universal because we can simulate a register machine. On the other hand, the other variant with objects avoiding empty regions (subsection \ref{sub:objects_avoiding_empty_regions}) appears to be more promising for our goal because the standard contruction of register machine do not work. We conjecture this variant is not universal, possibly equivalent with Petri nets or other model of computation weaker than Turing machine.

% subsection concluding_remarks (end)