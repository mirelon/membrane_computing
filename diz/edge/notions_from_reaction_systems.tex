% !TEX root = ../diz.tex
The variants of P systems inspired by \index{Reaction systems} reaction systems introduced in subsection \ref{sub:variants_inspired_by_reaction_systems} have been inspected only for maximal parallel semantics. The various proposed semantics turned out to be deterministic and the research continued to define various semantics for computational step \cite{Kleijn11SetMembrane} and halting conditions \cite{Paun12BridgingReactionSystems}. The sequential mode was only mentioned in \cite{Kleijn11SetMembrane} under the notion of ``min-enabled'' computational step. As well as the maximal parallel mode, the sequential set membrane systems can only generate the regular languages \cite{Alhazov05WithoutMultiplicities}. The situation is changed with active membranes. It seems that it is not so black-and-white in terms of computational power of the sequential active P systems that are working with:
\begin{itemize}
  \item sets instead of multisets,
  \item the assumption of non-permanency of objects.
\end{itemize}

In following subsections we study variants of P systems with active membranes that are working in the sequential mode with sets instead of multisets. We challenge the original definition of a membrane creation because possible multiplicity of labels of child membranes are in conflict with no multiplicity of objects in reaction systems. We propose more suitable notions of membrane creation and prove computational completeness by simulating a register machine. In the proofs we use $(op(r), \_, \_)$ to denote any of the operations $add$ and $sub$ operating on the register $r$ having arbitrary following instruction. These results were published in \cite{Kovac15SequentialActiveSet}.

We leave the second proposal of assuming non-permanency of objects in sequential mode with active membranes as a topic for the future study.

\subsection{Alternative definition of active P systems} % (fold)
\label{sub:alternative_definition_of_active_p_systems}
  The rules in active P systems can modify the membrane structure by dissolving and creating new membranes. The skin membrane is not allowed to be dissolved, but object can be sent out through the skin membrane (e.g. to be observed by the surroundings). That is why we will define the configuration to include the membrane structure as well as in \cite{Kovac15SequentialActiveSet}.

  Let $\Sigma$ be a set of objects. A {\bf membrane configuration} is a tuple $(T, l, c)$, where:
  \begin{itemize}
    \item $T$ is a rooted tree,
    \item $l\in\mathbb N^{V(T)}$ is a mapping that assigns for each node of $T$ a number (label),
    \item $c\in(2^\Sigma)^{V(T)}$ is a mapping that assigns for each node of $T$ a set of objects from $\Sigma$, so it represents the contents of the membrane.
  \end{itemize}

  The common representation of a membrane structure in this section is by a string, where a membrane is denoted by a pair of matching square brackets with the subscript indicating the label of the membrane, e.g. $[_1 [_2 ab ]_2 ac ]_1$.

  A {\bf sequential active set P system} is a tuple $(\Sigma, C_0, R_1, R_2, \dots , R_m)$, where:
  \begin{itemize}
    \item $\Sigma$ is a set of objects,
    \item $C_0$ is the initial membrane configuration,
    \item $R_1,R_2,\dots, R_m$ are finite sets of rewriting rules associated with the labels $1,2,\dots,m$ and can be of forms:
    \begin{itemize}
      \item $u\rightarrow w$, where $u\subseteq \Sigma$, $|u|\geq 1$, $w\subseteq (\Sigma\times\{\cdot, \uparrow, \downarrow_j\})$ and $1\leq j\leq m$,
      \item a dissolving rule $u\rightarrow w\delta$, where $u\subseteq \Sigma$, $|u|\geq 1$, $w\subseteq (\Sigma\times\{\cdot, \uparrow, \downarrow_j\})$ and $1\leq j\leq m$,
      \item a membrane creation $u\rightarrow [_j v_1]_j v_2$, where $u\subseteq \Sigma$, $|u|\geq 1$, $v_1, v_2\subseteq \Sigma$ and $1\leq j\leq m$.
    \end{itemize}
  \end{itemize}

  For the first two forms, each rewriting rule may specify for each object on the right side, whether it stays in the current region (we will omit the symbol $\cdot$), moves through the membrane to the parent region ($\uparrow$) or to a specific child region ($\downarrow_j$, where $j$ is a label of a membrane).
  We denote these transfers with an arrow immediately after the symbol.
  An example of such rule is the following: $ab\rightarrow ab\downarrow_2 c\uparrow c\delta$.

  By applying the rule we mean the removal of objects specified on the left side and the addition of the objects on the right side with respect to set union semantics.
  Symbol $\delta\notin\Sigma$ does not represent an object. It may be present only at the end of the rule, which means that after the application of the rule, the membrane is dissolved and its contents (objects, child membranes) are propagated to the parent membrane. The dissolution of the skin membrane is not allowed. 

  Active P systems differ from classic (passive) P systems in their ability to create new membranes by rules of the third form. Such rule will create new child membrane with a given label $j$ and a given set of objects $v_1$ as its contents, while the set $v_2$ is the set of products that stays in the current membrane. If the current membrane already contains a child membrane with label $j$, then such rule is not applicable.

  % applicable rule definition

  For a sequential active set P system $(\Sigma, C_0, R_1, R_2, \dots , R_m)$, configuration $C = (T, l, c)$, membrane $d\in V(T)$ the rule $r\in R_{l(d)}$ is {\bf applicable} iff:
  \begin{itemize}
    \item $r = u\rightarrow w$ and $u\subseteq c(d)$ and for all $(a,\downarrow_k)\in w$ there exists $d_2\in V(T)$ such that $l(d_2)=k \wedge parent(d_2) = d$,
    \item $r = u\rightarrow w\delta$ and $u\subseteq c(d)$ and for all $(a,\downarrow_k)\in w$ there exists $d_2\in V(T)$ such that $l(d_2)=k \wedge parent(d_2) = d$ and $d\neq r_T$,
    \item $r = u\rightarrow [_j v_1]_j v_2$, $u\subseteq c(d)$ and $d$ has no child membrane with label $j$ ($\forall d_2\in V(T):  parent(d_2)=d \Rightarrow l(d_2)\neq j$).
  \end{itemize}

  In this section we assume only sequential systems, so in each step of the computation, there is one rule nondeterministically chosen among all applicable rules in all membranes to be applied.

  % Computation step

  A {\bf computation step} of a sequential active P system is a relation $\Rightarrow$ on the set of membrane configurations such that $C_1 \Rightarrow C_2$ holds iff there is an applicable rule in a membrane in $C_1$, such that applying that rule results in $C_2$.

  % Result of a computation

  The P system can work in generating or in accepting mode. For the generating mode we consider the concatenation of the objects which leave the system, in the order they are sent out of the skin membrane (if several symbols are expelled at the same time, then any ordering of them is considered). In this case we generate a language. The result of a single computation is clearly only one multiset or a string, but for one initial configuration there can be multiple possible computations. It follows from the fact that there can be more than one applicable rule in each configuration and they are chosen nondeterministically.

  For the accepting mode the input word is encoded into a membrane structure by a given encoding and it is accepted if and only if a given accepting configuration can be reached\cite{Ibarra05Active}.

% subsection alternative_definition_of_active_p_systems (end)

\subsection{Simulation of register machine} % (fold)
\label{sub:simulation_of_register_machine}
  In this section we will show that sequential active set P systems are powerful computing devices as they can simulate the register machine. We start the formulation of the main theorem followed by a proof made by a simulation. At last the efficiency of the simulation is questioned and various improvements are proposed.

  \subsubsection{Simple simulation} % (fold)
  \label{ssub:simple_simulation}
    
    The main theorem is stated as follows:

    \begin{veta}
      Sequential active set P systems are computationally complete.
    \end{veta}

    \begin{dokaz}
      Computational completeness is proved by a direct simulation of a register machine, which is also computationally complete.

      For a register machine $(m,P,i,h,Lab)$ we will construct a sequential active set P system $(\Sigma, C_0, R_1, \ldots R_{m+1})$, where $$\Sigma = \{x_j, y_j \text{~for~} j\in Lab\}\cup\{t_i \text{~for each register ~}i\}$$. Skin membrane will be labeled with $m+1$ and labels 1 to $m$ are reserved for membranes representing registers 1 to $m$. $C_0$ will be the input word for the register machine encoded into a membrane structure by the following encoding: 

      For a configuration of register machine $(r_1, r_2, \ldots r_m, ip)$ the membrane structure will consist of a skin membrane, which will contain $m$ chains consisting of $r_i$ membranes embedded one into another like in a Matryoshka doll with label $i$. The innermost membranes will contain a single object $t_i$. If $r_i = 0$ then $t_i$ is in the skin membrane and there is no membrane with label $i$. Object representing the label of the current instruction ($x_{ip}$) is in the skin membrane.
      \begin{example}
        For a configuration of register machine $(2,0,1)$ the membrane structure is $[_3 [_1 [_1 t_1 ]_1 ]_1 t_2 x_1 ]_3$. The register 1 has value 2, so there are 2 nested membranes with label 1 and the innermost membrane contains an object $t_1$. The register 2 has value 0 so there is object $t_2$ is in the skin membrane with no membrane labeled with 2. In the skin membrane the object $x_1$ represents the current instruction label 1.
      \end{example}

      We will have following rules in the skin membrane:
      \begin{enumerate}
        \item\label{simple_skin_next_instruction} $y_j \rightarrow x_j$,
        \item\label{simple_skin_send_down} $x_j \rightarrow x_j\downarrow_{i}$ for instruction $j: (op(i), \_, \_)$,
        \item\label{simple_skin_create} $x_j, t_i \rightarrow [_i y_k, t_i ]_i$ for instruction $j: (add(i), k, k)$,
        \item\label{simple_skin_sub_empty} $x_j, t_i \rightarrow x_l, t_i$ for instruction $j: (sub(i), \_, l)$
      \end{enumerate}

      For the membrane $i$:
      \begin{enumerate}[resume]
        \item\label{simple_inner_resend_down} $x_j \rightarrow x_j\downarrow_{i}$ for instruction $j: (op(i), \_, \_)$,
        \item\label{simple_inner_create} $x_j, t_i \rightarrow [_1 y_k, t_i ]_1$ for instruction $j: (add(i), k, k)$,
        \item\label{simple_inner_resend_up} $y_j \rightarrow y_j\uparrow$ for instruction $j: (op(i), \_, \_)$,
        \item\label{simple_inner_dissolve} $x_j, t_i \rightarrow y_k, t_i, \delta$ for instruction $j: (sub(i), k, l)$
      \end{enumerate}

      Object $x_j$ represents the instruction currently executed. It is sent down the chain of membranes by rules \ref{simple_skin_send_down} and \ref{simple_inner_resend_down}. In the innermost membrane the creation of a new membrane (rule \ref{simple_inner_create}), or the dissolution (rule \ref{simple_inner_dissolve}) is performed. Then the next instruction represented by object $y_j$ is sent upwards all the way to the skin membrane by the rule \ref{simple_inner_resend_up}. The object $t_i$ is always present in the innermost membrane. For a SUB instruction there are two rules in the skin membrane, together they implement the zero-test. The rule \ref{simple_skin_send_down} is applicable only if the register is nonempty and the rule \ref{simple_skin_sub_empty} is applicable only if the register is empty by requiring the presence of $t_i$, meaning that the value of register $i$ is zero.
    \end{dokaz}

    \begin{example}
      Assume a register machine with two registers with values $r_1=2$, $r_2=0$ and instructions:
      \begin{itemize}
        \item $1: sub(1),2,3$
        \item $2: add(2),1,1$
        \item $3: halt$
      \end{itemize}

      The initial configuration is $[_3 [_1 [_1 t_1 ]_1 ]_1 t_2 x_1 ]_3$. The computation of the P system is deterministic, at each step there is only one applicable rule. Starting with the rule \ref{simple_skin_send_down} and then the rule \ref{simple_inner_resend_down}, $x_1$ enters the innermost membrane, where the rule \ref{simple_inner_dissolve} dissolves the membrane. The full example computation:

      \begin{enumerate}
        \item $[_3 [_1 [_1 t_1 ]_1 ]_1 t_2 x_1 ]_3$
        \item $[_3 [_1 [_1 t_1 ]_1 x_1 ]_1 t_2 ]_3$ (used rule \ref{simple_skin_send_down})
        \item $[_3 [_1 [_1 t_1 x_1 ]_1 ]_1 t_2 ]_3$ (used rule \ref{simple_inner_resend_down})
        \item $[_3 [_1 t_1 y_2 ]_1 t_2 ]_3$ (used rule \ref{simple_inner_dissolve})
        \item $[_3 [_1 t_1 ]_1 t_2 y_2 ]_3$ (used rule \ref{simple_inner_resend_up})
        \item $[_3 [_1 t_1 ]_1 t_2 x_2 ]_3$ (used rule \ref{simple_skin_next_instruction})
        \item $[_3 [_1 t_1 ]_1 [_2 t_2 y_1 ]_2 ]_3$ (used rule \ref{simple_skin_create})
        \item $[_3 [_1 t_1 ]_1 [_2 t_2 ]_2 y_1 ]_3$ (used rule \ref{simple_inner_resend_up})
        \item $[_3 [_1 t_1 ]_1 [_2 t_2 ]_2 x_1 ]_3$ (used rule \ref{simple_skin_next_instruction})
        \item $[_3 [_1 t_1 x_1 ]_1 [_2 t_2 ]_2 ]_3$ (used rule \ref{simple_skin_send_down})
        \item $[_3 [_2 t_2 ]_2 t_1 y_2 ]_3$ (used rule \ref{simple_inner_dissolve})
        \item $[_3 [_2 t_2 ]_2 t_1 x_2 ]_3$ (used rule \ref{simple_skin_next_instruction})
        \item $[_3 [_2 t_2 x_2 ]_2 t_1 ]_3$ (used rule \ref{simple_skin_send_down})
        \item $[_3 [_2 [_2 t_2 y_1 ]_2 ]_2 t_1 ]_3$ (used rule \ref{simple_inner_create})
        \item $[_3 [_2 [_2 t_2 ]_2 y_1 ]_2 t_1 ]_3$ (used rule \ref{simple_inner_resend_up})
        \item $[_3 [_2 [_2 t_2 ]_2 ]_2 t_1 y_1 ]_3$ (used rule \ref{simple_inner_resend_up})
        \item $[_3 [_2 [_2 t_2 ]_2 ]_2 t_1 x_1 ]_3$ (used rule \ref{simple_skin_next_instruction})
        \item $[_3 [_2 [_2 t_2 ]_2 ]_2 t_1 x_3 ]_3$ (used rule \ref{simple_skin_sub_empty})
      \end{enumerate}

      We have shown an example computation of P system which simulates the register machine which moves contents of register 1 to the register 2.
    \end{example}

    The simulation of the register machine was quite straightforward. We proved that the model is computationally complete. However, the simulation is not very effective. It uses alphabet of size $2 * \text{number of instructions} + \text{number of registers}$, and its number of membranes is linearly dependent on the sum of values of registers. The time needed for executing an instruction on register $i$ is linearly dependent on $r_i$.

  % subsubsection simple_simulation (end)

  \subsubsection{Optimalization of the simulation} % (fold)
  \label{ssub:optimalization_of_the_simulation}
    In this subsection we address the inefficient usage of membranes in the previous simulation. New, optimized simulation will reduce it to logarithmic dependency.

    For a register machine $(m,P,i,h,Lab)$ we will construct a sequential active set P system, where $\Sigma = \{0,1,p,q,s,t\}\cup\{x_j, y_j, z_j \text{~for~} j\in Lab\}$. Skin membrane will be labeled with $m+1$ and labels 1 to $m$ are reserved for membranes representing registers 1 to $m$.

    Assume configuration of register machine $(r_1, r_2, \ldots r_m, ip)$. For register $i$, let $b_1b_2\ldots b_k$ be a binary representation of $r_i$. The skin membrane will contain a chain of $k$ membranes embedded one into another like in a Matryoshka doll with label $i$.
    The membrane in depth $d$ will contain the object $b_{k-d}$, which is either 0 or 1. So the highest-order position in the binary number is represented by the innermost membrane and positions which are more often changed by increments are in membranes closer to the skin membrane. Moreover, the innermost membranes contain a single object $t$. The skin membrane contains the label of the current instruction $x_{ip}$. Other membranes (not skin and not innermost) contain $s$. Direct children of the skin membrane will contain a single object $q$ representing the fact that we don't want the membrane to be dissolved. Object $p$ will be in all membranes except the skin membrane and direct children of the skin membrane. It represents the fact that the membrane can be dissolved, while keeping at least one membrane for binary representation of the register value.

    The basic idea is to recursively decide the next action based on lowest position. For incrementing number ending with zero, incrementing the lowest position is enough. Similar simple case is when decrementing number ending with one. For incrementing number ending with one, we decrement the lowest position and recursively call increment on the binary number omitting the lowest position. Similarly, for decrementing number ending with zero, we increment the lowest position and recursively call decrement on the binary number omitting the lowest position.
    There are some special cases, like incrementing 111 to 1000 or decrementing 1000 to 111. In these cases we should change the number of membranes representing positions. There is also an exception to this - decrementing 1 to 0 does not change the number of bits.

    We will have following rules in the skin membrane:
    \begin{enumerate}
      \item\label{optim_skin_next_instruction} $y_j \rightarrow x_j$,
      \item\label{optim_skin_send_down} $x_j \rightarrow x_j\downarrow_{i}$ for instruction $j: op(i), \_, \_$
    \end{enumerate}

    For the membrane $i$ and instruction $j$:
    \begin{enumerate}[resume]
      \item\label{optim_inner_resend_up} $y_j \rightarrow y_j \uparrow$ (return the next instruction to the skin membrane).
    \end{enumerate}

    For the membrane $i$ and instruction $j: add(i),k,k$:
    \begin{enumerate}[resume]
      \item\label{optim_inner_add_1} $x_j1s \rightarrow x_j\downarrow_{i}0s$ (we decremented lower position, so we must increment higher position (011 to 100, now at 1 to 0)),
      \item\label{optim_inner_add_0} $x_j0 \rightarrow y_k \uparrow 1$ (we incremented a position and can return and proceed to the next instruction),
      \item\label{optim_inner_add_1_highest} $x_j1t \rightarrow [_i 1tp]_iy_k\uparrow 0s$ (incrementing 111 to 1000).
    \end{enumerate}

    For the membrane $i$ and instruction $j: sub(i),k,l$:
    \begin{enumerate}[resume]
      \item\label{optim_inner_sub_1} $x_j1s \rightarrow y_k \uparrow 0s$ (we found position to decrement, proceed to the next instruction),
      \item\label{optim_inner_sub_0} $x_j0s \rightarrow x_j\downarrow_i 1s$ (1000 is decremented to 0111 and now we encountered a 0),
      \item\label{optim_inner_sub_1_highest} $x_j1tp \rightarrow z_jt \delta$ (decrementing the number of bits),
      \item\label{optim_inner_sub_1_highest_1} $x_j1tq \rightarrow y_k\uparrow 0tq$ (exception to the rule \ref{optim_inner_sub_1_highest} - decrementing 1 to 0 does not decrement the number of bits),
      \item\label{optim_inner_sub_remove_s} $z_jst \rightarrow y_kt$ (after decremented the number of bits, remove $s$ in the new highest-order position),
      \item\label{optim_inner_sub_0_highest} $x_j0t \rightarrow y_l \uparrow 0t$ (trying to decrement a zero)
    \end{enumerate}

    \begin{example}
      Assume a register machine with two registers with values $r_1=3$, $r_2=0$ and the current instruction $j: add(1),k,k$. The corresponding membrane configuration is $[_3 [_1 [_1 1 t p ]_1 1 s q ]_1 [_2 0 t q ]_2 x_j ]_3$.

      The computation of the P system is deterministic, at each step there is only one applicable rule. Starting with the rule \ref{optim_skin_send_down}, $x_j$ will meet the object $1$ in the configuration $[_3 [_1 [_1 1 t p ]_1 1 x_j s q ]_1 [_2 0 t q ]_2 ]_3$. The only applicable rule then is \ref{optim_inner_add_1}, resulting in $[_3 [_1 [_1 1 t p x_j ]_1 0 s q ]_1 [_2 0 t q ]_2 ]_3$. $x_j$ now meets objects 1 and $t$, which means that the only applicable rule is \ref{optim_inner_add_1_highest}, creating a new innermost membrane and resulting in the configuration $[_3 [_1 [_1 [_1 1 t p ]_1 0 p ]_1 0 y_k s q ]_1 [_2 0 t q ]_2 ]_3$. The object $y_k$ is by the rule \ref{optim_inner_resend_up} propagated to the skin membrane, where it is prepared for the next instruction by the rule \ref{optim_skin_next_instruction}.
    \end{example}

    \begin{example}
      Assume a register machine with two registers with values $r_1 = 2$, $r_2 = 0$ and instructions:
      \begin{itemize}
        \item $1: sub(1),2,3$
        \item $2: add(2),1,1$
        \item $3: halt$
      \end{itemize}

      The initial configuration is $[_3 [_1 [_1 1 t p ]_1 0 s q ]_1 [_2 0 t q ]_2 x_1 ]_3$. The computation of the P system is deterministic, at each step there is only one applicable rule. Starting with the rule \ref{optim_skin_send_down} and then the rule \ref{optim_inner_sub_0}, $x_1$ enters the innermost membrane, where the 
      rule \ref{optim_inner_sub_1_highest} dissolves the membrane. The full example computation:

      \begin{enumerate}
        \item $[_3 [_1 [_1 1 t p ]_1 0 s q ]_1 [_2 0 t q ]_2 x_1 ]_3$
        \item $[_3 [_1 [_1 1 t p ]_1 0 s q x_1 ]_1 [_2 0 t q ]_2 ]_3$ (used rule \ref{optim_skin_send_down})
        \item $[_3 [_1 [_1 1 t p x_1 ]_1 1 s q ]_1 [_2 0 t q ]_2 ]_3$ (used rule \ref{optim_inner_sub_0})
        \item $[_3 [_1 1 s t q z_1 ]_1 [_2 0 t q ]_2 ]_3$ (used rule \ref{optim_inner_sub_1_highest})
        \item $[_3 [_1 1 t q y_2 ]_1 [_2 0 t q ]_2 ]_3$ (used rule \ref{optim_inner_sub_remove_s})
        \item $[_3 [_1 1 t q ]_1 [_2 0 t q ]_2 y_2 ]_3$ (used rule \ref{optim_inner_resend_up})
        \item $[_3 [_1 1 t q ]_1 [_2 0 t q ]_2 x_2 ]_3$ (used rule \ref{optim_skin_next_instruction})
        \item $[_3 [_1 1 t q ]_1 [_2 0 t q x_2 ]_2 ]_3$ (used rule \ref{optim_skin_send_down})
        \item $[_3 [_1 1 t q ]_1 [_2 1 t q ]_2 y_1 ]_3$ (used rule \ref{optim_inner_add_0})
        \item $[_3 [_1 1 t q ]_1 [_2 1 t q ]_2 x_1 ]_3$ (used rule \ref{optim_skin_next_instruction})
        \item $[_3 [_1 1 t q x_1 ]_1 [_2 1 t q ]_2 ]_3$ (used rule \ref{optim_skin_send_down})
        \item $[_3 [_1 0 t q ]_1 [_2 1 t q ]_2 y_2 ]_3$ (used rule \ref{optim_inner_add_1_highest})
        \item $[_3 [_1 0 t q ]_1 [_2 1 t q ]_2 x_2 ]_3$ (used rule \ref{optim_skin_next_instruction})
        \item $[_3 [_1 0 t q ]_1 [_2 1 t q x_2 ]_2 ]_3$ (used rule \ref{optim_skin_send_down})
        \item $[_3 [_1 0 t q ]_1 [_2 [_2 1 t p ]_2 0 s q ]_2 y_1 ]_3$ (used rule \ref{optim_inner_add_1_highest})
        \item $[_3 [_1 0 t q ]_1 [_2 [_2 1 t p ]_2 0 s q ]_2 x_1 ]_3$ (used rule \ref{optim_skin_next_instruction})
        \item $[_3 [_1 0 t q x_1 ]_1 [_2 [_2 1 t p ]_2 0 s q ]_2 ]_3$ (used rule \ref{optim_skin_send_down})
        \item $[_3 [_1 0 t q ]_1 [_2 [_2 1 t p ]_2 0 s q ]_2 y_3 ]_3$ (used rule \ref{optim_inner_sub_0_highest})
        \item $[_3 [_1 0 t q ]_1 [_2 [_2 1 t p ]_2 0 s q ]_2 x_3 ]_3$ (used rule \ref{optim_skin_next_instruction})
      \end{enumerate}

      We have shown an example computation of P system which simulates the register machine which moves contents of register 1 to the register 2.
    \end{example}

    One instruction of the register machine is performed by number of computational steps which is logarithmic on the value of the register the instruction is operated on. The number of membranes is logarithmic as well. The number of objects is $3 * \text{number of instructions} + 6$.
  % subsubsection optimalization_of_the_simulation (end)

  \subsubsection{Further optimalizations} % (fold)
  \label{ssub:further_optimalizations}
    Could the simulation be optimized even more? Encoding the register value to a chain of membranes is not making full use of membrane structure. There are many options for a representation of an integer by a tree. For efficient implementation of the increment and decrement instructions, we need an encoding with a property that a local change in the value of the encoding of the entire tree corresponds to a local change in the value of the encodings of its child subtrees. Stein in 1999 \cite{Stein99Plowing} proposed a boustrophedonic (``ox-plowing'') variant of Cantor pairing function. The implementation of a sequential active set P system simulating a register machine using this pairing function to encode child subtrees would be quite easy, but we would stick to the logarithmic time in the worst case (diagonal of the pairing function).
    Catalan pairing function \cite{Stanley1986EnumerativeCombinatorics} orders full binary trees by the number of nodes. The time would be logarithmic with a base 4, which is a slight improvement, but asymptotically still the same.
    Searching for a suitable encoding of a tree is a good proposal for the further study.
  % subsubsection further_optimalizations (end)

% subsection simulation_of_register_machine (end)

\subsection{Modified membrane creation semantics} % (fold)
\label{sub:modified_membrane_creation_semantics}
  
  In this subsection we will investigate the effect of other semantics of membrane creation. The previous semantics assumed an explicit membrane creation rule. If the current membrane already contains child membrane with the same label as the membrane about to be created, then the rule is not applicable, and the membrane creation is aborted. Similar behavior is in the definition of sending objects to the child membrane. If such membrane does not exist, objects cannot be sent and the rule is not applicable.

  These two behaviors are in fact complementary. It seems natural to join these two artificial rule abortions and provide a rule that will always be applicable if the precondition of left side inclusion is fulfilled.

  \subsubsection{Semantics inject-or-create} % (fold)
  \label{ssub:semantics_inject_or_create}
    
    We will first examine the case where we have no explicit membrane creation rule. Any rule which is sending some objects to child membrane labeled $j$ will create child membrane $j$ if it does not exist.

    Formally, rules can be of form:
    \begin{itemize}
      \item $u\rightarrow w$, where $u\subseteq \Sigma$, $|u|\geq 1$, $w\subseteq (\Sigma\times\{\cdot, \uparrow, \downarrow_j\})$ and $1\leq j\leq m$,
      \item a dissolving rule $u\rightarrow w\delta$, where $u\subseteq \Sigma$, $|u|\geq 1$, $w\subseteq (\Sigma\times\{\cdot, \uparrow, \downarrow_j\})$ and $1\leq j\leq m$.
    \end{itemize}

    For a sequential active set P system $(\Sigma, C_0, R_1, R_2, \dots , R_m)$, configuration $C = (T, l, c)$, membrane $d\in V(T)$ the rule $r\in R_{l(d)}$ is {\bf applicable} iff:
    \begin{itemize}
      \item $r = u\rightarrow w$ and $u\subseteq c(d)$,
      \item $r = u\rightarrow w\delta$ and $u\subseteq c(d)$ and $d\neq r_T$.
    \end{itemize}

    \begin{example}
      Assume the membrane configuration $[_1 [_2 ]_2 a ]_1$. If we apply the rule $a \rightarrow a\downarrow_2$ in the skin membrane, the object $a$ is sent to the membrane 2 because such child membrane already exists. The resulting membrane configuration is $[_1 [_2 a ]_2 ]_1$.

      If we apply the rule $a \rightarrow a\downarrow_3$ in the skin membrane, a new membrane labeled 3 is created, because such child membrane does not exist yet. The resulting membrane configuration is $[_1 [_2 ]_2 [_3 a ]_3 ]_1$.
    \end{example}

    % TODO: add picture to the example

    \begin{veta}
      Sequential active set P systems with inject-or-create semantics are computationally complete.
    \end{veta}

    \begin{dokaz}
      The simulation is essentially the same as in section \ref{ssub:optimalization_of_the_simulation}. All the rules which are sending objects into a child membrane are already assuming that the child membrane already exists. The only difference is in the rule for membrane creation: $x_j1t \rightarrow [_i 1tp]_iy_k\uparrow 0s$. This rule is applied always in the innermost membrane with no child membranes. Modified simulation will therefore use rule $x_j1t \rightarrow 1\downarrow_i t\downarrow_i p\downarrow_i y_k\uparrow 0s$, which, when applied, creates a child membrane $i$, because no such child membrane exists. It results in the same configuration as in the simulation with original semantics. 
    \end{dokaz}

    The efficiency of this simulation is essentially the same as in section \ref{ssub:optimalization_of_the_simulation}, that means logarithmic number of steps for simulating one instruction of the register machine as well as logarithmic number of membranes. The number of objects used is one less than in the simulation with original semantics: $3 * \text{number of instructions} + 5$.

  % subsubsection semantics_inject_or_create (end)

  \subsubsection{Semantics wrap-or-create} % (fold)
  \label{ssub:semantics_wrap_or_create}
    
    In this variant we stay with explicit membrane creation rule, but when the membrane with the same label is already contained in the current membrane, the rule remains applicable and the child membrane will be wrapped by a new membrane with the given contents. For example, applying the rule $a \rightarrow [_2 b ]_2 c$ in the membrane 1 of membrane structure $[_1 a [_2 d ]_2 ]_1$ would result in $[_1 c [_2 b [_2 d ]_2 ]_2 ]_1$.

    % TODO: example with a picture

    \begin{veta}
      Sequential active set P systems with wrap-or-create semantics are computationally complete.
    \end{veta}

    \begin{dokaz}
      Again, we will show how to simulate the register machine. The simulation will be similar to the one defined in subsection \ref{ssub:simple_simulation}, but with additional control objects similar to the second simulation \ref{ssub:optimalization_of_the_simulation}.

      For a register machine $(m,P,i,h,Lab)$ we will construct a sequential active set P system $(\Sigma, C_0, R_1, \ldots R_{m+1})$, where $$\Sigma = \{x_j \text{~for~} j\in Lab\}\cup\{t_i, s_i \text{~for each register ~}i\}$$. Skin membrane will be labeled with $m+1$ and labels 1 to $m$ are reserved for membranes representing registers 1 to $m$. $C_0$ will be the input word for the register machine encoded into a membrane structure by the following encoding: 

      For a configuration of register machine $(r_1, r_2, \ldots r_m, ip)$ the membrane structure will consist of a skin membrane, which will contain $m$ chains consisting of $r_i$ membranes embedded one into another like in a Matryoshka doll with label $i$. Membranes with a child labeled $i$ will contain a single object $s_i$. If the membrane has no child labeled $i$, it contains an object $t_i$. If $r_i = 0$ then $t_i$ is in the skin membrane and there is no membrane with label $i$. Object representing the label of the current instruction ($x_{ip}$) is in the skin membrane.

      We will have following rules in the skin membrane:
      \begin{enumerate}
        \item\label{wrap_skin_add_s} $x_j s_i\rightarrow [_i s_i ]_i s_i x_k$ for instruction $j: (add(i), k, \_)$,
        \item\label{wrap_skin_add_t} $x_j t_i\rightarrow [_i t_i ]_i s_i x_k$ for instruction $j: (add(i), k, \_)$,
        \item\label{wrap_skin_sub_t} $x_j t_i\rightarrow x_l t_i$ for instruction $j: (sub(i), k, l)$,
        \item\label{wrap_skin_sub_s} $x_j s_i\rightarrow x_j\downarrow_i$ for instruction $j: (sub(i), k, l)$,
      \end{enumerate}

      For the membrane $i$:
      \begin{enumerate}[resume]
        \item\label{wrap_inner_dissolve} $x_j \rightarrow x_k\delta$
      \end{enumerate}

      For every $add$ instruction there is just one rule applied in the simulation and for each $sub$ instruction there is one or two instructions, depending on the register value. If $r_i>0$ then the instruction enters the membrane labeled $i$ and dissolves it, decreasing the number of stacked membranes with label $i$.
    \end{dokaz}

    \begin{example}
      Assume a register machine with two registers with values $r_1=2$ and $r_2=0$ and instructions:
      \begin{itemize}
        \item $1: sub(1),2,3$
        \item $2: add(2),1,1$
        \item $3: halt$
      \end{itemize}.

      The initial membrane configuration is $[_3 [_1 [_1 t_1 ]_1 s_1 ]_1 s_1 t_2 x_1 ]_3$. The computation of the P system is deterministic, at each step there is only one applicable rule. Starting with the rule \ref{wrap_skin_sub_s}, $x_1$ enters the inner membrane, where the rule \ref{wrap_inner_dissolve} dissolves the membrane. The full example computation:

      \begin{enumerate}
        \item $[_3 [_1 [_1 t_1 ]_1 s_1 ]_1 s_1 t_2 x_1 ]_3$
        \item $[_3 [_1 [_1 t_1 ]_1 s_1 x_1 ]_1 t_2 ]_3$ (used rule \ref{wrap_skin_sub_s})
        \item $[_3 [_1 t_1 ]_1 s_1 t_2 x_2 ]_3$ (used rule \ref{wrap_inner_dissolve})
        \item $[_3 [_1 t_1 ]_1 [_2 t_2 ]_2 s_1 s_2 x_1 ]_3$ (used rule \ref{wrap_skin_add_t})
        \item $[_3 [_1 t_1 x_1 ]_1 [_2 t_2 ]_2 s_2 ]_3$ (used rule \ref{wrap_skin_sub_s})
        \item $[_3 [_2 t_2 ]_2 t_1 s_2 x_2 ]_3$ (used rule \ref{wrap_inner_dissolve})
        \item $[_3 [_2 [_2 t_2 ]_2 s_2 ]_2 t_1 s_2 x_1 ]_3$ (used rule \ref{wrap_skin_add_s})
        \item $[_3 [_2 [_2 t_2 ]_2 s_2 ]_2 t_1 s_2 x_3 ]_3$ (used rule \ref{wrap_skin_sub_t})
      \end{enumerate}

      We have shown an example computation of P system which simulates the register machine which moves contents of register 1 to the register 2.
    \end{example}

    This simulation is the most suitable for simulating the register machine because for incrementing the number of stacked membranes we just need to wrap the topmost membrane into a new membrane. This gained us constant time for executing one instruction, however the number of membranes remains linear on the sum of register values. The number of objects is $\text{number of instructions} + 2 * \text{number of registers}$.

  % subsubsection semantics_wrap_or_create (end)

  \subsubsection{Comparison of semantics} % (fold)
  \label{ssub:comparison_of_semantics}
    
    We have investigated the bridge between membrane systems and reaction systems for the sequential variant with active membranes. We have shown computational completeness by a simulation of a register machine. When using sets instead of multisets, the original definition of creating a membrane may seem obsolete. Therefore, we have proposed alternative definitions for membrane creation: inject-or-create and wrap-or-create. In either case the resulting system has been shown to be universal.

    As some simulations are not very effective, we have also proposed ways to improve the efficiency. For the simulation with original membrane creation we managed to reduce time needed for executing one instruction of the register machine from linear to logarithmic time. The wrap-or-create semantics is the most suitable for the simulation as for every instruction of the register machine only constant number of steps of the P system is needed.

    \begin{figure}[h]
      \centering
      \begin{tabular}{|c|c|c|l|}
        \hline
        semantics & membranes & time & alphabet \\ \hline
        original & $O(n)$ & $O(n)$ & $2i+r$ \\ \hline
        original & $O(log(n))$ & $O(log(n))$ & $3i+6$ \\ \hline
        inject-or-create & $O(log(n))$ & $O(log(n))$ & $3i+6$ \\ \hline
        wrap-or-create & $O(n)$ & $O(1)$ & $i+2r$ \\ \hline
      \end{tabular}
      \caption{A comparison of different membrane creation semantics. Alphabet size in the last column depends on number of instructions $i$ and number of registers $r$.}
    \end{figure}

    The register value in the simulations is encoded in either unary or binary form. We propose other options to encode the register value as a tree structure of membranes and leave the construction of the simulation as a topic for further study.

    For the future study we suggest an investigation of decidability properties of these models as well as other inspirations from reaction systems, e. g. non-permanency of objects.

  % subsubsection comparison_of_semantics (end)

% subsection modified_membrane_creation_semantics (end)