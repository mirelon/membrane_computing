% !TEX root = ../diz.tex
The variants of P systems inspired by reaction systems introduced in subsection \ref{sub:variants_inspired_by_reaction_systems} have been inspected only for maximal parallel semantics. The various proposed semantics turned out to be deterministic and the research continued to define various semantics for computational step \cite{Kleijn11SetMembrane} and halting conditions \cite{Paun12BridgingReactionSystems}. The sequential mode was only mentioned in \cite{Kleijn11SetMembrane} under the notion of ``min-enabled'' computational step. As well as the maximal parallel mode, the sequential set membrane systems can only generate the regular languages \cite{Alhazov05WithoutMultiplicities}. The situation is changed with active membranes. It seems that it is not so black-and-white in terms of computational power of the sequential active P systems that are working with:
\begin{itemize}
   \item sets instead of multisets,
   \item the assumption of non-permanency of objects.
 \end{itemize}
 We left these proposals as a topic for the future study.