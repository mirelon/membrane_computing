\documentclass[12pt,oneside,openany,pagenumber=footcenter]{book}
%\documentclass[a4paper,10pt]{report}

\usepackage[utf8]{inputenc}
\usepackage{lmodern}
\usepackage[T1]{fontenc}
\usepackage{amsfonts}

\usepackage{slovak}
\usepackage{fontenc}
\usepackage{graphicx}
\usepackage{graphics}
\usepackage{graphicx}
\usepackage{hyperref}
\usepackage{makeidx}

\newtheorem{definicia}{Definícia}[section]
\newtheorem{HLPpoznamka}{Poznámka}[section]
\newtheorem{HLPpriklad}{Príklad}[section]
\newtheorem{HLPcvicenie}[HLPpriklad]{Cvičenie}
\newtheorem{zadanie}{Úloha}[section]
\newenvironment{poznamka}{\begin{HLPpoznamka}\rm}{\end{HLPpoznamka}}
\newenvironment{priklad}{\begin{HLPpriklad}\rm}{\end{HLPpriklad}}
\newenvironment{cvicenie}{\begin{HLPcvicenie}\rm}{\end{HLPcvicenie}}
\newtheorem{veta}{Veta}[section]
\newtheorem{lema}[veta]{Lema}
\newtheorem{dosledok}[veta]{Dôsledok}
\newtheorem{teza}[veta]{Téza}
\newtheorem{dokaz}[veta]{Dôkaz}


\pagestyle{headings}

\bibliographystyle{unsrt}

\def\indexname{Register}
% pekne pokope definujeme potrebne udaje
\def\mftitlea{Biologicky motivované výpočtové modely}
\def\mftitle{\mftitlea}
\def\mfthesistype{Dizertačná práca}
\def\mfauthor{Michal Kováč}
\def\mfadvisor{doc. RNDr. Damas Gruska, PhD.}
\def\mfplacedate{Bratislava, 2011}


\ifx\pdfoutput\undefined\relax\else\pdfinfo{ /Title (\mftitle) /Author (\mfauthor) /Creator (PDFLaTeX) } \fi

\def\eps{\varepsilon}
\def\goodgap{\hspace{\subfigcapskip}}
\makeindex

\begin{document}
\frontmatter
\thispagestyle{empty}
\begin{minipage}{0.20\textwidth}
\includegraphics[width=0.9\textwidth]{img/comenius_half.png}
\end{minipage}
\begin{minipage}{0.79\textwidth}
\begin{center}
\sc Katedra Informatiky \\
Fakulta Matematiky, Fyziky a Informatiky \\
Univerzita Komenského, Bratislava
\end{center}
\end{minipage}

\vfill
\begin{center}
\begin{minipage}{0.8\textwidth}
\hrule
\bigskip\bigskip
\centerline{\LARGE\sc\mftitlea}
\smallskip
\centerline{(\mfthesistype)}
\bigskip
\bigskip
\centerline{\large\sc\mfauthor}
\bigskip\bigskip
\hrule
\end{minipage}
\end{center}
\vfill
{\bf Vedúci:} \mfadvisor
\hfill\mfplacedate
\eject
\eject

\thispagestyle{empty}
{~}\vspace{12cm}

{~}\vspace{12cm}

\begin{minipage}{0.25\textwidth}~\end{minipage}
\begin{minipage}{0.69\textwidth}
Čestne prehlasujem, že som túto dizertačnú prácu vypracoval samostatne s použitím citovaných zdrojov.

\bigskip\bigskip

\hfill\hbox to 6cm{\dotfill}
\end{minipage}

\chapter*{Poďakovanie}
Osobitná vďaka patrí vedúcemu diplomovej práce doc. RNDr. Damasovi Gruskovi PhD. za cenné rady, námety, podnetné pripomienky a všestrannú pomoc, ktorú si hlboko vážim. Len vďaka mnohým prínosným konzultáciam a intenzívnej spolupráci som bol schopný napísať toto dielo. Nesmiem zabudnúť ani na RNDr. Branislava Rovana, CSc. a spolužiakov za to, že si na dizertačnom seminári našli čas, aby si vypočuli moju prezentáciu dizertačnej práce. Ďalšie poďakovania venujem rodičom a známym, ktorí to so mnou dokázali vydržať posledné týždne pred odovzdaním.

% !TEX root = diz.tex
\chapter*{Abstract}
\begin{description} \itemsep1pt \parskip0pt \parsep0pt
  \item[Author:] \mfauthor
  \item[Title:] \mftitle
  \item[University:] Comenius University in Bratislava
  \item[Faculty:] Faculty of Mathematics, Physics and Informatics
  \item[Department:] Department of Applied Informatics
  \item[Supervisor:] \mfadvisor
\end{description}

This work discusses the research in the membrane systems, an emerging field of natural computing. Many variants of membranes systems have already been studied, most of them uses parallel rewriting and are computationally complete. Various sequential models have been proposed, however, in many cases they are weaker than their parallel variant. We investigated several variants in terms of computational power and decidability of behavioral properties. We have also proposed own variants and suggested many topics for future study.

\begin{description}
  \item[Keywords:] Computation models inspired by biology, Membrane systems, Sequential P systems, Maximal parallelism, Computational power
\end{description}

\chapter*{Abstrakt}
\begin{description} \itemsep1pt \parskip0pt \parsep0pt
  \item[Autor:] \mfauthor
  \item[Názov dizertačnej práce:] \mftitle
  \item[Škola:] Univerzita Komenského v Bratislave
  \item[Fakulta:] Fakulta matematiky, fyziky a informatiky
  \item[Katedra:] Katedra aplikovanej informatiky
  \item[Vedúci dizertačnej práce:] \mfadvisor
  \item \mfplacedate
\end{description}

V tejto práci sa zaoberáme výskumom v oblasti membránových systémov. Veľa variantov už bolo preskúmaných, väčšinou pri výpočte používajú paralelizmus a sú Turingovsky úplné. Navrhlo sa aj veľa sekvenčných modelov, ale väčšina z nich má slabšiu výpočtovú silu ako ich paralelný variant. Preskúmali sme niekoľko variantov z hľadiska výpočtovej sily, ako aj z hľadiska rozhodnuteľnosti behaviorálnych vlastností. Navrhli sme aj vlastné varianty a naznačili smery, akými sa môže rozvíjať tento výskum.

\begin{description}
  \item[Kľúčové slová:] Výpočtové modely inšpirované biológiou, Membránové systémy, Sekvenčné P systémy, Maximálny paralelizmus, Výpočtová sila
\end{description}
\tableofcontents{}
\listoffigures{}
\listoftables{}

\mainmatter
% !TEX root = diz.tex
\addcontentsline{toc}{chapter}{Introduction}

There are a lot of areas in the theoretical computer science that are motivated by other science fields. Computation models motivated by biology forms a large group of them. They include neural networks, computational models based on DNA evolutionary algorithms, which have already found their use in computer science and proved that it is worth to be inspired by biology. L-systems are specialized for describing the growth of plants, but they have also found the applications in computer graphics, especially in fractal geometry. Other emerging areas are still awaiting for their more significant uses.

One of them is the membrane computing \cite{Paun10OxfordHandbookMembraneComputing}. It is relatively young field of natural computing - in comparison: neural networks have been researched since 1943 and membrane systems since 1998 \cite{Paun98}.

Membrane systems (P systems) are distributed parallel computing devices inspired by the structure and functionality of cells. Recently, many P system variants have been developed in order to simulate the cells more realistically or just to improve the computational power.

We will start by an introduction of various natural computing areas including models inspired by biology in Chapter \ref{cha:natural_computing}. In Chapter \ref{cha:preliminaries} we recall some computer science basic notions that we will use through the work. P systems are formally presented in Chapter \ref{cha:p_systems}, with the current state of the research in their variants, overview of software simulator MeCoSym and various case studies.

In Chapter \ref{cha:on_the_edge_of_universality_of_sequential_p_systems} we will present the current state of our work, mainly from theoretic viewpoint (computational power, decidability of behavioral properties), including the published results in sections \ref{sec:inhibitors}, \ref{sec:active_membranes} and \ref{sec:notions_from_reaction_systems}.


\chapter{Preliminaries}

\chapter{Membrane computing}

Membranes intro

Natural computing is a recent field of research wchi tries to imitate nature in the way it "computes", learning new computing models and computing paradigms experimented for billions of years by nature.

Neural networks, genetic algorithms and DNA computing are already well established research fields.

However, nature computes not only at the neural or genetic level, but also at the cellular level. In general, any non-trivial biological system has a hierarchical structure where objects and information flows between regions, what can be interpreted as a computation process.

% The notion of membrane

The regions are typically delimited by various types of membranes at different levels from cell membranes, through skin membrane to virtual membranes which delimits different parts of an ecosystem.
This hierarchical system can be seen in other field such as distributed computing, where again well delimited computing units coexist and are hierarchically arranged in complex systems from single processors to the internet.

Membranes keep together certain chemicals or information and selectively determines which of them may pass through.

% The notion of membrane structure

From these observations, Paun \cite{Paun98} introduces the notion of a membrane structure as a mathematical representation of hierarchical architectures composed of membranes. It is usually represented as a Venn diagram with all the considered sets being subsets of a unique set and not allowed to be intersected. Every two sets are either one the subset of the other, or disjoint.






P systems

\chapter{P systems}

In previous chapter we introduced the notions of membrane and membrane structure.

% Place objects in the regions.

The next step is to place certain objects in the regions delimited by the membranes. The objects are identified by their names, mathematically symbols from a given alphabet.

% Multisets of objects.

Several copies of the same object can appear in a region, so we will work with multisets of objects.

% Evolution rules

In order to obtain a computing device, we will allow the objects to evolve according to evolution rules. Any object, alone or together with another objects, can be transformed in other objects, can pass through a membrane, and can dissolve the membrane in which it is placed.

% Parallelism

All objects evolve at the same time, in parallel manner across all membranes.

% Priorities

The evolution rules are hierarchizes by a priority relation, which is a partial order.

% P system

These aspects all together forms a P system as introduced in \cite{Paun98}.

In section~\ref{sec:p-systems-definitions} we will provide formal definition of a P system.

\section{Definitions}
\label{sec:p-systems-definitions}

% Definition taken from my article

{\bf P system with active objects} is a tuple $(V, \mu, w_1, w_2,\dots , w_m, R_1, R_2, \dots , R_m)$, where:
\begin{itemize}
  \item $V$ is the alphabet of symbols,
  \item $\mu$ is a membrane structure consisting of $m$ membranes labeled with numbers $1,2,\dots,m$,
  \item $w_1,w_2,\dots w_m$ are multisets of symbols present in the regions $1,2,\dots,m$ of the membrane structure,
  \item $R_1,R_2,\dots R_m$ are finite sets of rewriting rules associated with the regions $1,2,\dots,m$ of the membrane structure. $R_i\in V^+\times\dots$
\end{itemize}

Each rewriting rule may specify for each symbol on the right side, whether it stays in the current region, moves through the membrane to the parent region ($\uparrow$)
or through membrane to all of the child regions ($\downarrow$)
or to a specific child region ($\downarrow_m$, where $m$ is a label of a membrane).
We denote these transfers with arrows immediately after the symbol.
An example of such rule is the following: $a|b|b\rightarrow a|b\downarrow |c\uparrow|c$.

A {\bf configuration} of a P system is represented by its membrane structure and the multisets of objects in the regions.

A {\bf computation step} of P system is a relation $\Rightarrow$ on the set of configurations such that $C_1 \Rightarrow C_2$ iff:

For every region in $C_1$ (suppose it contains a multiset of objects $w$) the multiset in corresponding region in $C_2$ is the result of applying a multiset of simultaneously applicable multiset rewriting rules to $w$.

In the default P system, which works in maximal parallel mode, a maximal multiset of these rules is applied in each step and region.

For example, let us have two regions with multisets $a|a$ and $b$. In the first region there is a rule $a\rightarrow b$ and in the second membrane there is a rule $b\rightarrow a|a$. The only possible result of a computation step is $b|b$, $a|a$. The first rule was applied twice and the second rule once. No more object could be consumed by rewriting rules.

A {\bf computation} of a P system consists of a sequence of steps. The step $S_i$ is applied to result of previous step $S_{i-1}$. So when $S_i = (C_j,C_{j+1})$, $S_{i-1} = (C_{j-1},C_j)$.

% Result of a computation

There are two possible ways of assigning a result of a computation:

\begin{enumerate}
    \item By considering the multiplicity of objects present in a designated membrane in a halting configuration. In this case we obtain a vector of natural numbers. We can also represent this vector as a multiset of objects or as Parikh image of a language.
    \item By concatenating the symbols which leave the system, in the order they are sent out of the skin membrane (if several symbols are expelled at the same time, then any ordering of them is considered). In this case we generate a language.
\end{enumerate}

The result of a computation is clearly only one multiset or a string, but for one initial configuration there can be multiple possible computations. It follows from the fact that there exist more than one maximal multiset of rules that can be applied in each step.

How is P system defined. Mostly taken from my article. Multiset rewriting. Accepting vs generating model, active or passive membranes, quality vs quantity aspects.

\section{P system variants}
We are interested in computation power of various variants of P systems. Especially those that are universal (Turing completeness).

\subsection{Parallelism options}
Maximal parallelism, minimal parallelism, n-parallelism, sequential models.

\subsection{Contextivity rules}
Context rules vs cooperational rules, catalytic rules, symmetric cooperational rules, catalytic rules, promoters, inhibitors, context-free rules.

\subsection{Priority rules}
\subsection{Energy of membranes}
\subsection{Calculi of Looping Sequences}

\section{Case studies}
Vultures in Pyrenees, Scavangers of Pyrenees.

% !TEX root = diz.tex
\chapter*{Conclusions}
\addcontentsline{toc}{chapter}{Conclusions}
We have studied several variants of sequential P systems in order to obtain universality without using maximal parallelism. A variant with rewriting rules that can use inhibitors was shown to be universal in both generating and accepting case. The generating model is able to simulate maximal parallel P system and the accepting model can simulate a register machine.
The constructive proof for the generating case is valuable not only for the universality, but also can be seen as a method of conversion between P systems in sequential manner and maximally parallel manner, which may be essential for future works on P systems and other multiset rewriting systems. The simulation also shows a method how to synchronize application of multiple rules in a membrane and how to synchronize this parallel rule application across whole membrane structure. Sequential variants are promising alternative to traditional maximal parallel variants and will be good subject for the further research. Future plans include research of other more restricted variants such as omitting cooperation in the rules or restricting the power of inhibitors.

In addition, we have defined a new variants of zero-testing, aiming to fit in layers between mere reformulations of the basic sequential P system and universal sequential P systems with inhibitors and possibly to reveal some unexpected connection with other models of computation. We studies variants with various forms of detection of empty membranes - a notion specific for membrane systems. The results obtained have been just the computational completeness. However, one variant with objects avoiding empty regions is more promising for our goal because the standard contruction of register machine do not work. We conjecture this variant is not universal, possibly equivalent with Petri nets or other model of computation weaker than Turing machine.

There are many features not yet combined, so we suggest them for the further research (non-cooperative rules, rules with priorities, decaying objects, deterministic steps, \ldots).

Aside from the research of the computational power, there are many open problems in the area of decision problems of certain properties. Interesting ideas for future work can be taken from \cite{Bottoni06Inhibitors}. They define an abstract notion of negative application conditions for general rewriting systems, which is for multiset rewriting rendered as the usage of inhibitors. Although they considered only nondeleting rules (after application of each rule the resulting multiset is a superset of the current multiset), interesting results were shown that the termination of rewriting was shown to be decidable.

We have investigated the decidability problems of existence of (in)finite computation for a universal class of P systems with active membranes. We have shown and published our results that are on both sides of the decidability barrier. Regarding the open problem stated in \cite{Ibarra05Active} about sequential active P systems with hard membranes (without communication between membranes), it could be interesting to find a connection between the universality and decidability of these termination problems.

We research sequential P systems with active membranes also in combination with notions inspired by reaction systems. Variants using sets instead of multisets are shown to be computational complete. We have provided a proof by a simulation of a register machine. We have proposed alternative definitions for membrane creation: inject-or-create and wrap-or-create. In either case the resulting system has been shown to be universal. We suggest investigating of decidability properties of these models as well as other inspirations from reaction systems, e. g. non-permanency of objects. There are no results yet in this area and our proposals could be set as a single topic for the future study.


\backmatter

% \begin{thebibliography}{1}
\bibliography{diz}
% \end{thebibliography}

% !TEX root = diz.tex
\appendix
\ifdefined\godzilla
  \chapter*{Appendix}
  Some statistics:
  \begin{itemize}
    \item This thesis consists of 98 definitions, 6 theorems, 15 proofs, 9 lemmas, 18 figures, 19 examples, 11 chapters, 33 sections, 36 subsections, 21 subsubsections, 348 begins, 40 tex files
    \item Github repository consists of 15 milestones, 164 issues, 342 commits
    \item Editors used: vim, SublimeText, Inkscape, Gimp, Github, Google docs
    \item LaTeX packages: cmap, inputenc, lmodern, fontenc, babel, amsfonts, amsmath, msthm, mathtools, import, algorithm, noend, algpseudocode, graphicx, graphics, rotating, tikz, caption, hyperref, varioref, imakeidx, scrpage2, etoolbox, bibunits, pdfpages
    \item Grants requested: 3
    \item Grants approved: 0
    \item Papers submitted: 4
    \item Papers accepted: 2
  \end{itemize}
  \begin{sidewaysfigure}
    \centering
    \def\svgwidth{\columnwidth}
    \input{punchcard.pdf_tex}
    \caption{Github punchcard}
  \end{sidewaysfigure}
\fi


\printindex

\end{document}
