\documentclass[12pt,oneside,openany,pagenumber=footcenter]{book}
%\documentclass[a4paper,10pt]{report}

\usepackage[utf8]{inputenc}
\usepackage{lmodern}
\usepackage[T1]{fontenc}
\usepackage{amsfonts}

\usepackage{slovak}
\usepackage{fontenc}
\usepackage{graphicx}
\usepackage{graphics}
\usepackage{graphicx}
\usepackage{hyperref}
\usepackage{makeidx}

\newtheorem{definition}{Definition}[section]
\newtheorem{HLPpoznamka}{Note}[section]
\newtheorem{HLPpriklad}{Example}[section]
\newtheorem{HLPcvicenie}[HLPpriklad]{Exercise}
\newtheorem{HLPdokaz}{Proof}[section]
\newtheorem{zadanie}{Task}[section]
\newenvironment{poznamka}{\begin{HLPpoznamka}\rm}{\end{HLPpoznamka}}
\newenvironment{example}{\begin{HLPpriklad}\rm}{\end{HLPpriklad}}
\newenvironment{cvicenie}{\begin{HLPcvicenie}\rm}{\end{HLPcvicenie}}
\newenvironment{dokaz}{\begin{HLPdokaz}\rm}{\end{HLPdokaz}}
\newtheorem{veta}{Theorem}[section]
\newtheorem{lema}[veta]{Lemma}
\newtheorem{dosledok}[veta]{Corollary}
\newtheorem{teza}[veta]{Proposition}


\pagestyle{headings}

\bibliographystyle{unsrt}

\def\indexname{Register}
% pekne pokope definujeme potrebne udaje
\def\mftitlea{Biologicky motivované výpočtové modely}
\def\mftitle{\mftitlea}
\def\mfthesistype{Dizertačná práca}
\def\mfauthor{Michal Kováč}
\def\mfadvisor{doc. RNDr. Damas Gruska, PhD.}
\def\mfplacedate{Bratislava, 2011}


\ifx\pdfoutput\undefined\relax\else\pdfinfo{ /Title (\mftitle) /Author (\mfauthor) /Creator (PDFLaTeX) } \fi

\def\eps{\varepsilon}
\def\goodgap{\hspace{\subfigcapskip}}
\makeindex

\begin{document}
\frontmatter
\thispagestyle{empty}
\begin{minipage}{0.20\textwidth}
\includegraphics[width=0.9\textwidth]{img/comenius_half.png}
\end{minipage}
\begin{minipage}{0.79\textwidth}
\begin{center}
\sc Katedra Informatiky \\
Fakulta Matematiky, Fyziky a Informatiky \\
Univerzita Komenského, Bratislava
\end{center}
\end{minipage}

\vfill
\begin{center}
\begin{minipage}{0.8\textwidth}
\hrule
\bigskip\bigskip
\centerline{\LARGE\sc\mftitlea}
\smallskip
\centerline{(\mfthesistype)}
\bigskip
\bigskip
\centerline{\large\sc\mfauthor}
\bigskip\bigskip
\hrule
\end{minipage}
\end{center}
\vfill
{\bf Vedúci:} \mfadvisor
\hfill\mfplacedate
\eject
\eject

\thispagestyle{empty}
{~}\vspace{12cm}

{~}\vspace{12cm}

\begin{minipage}{0.25\textwidth}~\end{minipage}
\begin{minipage}{0.69\textwidth}
Čestne prehlasujem, že som túto dizertačnú prácu vypracoval samostatne s použitím citovaných zdrojov.

\bigskip\bigskip

\hfill\hbox to 6cm{\dotfill}
\end{minipage}

\chapter*{Poďakovanie}
Osobitná vďaka patrí vedúcemu diplomovej práce doc. RNDr. Damasovi Gruskovi PhD. za cenné rady, námety, podnetné pripomienky a všestrannú pomoc, ktorú si hlboko vážim. Len vďaka mnohým prínosným konzultáciam a intenzívnej spolupráci som bol schopný napísať toto dielo. Nesmiem zabudnúť ani na RNDr. Branislava Rovana, CSc. a spolužiakov za to, že si na dizertačnom seminári našli čas, aby si vypočuli moju prezentáciu dizertačnej práce. Ďalšie poďakovania venujem rodičom a známym, ktorí to so mnou dokázali vydržať posledné týždne pred odovzdaním.

% !TEX root = diz.tex
\chapter*{Abstract}
\begin{description} \itemsep1pt \parskip0pt \parsep0pt
  \item[Author:] \mfauthor
  \item[Title:] \mftitle
  \item[University:] Comenius University in Bratislava
  \item[Faculty:] Faculty of Mathematics, Physics and Informatics
  \item[Department:] Department of Applied Informatics
  \item[Supervisor:] \mfadvisor
\end{description}

This work discusses the research in the membrane systems, an emerging field of natural computing. Many variants of membranes systems have already been studied, most of them uses parallel rewriting and are computationally complete. Various sequential models have been proposed, however, in many cases they are weaker than their parallel variant. We investigated several variants in terms of computational power and decidability of behavioral properties. We have also proposed own variants and suggested many topics for future study.

\begin{description}
  \item[Keywords:] Computation models inspired by biology, Membrane systems, Sequential P systems, Maximal parallelism, Computational power
\end{description}

\chapter*{Abstrakt}
\begin{description} \itemsep1pt \parskip0pt \parsep0pt
  \item[Autor:] \mfauthor
  \item[Názov dizertačnej práce:] \mftitle
  \item[Škola:] Univerzita Komenského v Bratislave
  \item[Fakulta:] Fakulta matematiky, fyziky a informatiky
  \item[Katedra:] Katedra aplikovanej informatiky
  \item[Vedúci dizertačnej práce:] \mfadvisor
  \item \mfplacedate
\end{description}

V tejto práci sa zaoberáme výskumom v oblasti membránových systémov. Veľa variantov už bolo preskúmaných, väčšinou pri výpočte používajú paralelizmus a sú Turingovsky úplné. Navrhlo sa aj veľa sekvenčných modelov, ale väčšina z nich má slabšiu výpočtovú silu ako ich paralelný variant. Preskúmali sme niekoľko variantov z hľadiska výpočtovej sily, ako aj z hľadiska rozhodnuteľnosti behaviorálnych vlastností. Navrhli sme aj vlastné varianty a naznačili smery, akými sa môže rozvíjať tento výskum.

\begin{description}
  \item[Kľúčové slová:] Výpočtové modely inšpirované biológiou, Membránové systémy, Sekvenčné P systémy, Maximálny paralelizmus, Výpočtová sila
\end{description}
\tableofcontents{}
\listoffigures{}
\listoftables{}

\mainmatter
% !TEX root = diz.tex
\addcontentsline{toc}{chapter}{Introduction}

There are a lot of areas in the theoretical computer science that are motivated by other science fields. Computation models motivated by biology forms a large group of them. They include neural networks, computational models based on DNA evolutionary algorithms, which have already found their use in computer science and proved that it is worth to be inspired by biology. L-systems are specialized for describing the growth of plants, but they have also found the applications in computer graphics, especially in fractal geometry. Other emerging areas are still awaiting for their more significant uses.

One of them is the membrane computing \cite{Paun10OxfordHandbookMembraneComputing}. It is relatively young field of natural computing - in comparison: neural networks have been researched since 1943 and membrane systems since 1998 \cite{Paun98}.

Membrane systems (P systems) are distributed parallel computing devices inspired by the structure and functionality of cells. Recently, many P system variants have been developed in order to simulate the cells more realistically or just to improve the computational power.

We will start by an introduction of various natural computing areas including models inspired by biology in Chapter \ref{cha:natural_computing}. In Chapter \ref{cha:preliminaries} we recall some computer science basic notions that we will use through the work. P systems are formally presented in Chapter \ref{cha:p_systems}, with the current state of the research in their variants, overview of software simulator MeCoSym and various case studies.

In Chapter \ref{cha:on_the_edge_of_universality_of_sequential_p_systems} we will present the current state of our work, mainly from theoretic viewpoint (computational power, decidability of behavioral properties), including the published results in sections \ref{sec:inhibitors}, \ref{sec:active_membranes} and \ref{sec:notions_from_reaction_systems}.


This chapter is about some basic notions of computer science which will be used through the work. We start by defining formal languages and basic models (grammars, machines) that define language families and end by defining multiset languages.

\section{Formal languages} % (fold)
\label{sec:formal_languages}

Our study is based on the classical theory of formal languages. We will recall some definitions:

\begin{definition}
An {\bf alphabet} is a finite nonempty set of symbols.
\end{definition}

\begin{definition}
A {\bf string} over an alphabet is a finite sequence of symbols from alphabet.
\end{definition}

The length of the string $s$ is denoted by $|s|$. We denote by $V^*$ the set of all strings over an alphabet $V$. By $V^+$ = $V^* - \{\eps\}$ we denote the set of all nonempty strings over V.

\begin{definition}
A {\bf language} over the alphabet $V$ is any subset of $V^*$.
\end{definition}

\begin{definition}
A {\bf family of languages} is a set of languages.
\end{definition}


\section{Formal grammars} % (fold)
\label{sec:formal_grammars}

\begin{definition}
A {\bf formal grammar} is a tuple $G = (N,T,P,\sigma)$, where
\begin{itemize}
  \item $N, T$ are disjoint alphabets of non-terminal and terminal symbols,
  \item $\sigma\in N$ is the initial non-terminal,
  \item $P$ is a finite set of rewriting rules of the form $u\rightarrow v$, with $u\in (N\cup T)^*N(N\cup T)^*$ and $v\in (N\cup T)^*$.
\end{itemize}
\end{definition}

\begin{definition}
A {\bf rewriting step} in the grammar $G$ is a binary relation $\Rightarrow$ on $(N\cup T)^*$, where $x\Rightarrow y$ only if $\exists w_1, w_2\in (N\cup T)^+$ and a rule $u\rightarrow v \in P$ such that $x=w_1uw_2$ and $y=w_1vw_2$.
\end{definition}

\begin{definition}
Language defined by a grammar $G$ is a set $L(G)=\{w\in T^*|\sigma\Rightarrow w\}$.
\end{definition}

Languages that can be generated by a formal grammar are the recursively enumerable languages $RE$.

% section formal_languages (end)

% section formal_grammars (end)

\section{Chomsky hierarchy} % (fold)
\label{sec:chomsky_hierarchy}

In this section we introduce several well-known families of languages.

\begin{definition}
A {\bf regular grammar} is a formal grammar, where the rewriting rules are of the form $u\rightarrow v$, where $u\in N$ and $v\in T^*(N\cup \{\eps\})$.
\end{definition}

\begin{definition}
A {\bf regular language} is a language generated by a regular grammar. The family of regular languages is denoted $R$.
\end{definition}

\begin{definition}
A {\bf context-free grammar} is a formal grammar, where rewriting rules are of the form $u\rightarrow v$, where $u\in N$ and $v\in (N\cup T)^*$.
\end{definition}

\begin{definition}
A {\bf context-free language} is a language generated by a context-free grammar. The family of context-free languages is denoted $CF$.
\end{definition}

\begin{definition}
A {\bf context-sensitive grammar} is a formal grammar, where rewriting rules are of the form $u\rightarrow v$, where $u\in (N\cup T)^*N(N\cup T)^*$, $v\in (N\cup T)^*$ and $|u| < |v|$.
\end{definition}

\begin{definition}
A {\bf context-sensitive language} is a language generated by a context-sensitive grammar. The family of context-sensitive languages is denoted $CS$.
\end{definition}

These families of languages forms the Chomsky hierarchy by means of inclusions: $R \subset CF \subset CS \subset RE$.

% section chomsky_hierarchy (end)

\section{Matrix grammars} % (fold)
\label{sec:matrix_grammars}

\begin{definition}
A {\bf matrix grammar} is a tuple $G = (N,T,M,\sigma)$, where:
\begin{itemize}
  \item $N, T$ are disjoint alphabets of non-terminal and terminal symbols,
  \item $\sigma\in N$ is the initial non-terminal,
  \item $M$ is a finite set of matrices, which are sequences of context-free rules of the form $u\rightarrow v$, where $u\in N$ and $v\in (N\cup T)^*$.
\end{itemize}
\end{definition}

\begin{definition}
A {\bf rewriting step} $x\Rightarrow y$ holds only if there is a matrix $(u_1\rightarrow v_1, u_2\rightarrow v_2, \dots, u_n\rightarrow v_n) \in M$ such that for each $1\leq i\leq n$ the following holds: $x_i = x_i^{\prime}u_ix_i^{\prime\prime}$ and $x_{i+1} = x_i^{\prime}v_ix_i^{\prime\prime}$, where $x_i, x_i^{\prime}, x_i^{\prime\prime} \in (N\cup T)^*$ and $x_1 = x$ and $x_{n+1} = y$.
\end{definition}

\begin{example}
Consider the matrix grammar $G=(\{\sigma, X,Y\}, \{ a,b,c\}, M, \sigma)$, where $M$ contains three matrices: $[S\rightarrow XY], [X\rightarrow aXb, Y\rightarrow cY], [X\rightarrow ab, Y\rightarrow c]$. There are only context-free rules, yet the grammar generate the context-sensitive language $\{a^nb^nc^n|n\geq 1\}$.
\end{example}

The family of matrix grammars is denoted $MAT$.

It is known that $CF \subset MAT \subset RE$. Interestingly, $MAT \cap {a}^* \subset R$ (see \cite{Besozzi:PhD:2004}).

% section matrix_grammars (end)

\section{Register machines} % (fold)
\label{sec:register_machines}

% We will use the notion of register machine as defined in our article

\begin{definition}
  A {\bf $n$-register machine} is a tuple $M = (n,P,i,h)$, where:
  \begin{itemize}
    \item $n$ is the number of registers,
    \item $P$ is a set of labeled instructions of the form $j : (op(r),k,l)$, where $op(r)$ is an operation on register $r$ of $M$, and $j$, $k$, $l$ are labels from the set $Lab(M)$ (which numbers the instructions in a one-to-one manner),
    \item $i$ is the initial label, and
    \item $h$ is the final label.
  \end{itemize}
\end{definition}

The machine is capable of the following instructions:
\begin{itemize}
  \item $(add(r),k,l)$ : Add one to the contents of register $r$ and proceed to instruction $k$ or to instruction $l$; in the deterministic variants usually considered in the literature we demand $k = l$.
  \item $(sub(r),k,l)$ : If register $r$ is not empty, then subtract one from its contents and go to instruction $k$, otherwise proceed to instruction $l$.
  \item $halt$ : This instruction stops the machine. This additional instruction can only be assigned to the final label $h$.
\end{itemize}

A deterministic $m$-register machine can analyze an input $(n_1,\dots,n_m)\in N_0^m$ in registers 1 to $m$, which is recognized if the register machine finally stops by the halt instruction with all its registers being empty (this last requirement is not necessary). If the machine does not halt, the analysis was not successful.

% section register_machines (end)

\section{Lindenmayer systems} % (fold)
\label{sec:lindenmayer_systems}

In 1968, a Hungarian botanist and theoretical biologist Aristid Lindenmayer introduced \cite{Lindenmayer68} a new string rewriting algorithm named Lindenmayer systems (or L-systems for short). They are used by biologists and theoretical computer scientists to mathematically model growth processes of living organisms, especially plants. The difference with Chomsky grammars is that rewriting is parallel, not sequential.

The simplest version of L-systems assumes that the development of a cell is free of influence of other cells.
This type of L-systems is called $0L$ systems, where ``0'' stands for zero-sided communication between cells.

\begin{definition}
A $0L$ system is a triple $(\Sigma, P, \omega)$, where $\Sigma$ is an alphabet, $\omega$ is a word over $\Sigma$ and $P$ is a finite set of rewriting rules of the form $a\rightarrow x$, where $a\in\Sigma, x\in\Sigma^*$.
\end{definition}

It is assumed there is at least one rewriting rule for each letter of $\Sigma$. $0L$ system works in parallel way, so all the symbols are rewritten in each step.

\begin{example}
Consider a $0L$ system with alphabet $\Sigma = \{a,b\}$, initial word $\omega = a$ and rewriting rules $P = \{a\rightarrow b, b\rightarrow ab\}$.
Since in this system there is exactly one rule for every letter of the alphabet, the rewriting is thus deterministic and the generated words will be $\{a, b, ab, bab, abbab, \dots \}$. 
\end{example}

$1L$ systems allows the rewriting rules to include context of size 1, so it allows for rules of type $yaz\rightarrow x$.

L-systems with tables ($T$) have several sets of rewriting rules instead of just one set. At one step of the rewriting process, rules belonging to the same set have to be applied. The biological motivation for introducing tables is that one may want different rules to take care of different environmental conditions (heat, light, etc.) or of different stages of development.

\begin{definition}
An extended ($E0L$) system is a pair $G_1 = (G, \Sigma_T)$, where $G = (\Sigma, P, \omega)$ is an $0L$ system, where $\Sigma_T \subseteq \Sigma$, referred to as the terminal alphabet. The language generated by $G_1$ is defined by $L(G_1) = L(G)\cap \Sigma_T^*$.
\end{definition}

Such languages are called $E0L$ languages. $E0L$ languages with tables are called $ET0L$ languages.

It is known that $CF \subset E0L \subset ET0L \subset CS$ (see section \ref{sec:chomsky_hierarchy} for definitions of $CF$ and $CS$).
% section lindenmayer_systems (end)

\section{Semilinear sets} % (fold)
\label{sec:semilinear_sets}

% section semilinear_sets (end)

\section{Vector addition systems} % (fold)
\label{sec:vector_addition_systems}

% section vector_addition_systems (end)

\section{Petri nets} % (fold)
\label{sec:petri_nets}

% section petri_nets (end)

\section{Büchi automaton} % (fold)
\label{sec:buchi_automaton}

% section buchi_automaton (end)

\section{Calculi of looping sequences} % (fold)
\label{sec:calculi_of_looping_sequences}

% section calculi_of_looping_sequences (end)

\section{Graph theory} % (fold)
\label{sec:graph_theory}

% section graph_theory (end)

\section{Multisets} % (fold)
\label{sec:multisets}

\begin{definition}
A multiset over a set $X$ is a mapping $M: X\rightarrow \mathbb N$.
\end{definition}

We denote by $M(x), x\in X$ the multiplicity of $x$ in the multiset $M$.

\begin{definition}
The {\bf support} of a multiset $M$ is the set $supp(M)=\{x\in X|M(x)\geq 1\}$.
\end{definition}

It is the set of items with at least one occurrence.

\begin{definition}
A multiset is {\bf empty} when its support is empty.
\end{definition}

A multiset $M$ with finite support $X = \{x_1, x_2, \dots, x_n\}$ can be represented by the string $x_1^{M(x_1)}x_2^{M(x_2)}\dots x_n^{M(x_n)}$.
As elements of a multiset can also be strings, we separate them with the pipe symbol, e.g. $element|element|other\_element$.

\begin{definition}
Multiset inclusion. We say that multiset $M_1$ is included in multiset $M_2$ if $\forall x \in X: M_1(x)\leq M_2(x)$. We denote it by $M_1\subseteq M_2$.
\end{definition}

\begin{definition}
The {\bf union} of two multisets $M_1\cup M_2$ is a multiset where $\forall x \in X: (M_1\cup M_2)(x)=M_1(x)+M_2(x)$.
\end{definition}

\begin{definition}
The {\bf difference} of two multisets $M_1-M_2$ is a multiset where $\forall x \in X: (M_1-M_2)(x)=M_1(x)-M_2(x)$.
\end{definition}

\begin{definition}
Product of multiset $M$ with natural number $n\in \mathbb N$ is a multiset where $\forall x \in X: (n\cdot M)(x)=n\cdot M(x)$.  
\end{definition}

% section multisets (end)

\section{Multiset languages} % (fold)
\label{sec:multiset_languages}

The number of occurrences of a given symbol $a\in V$ in the string $w\in V^*$ is denoted by $|w|_a$.

\begin{definition}
$\Psi_V(w)=(|w|_{a_1},|w|_{a_2},\dots,|w|_{a_n})$ is called a Parikh vector associated with the string $w\in V^*$, where $V=\{a_1,a_2,\dots a_n\}$.
\end{definition}

\begin{definition}
For a language $L\subseteq V^*$, $\Psi_V(L)=\{\Psi_V(w)|w\in L\}$ is the Parikh mapping associated with $V$.
\end{definition}

\begin{example}
Consider an alphabet $V=\{a,b\}$ and a language $L=\{a, ab, ba\}$.
$\Psi_V(L)=\{(1,0), (1,1)\}$. Notice that Parikh image of $L$ has only 2 element while $L$ has 3 elements.
\end{example}

\begin{definition}
If $FL$ is a family of languages, by $PsFL$ we denote the family of Parikh images of languages in $FL$.
\end{definition}

% section multiset_languages (end)

\section{Bisimulations} % (fold)
\label{sec:bisimulations}
\begin{definition}
  A {\bf state transition system} is a pair $(S, \rightarrow)$, where $S$ is a set of states and $\rightarrow\subseteq S\times S$ is a binary transition relation over $S$.
\end{definition}
  If $p,q\in S$, then $(p,q)\in \rightarrow$ is usually written as $p\rightarrow q$. This represents the fact that there is a transition from state $p$ to state $q$.

\begin{definition}
  A {\bf labelled state transition system} (LTS) is a tuple $(S, A, \rightarrow)$, where $S$ is a set of states, $A$ is a set of labels and $\rightarrow\subseteq S\times A\times S$ is a ternary transition relation.
\end{definition}
  If $p,q\in S$ and $a\in A$, then $(p,a,q)\in \rightarrow$ is usually written as $p\xrightarrow{a} q$. This represents the fact that there is a transition from state $p$ to state $q$ with a label $a$.

\begin{definition}
  Let $(S_1, A, \rightarrow)$ and $(S_2, A, \rightarrow)$ be two labelled transition systems.
  A {\bf simulation} is a binary relation $R\subseteq S_1\times S_2$ such that if $(s_1,s_2)\in R$ then for each $s_1\xrightarrow{a} t_1$ there is some $s_2\xrightarrow{a} t_2$ such that $(t_1, t_2)\in R$.
\end{definition}

\begin{definition}
  Let $(S_1, A, \rightarrow)$ and $(S_2, A, \rightarrow)$ be two labelled transition systems.
  A {\bf bisimulation} is a binary relation $R\subseteq S_1\times S_2$ such that if $(s_1,s_2)\in R$ then:
  \begin{enumerate}
    \item for each $s_1\xrightarrow{a} t_1$ there is some $s_2\xrightarrow{a} t_2$ such that $(t_1, t_2)\in R$,
    \item for each $s_2\xrightarrow{a} t_2$ there is some $s_1\xrightarrow{a} t_1$ such that $(t_1, t_2)\in R$.
  \end{enumerate}
\end{definition}
% section bisimulations (end)

\chapter{Membrane computing} % (fold)
\label{cha:membrane_computing}

Membranes intro

Natural computing is a recent field of research wchi tries to imitate nature in the way it "computes", learning new computing models and computing paradigms experimented for billions of years by nature.

Neural networks, genetic algorithms and DNA computing are already well established research fields.

However, nature computes not only at the neural or genetic level, but also at the cellular level. In general, any non-trivial biological system has a hierarchical structure where objects and information flows between regions, what can be interpreted as a computation process.

% The notion of membrane

The regions are typically delimited by various types of membranes at different levels from cell membranes, through skin membrane to virtual membranes which delimits different parts of an ecosystem.
This hierarchical system can be seen in other field such as distributed computing, where again well delimited computing units coexist and are hierarchically arranged in complex systems from single processors to the internet.

Membranes keep together certain chemicals or information and selectively determines which of them may pass through.

% The notion of membrane structure

From these observations, Paun \cite{Paun98} introduces the notion of a membrane structure as a mathematical representation of hierarchical architectures composed of membranes. It is usually represented as a Venn diagram with all the considered sets being subsets of a unique set and not allowed to be intersected. Every two sets are either one the subset of the other, or disjoint.

% chapter membrane_computing (end)

\chapter{P systems} % (fold)
\label{cha:p_systems}

In previous chapter we introduced the notions of membrane and membrane structure.

% Place objects in the regions.

The next step is to place certain objects in the regions delimited by the membranes. The objects are identified by their names, mathematically symbols from a given alphabet.

% Multisets of objects.

Several copies of the same object can appear in a region, so we will work with multisets of objects.

% Evolution rules

In order to obtain a computing device, we will allow the objects to evolve according to evolution rules. Any object, alone or together with another objects, can be transformed in other objects, can pass through a membrane, and can dissolve the membrane in which it is placed.

% Parallelism

All objects evolve at the same time, in parallel manner across all membranes.

% Priorities

The evolution rules are hierarchizes by a priority relation, which is a partial order.

% P system

These aspects all together forms a P system as introduced in \cite{Paun98}.

In section~\ref{sec:definitions} we will provide formal definition of a P system.

\section{Definitions} % (fold)
\label{sec:definitions}

% Definition taken from my article

{\bf P system with active objects} is a tuple $(V, \mu, w_1, w_2,\dots , w_m, R_1, R_2, \dots , R_m)$, where:
\begin{itemize}
  \item $V$ is the alphabet of symbols,
  \item $\mu$ is a membrane structure consisting of $m$ membranes labeled with numbers $1,2,\dots,m$,
  \item $w_1,w_2,\dots w_m$ are multisets of symbols present in the regions $1,2,\dots,m$ of the membrane structure,
  \item $R_1,R_2,\dots R_m$ are finite sets of rewriting rules associated with the regions $1,2,\dots,m$ of the membrane structure. $R_i\in V^+\times\dots$
\end{itemize}

Each rewriting rule may specify for each symbol on the right side, whether it stays in the current region, moves through the membrane to the parent region ($\uparrow$)
or through membrane to all of the child regions ($\downarrow$)
or to a specific child region ($\downarrow_m$, where $m$ is a label of a membrane).
We denote these transfers with arrows immediately after the symbol.
An example of such rule is the following: $a|b|b\rightarrow a|b\downarrow |c\uparrow|c$.

A {\bf configuration} of a P system is represented by its membrane structure and the multisets of objects in the regions.

A {\bf computation step} of P system is a relation $\Rightarrow$ on the set of configurations such that $C_1 \Rightarrow C_2$ iff:

For every region in $C_1$ (suppose it contains a multiset of objects $w$) the multiset in corresponding region in $C_2$ is the result of applying a multiset of simultaneously applicable multiset rewriting rules to $w$.

In the default P system, which works in maximal parallel mode, a maximal multiset of these rules is applied in each step and region.

For example, let us have two regions with multisets $a|a$ and $b$. In the first region there is a rule $a\rightarrow b$ and in the second membrane there is a rule $b\rightarrow a|a$. The only possible result of a computation step is $b|b$, $a|a$. The first rule was applied twice and the second rule once. No more object could be consumed by rewriting rules.

A {\bf computation} of a P system consists of a sequence of steps. The step $S_i$ is applied to result of previous step $S_{i-1}$. So when $S_i = (C_j,C_{j+1})$, $S_{i-1} = (C_{j-1},C_j)$.

% Result of a computation

There are two possible ways of assigning a result of a computation:

\begin{enumerate}
    \item By considering the multiplicity of objects present in a designated membrane in a halting configuration. In this case we obtain a vector of natural numbers. We can also represent this vector as a multiset of objects or as Parikh image of a language.
    \item By concatenating the symbols which leave the system, in the order they are sent out of the skin membrane (if several symbols are expelled at the same time, then any ordering of them is considered). In this case we generate a language.
\end{enumerate}

The result of a computation is clearly only one multiset or a string, but for one initial configuration there can be multiple possible computations. It follows from the fact that there exist more than one maximal multiset of rules that can be applied in each step.

Each rewriting rule may specify for each symbol on the right side, whether it stays in the current region, moves through the membrane to the parent region or through membrane to one of the child regions. An example of such rule is the following: $abb\rightarrow (a,here)(b,in)(c,out)(c,here)$.

% Configuration

A {\bf configuration} of a P system is represented by it's membrane structure and the multisets of objects in the regions.

% Step

A {\bf computation step} of P system is a relation $\Rightarrow$ on the set of configurations such that $C_1 \Rightarrow C_2$ iff:

For every region in $C_1$ (suppose it contains a multiset of objects $w$) the corresponding multiset in $C_2$ is the result of applying a multiset of maximal simultaneously applicable multiset rewriting rules in $R^{msap}_w$ to $w$.

In other words, a maximal multiset of rules is applied in each region.

For example, let's have two regions with multisets $aa$ and $b$. In the first region there is a rule $a\rightarrow b$ and in the second membrane there is a rule $b\rightarrow aa$. The only possible result of a computation step is $bb$, $aa$. The first rule was applied twice and the second rule once. No more object could be consumed by rewriting rules.

% Computation

{\bf Computation} of a P system consists of a sequence of steps. The step $S_i$ is appied to result of previous step $S_{i-1}$. So when $S_i = (C_j,C_{j+1})$, $S_{i-1} = (C_{j-1},C_j)$.

Result of computation is multiset of symbols that left the skin membrane in the configuration after the last computation step. For one initial configuration there can be multiple possible results. It follows from the fact that there exist more than one maximal multiset of rules that can be applied in each step.

P system defines a parikh image of a language: the set of possible results of computations.


% TODO: quality vs quantity aspects.


% section definitions (end)

\section{P system variants} % (fold)
\label{sec:p_system_variants}

We are interested in computation power of various variants of P systems. Especially those that are universal (Turing completeness).

\subsection{Accepting vs generating} % (fold)
\label{sub:accepting_vs_generating}

% subsection accepting_vs_generating (end)

\subsection{Active vs passive membranes} % (fold)
\label{sub:active_vs_passive_membranes}

% subsection active_vs_passive_membranes (end)


active or passive membranes


\subsection{Parallelism options} % (fold)
\label{sub:parallelism_options}

Maximal parallelism, minimal parallelism, n-parallelism, sequential models.

% subsection parallelism_options (end)

\subsection{Contextivity rules}
Context rules vs cooperational rules, catalytic rules, symmetric cooperational rules, catalytic rules, promoters, inhibitors, context-free rules.

\subsection{Priority rules}
\subsection{Energy of membranes}
\subsection{Calculi of Looping Sequences}


% section p_system_variants (end)

\section{Case studies} % (fold)
\label{sec:case_studies}

Vultures in Pyrenees, Scavangers of Pyrenees.

% section case_studies (end)

% chapter p_systems (end)

% !TEX root = diz.tex
\chapter*{Conclusions}
\addcontentsline{toc}{chapter}{Conclusions}
We have studied several variants of sequential P systems in order to obtain universality without using maximal parallelism. A variant with rewriting rules that can use inhibitors was shown to be universal in both generating and accepting case. The generating model is able to simulate maximal parallel P system and the accepting model can simulate a register machine.
The constructive proof for the generating case is valuable not only for the universality, but also can be seen as a method of conversion between P systems in sequential manner and maximally parallel manner, which may be essential for future works on P systems and other multiset rewriting systems. The simulation also shows a method how to synchronize application of multiple rules in a membrane and how to synchronize this parallel rule application across whole membrane structure. Sequential variants are promising alternative to traditional maximal parallel variants and will be good subject for the further research. Future plans include research of other more restricted variants such as omitting cooperation in the rules or restricting the power of inhibitors.

In addition, we have defined a new variants of zero-testing, aiming to fit in layers between mere reformulations of the basic sequential P system and universal sequential P systems with inhibitors and possibly to reveal some unexpected connection with other models of computation. We studies variants with various forms of detection of empty membranes - a notion specific for membrane systems. The results obtained have been just the computational completeness. However, one variant with objects avoiding empty regions is more promising for our goal because the standard contruction of register machine do not work. We conjecture this variant is not universal, possibly equivalent with Petri nets or other model of computation weaker than Turing machine.

There are many features not yet combined, so we suggest them for the further research (non-cooperative rules, rules with priorities, decaying objects, deterministic steps, \ldots).

Aside from the research of the computational power, there are many open problems in the area of decision problems of certain properties. Interesting ideas for future work can be taken from \cite{Bottoni06Inhibitors}. They define an abstract notion of negative application conditions for general rewriting systems, which is for multiset rewriting rendered as the usage of inhibitors. Although they considered only nondeleting rules (after application of each rule the resulting multiset is a superset of the current multiset), interesting results were shown that the termination of rewriting was shown to be decidable.

We have investigated the decidability problems of existence of (in)finite computation for a universal class of P systems with active membranes. We have shown and published our results that are on both sides of the decidability barrier. Regarding the open problem stated in \cite{Ibarra05Active} about sequential active P systems with hard membranes (without communication between membranes), it could be interesting to find a connection between the universality and decidability of these termination problems.

We research sequential P systems with active membranes also in combination with notions inspired by reaction systems. Variants using sets instead of multisets are shown to be computational complete. We have provided a proof by a simulation of a register machine. We have proposed alternative definitions for membrane creation: inject-or-create and wrap-or-create. In either case the resulting system has been shown to be universal. We suggest investigating of decidability properties of these models as well as other inspirations from reaction systems, e. g. non-permanency of objects. There are no results yet in this area and our proposals could be set as a single topic for the future study.


\backmatter

% \begin{thebibliography}{1}
\bibliography{diz}
% \end{thebibliography}

% !TEX root = diz.tex
\appendix
\ifdefined\godzilla
  \chapter*{Appendix}
  Some statistics:
  \begin{itemize}
    \item This thesis consists of 98 definitions, 6 theorems, 15 proofs, 9 lemmas, 18 figures, 19 examples, 11 chapters, 33 sections, 36 subsections, 21 subsubsections, 348 begins, 40 tex files
    \item Github repository consists of 15 milestones, 164 issues, 342 commits
    \item Editors used: vim, SublimeText, Inkscape, Gimp, Github, Google docs
    \item LaTeX packages: cmap, inputenc, lmodern, fontenc, babel, amsfonts, amsmath, msthm, mathtools, import, algorithm, noend, algpseudocode, graphicx, graphics, rotating, tikz, caption, hyperref, varioref, imakeidx, scrpage2, etoolbox, bibunits, pdfpages
    \item Grants requested: 3
    \item Grants approved: 0
    \item Papers submitted: 4
    \item Papers accepted: 2
  \end{itemize}
  \begin{sidewaysfigure}
    \centering
    \def\svgwidth{\columnwidth}
    \input{punchcard.pdf_tex}
    \caption{Github punchcard}
  \end{sidewaysfigure}
\fi


\printindex

\end{document}
