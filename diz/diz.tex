\documentclass[12pt,oneside,openany,pagenumber=footcenter]{book}
%\documentclass[a4paper,10pt]{report}

\usepackage[utf8]{inputenc}
\usepackage{lmodern}
\usepackage[T1]{fontenc}
\usepackage{amsfonts}

\usepackage{slovak}
\usepackage{fontenc}
\usepackage{graphicx}
\usepackage{graphics}
\usepackage{graphicx}
\usepackage{hyperref}
\usepackage{makeidx}

\newtheorem{definition}{Definition}[section]
\newtheorem{HLPpoznamka}{Note}[section]
\newtheorem{HLPpriklad}{Example}[section]
\newtheorem{HLPcvicenie}[HLPpriklad]{Exercise}
\newtheorem{HLPdokaz}{Proof}[section]
\newtheorem{zadanie}{Task}[section]
\newenvironment{poznamka}{\begin{HLPpoznamka}\rm}{\end{HLPpoznamka}}
\newenvironment{example}{\begin{HLPpriklad}\rm}{\end{HLPpriklad}}
\newenvironment{cvicenie}{\begin{HLPcvicenie}\rm}{\end{HLPcvicenie}}
\newenvironment{dokaz}{\begin{HLPdokaz}\rm}{\end{HLPdokaz}}
\newtheorem{veta}{Theorem}[section]
\newtheorem{lema}[veta]{Lemma}
\newtheorem{dosledok}[veta]{Corollary}
\newtheorem{teza}[veta]{Proposition}


\pagestyle{headings}

\bibliographystyle{unsrt}

\def\indexname{Register}
% pekne pokope definujeme potrebne udaje
\def\mftitlea{Biologicky motivované výpočtové modely}
\def\mftitle{\mftitlea}
\def\mfthesistype{Dizertačná práca}
\def\mfauthor{Michal Kováč}
\def\mfadvisor{doc. RNDr. Damas Gruska, PhD.}
\def\mfplacedate{Bratislava, 2011}


\ifx\pdfoutput\undefined\relax\else\pdfinfo{ /Title (\mftitle) /Author (\mfauthor) /Creator (PDFLaTeX) } \fi

\def\eps{\varepsilon}
\def\goodgap{\hspace{\subfigcapskip}}
\makeindex

\begin{document}
\frontmatter
\thispagestyle{empty}
\begin{minipage}{0.20\textwidth}
\includegraphics[width=0.9\textwidth]{img/comenius_half.png}
\end{minipage}
\begin{minipage}{0.79\textwidth}
\begin{center}
\sc Katedra Informatiky \\
Fakulta Matematiky, Fyziky a Informatiky \\
Univerzita Komenského, Bratislava
\end{center}
\end{minipage}

\vfill
\begin{center}
\begin{minipage}{0.8\textwidth}
\hrule
\bigskip\bigskip
\centerline{\LARGE\sc\mftitlea}
\smallskip
\centerline{(\mfthesistype)}
\bigskip
\bigskip
\centerline{\large\sc\mfauthor}
\bigskip\bigskip
\hrule
\end{minipage}
\end{center}
\vfill
{\bf Vedúci:} \mfadvisor
\hfill\mfplacedate
\eject
\eject

\thispagestyle{empty}
{~}\vspace{12cm}

{~}\vspace{12cm}

\begin{minipage}{0.25\textwidth}~\end{minipage}
\begin{minipage}{0.69\textwidth}
Čestne prehlasujem, že som túto dizertačnú prácu vypracoval samostatne s použitím citovaných zdrojov.

\bigskip\bigskip

\hfill\hbox to 6cm{\dotfill}
\end{minipage}

\chapter*{Poďakovanie}
Osobitná vďaka patrí vedúcemu diplomovej práce doc. RNDr. Damasovi Gruskovi PhD. za cenné rady, námety, podnetné pripomienky a všestrannú pomoc, ktorú si hlboko vážim. Len vďaka mnohým prínosným konzultáciam a intenzívnej spolupráci som bol schopný napísať toto dielo. Nesmiem zabudnúť ani na RNDr. Branislava Rovana, CSc. a spolužiakov za to, že si na dizertačnom seminári našli čas, aby si vypočuli moju prezentáciu dizertačnej práce. Ďalšie poďakovania venujem rodičom a známym, ktorí to so mnou dokázali vydržať posledné týždne pred odovzdaním.

% !TEX root = diz.tex
\chapter*{Abstract}
\begin{description} \itemsep1pt \parskip0pt \parsep0pt
  \item[Author:] \mfauthor
  \item[Title:] \mftitle
  \item[University:] Comenius University in Bratislava
  \item[Faculty:] Faculty of Mathematics, Physics and Informatics
  \item[Department:] Department of Applied Informatics
  \item[Supervisor:] \mfadvisor
\end{description}

This work discusses the research in the membrane systems, an emerging field of natural computing. Many variants of membranes systems have already been studied, most of them uses parallel rewriting and are computationally complete. Various sequential models have been proposed, however, in many cases they are weaker than their parallel variant. We investigated several variants in terms of computational power and decidability of behavioral properties. We have also proposed own variants and suggested many topics for future study.

\begin{description}
  \item[Keywords:] Computation models inspired by biology, Membrane systems, Sequential P systems, Maximal parallelism, Computational power
\end{description}

\chapter*{Abstrakt}
\begin{description} \itemsep1pt \parskip0pt \parsep0pt
  \item[Autor:] \mfauthor
  \item[Názov dizertačnej práce:] \mftitle
  \item[Škola:] Univerzita Komenského v Bratislave
  \item[Fakulta:] Fakulta matematiky, fyziky a informatiky
  \item[Katedra:] Katedra aplikovanej informatiky
  \item[Vedúci dizertačnej práce:] \mfadvisor
  \item \mfplacedate
\end{description}

V tejto práci sa zaoberáme výskumom v oblasti membránových systémov. Veľa variantov už bolo preskúmaných, väčšinou pri výpočte používajú paralelizmus a sú Turingovsky úplné. Navrhlo sa aj veľa sekvenčných modelov, ale väčšina z nich má slabšiu výpočtovú silu ako ich paralelný variant. Preskúmali sme niekoľko variantov z hľadiska výpočtovej sily, ako aj z hľadiska rozhodnuteľnosti behaviorálnych vlastností. Navrhli sme aj vlastné varianty a naznačili smery, akými sa môže rozvíjať tento výskum.

\begin{description}
  \item[Kľúčové slová:] Výpočtové modely inšpirované biológiou, Membránové systémy, Sekvenčné P systémy, Maximálny paralelizmus, Výpočtová sila
\end{description}
\tableofcontents{}
\listoffigures{}
\listoftables{}

\mainmatter
% !TEX root = diz.tex
\addcontentsline{toc}{chapter}{Introduction}

There are a lot of areas in the theoretical computer science that are motivated by other science fields. Computation models motivated by biology forms a large group of them. They include neural networks, computational models based on DNA evolutionary algorithms, which have already found their use in computer science and proved that it is worth to be inspired by biology. L-systems are specialized for describing the growth of plants, but they have also found the applications in computer graphics, especially in fractal geometry. Other emerging areas are still awaiting for their more significant uses.

One of them is the membrane computing \cite{Paun10OxfordHandbookMembraneComputing}. It is relatively young field of natural computing - in comparison: neural networks have been researched since 1943 and membrane systems since 1998 \cite{Paun98}.

Membrane systems (P systems) are distributed parallel computing devices inspired by the structure and functionality of cells. Recently, many P system variants have been developed in order to simulate the cells more realistically or just to improve the computational power.

We will start by an introduction of various natural computing areas including models inspired by biology in Chapter \ref{cha:natural_computing}. In Chapter \ref{cha:preliminaries} we recall some computer science basic notions that we will use through the work. P systems are formally presented in Chapter \ref{cha:p_systems}, with the current state of the research in their variants, overview of software simulator MeCoSym and various case studies.

In Chapter \ref{cha:on_the_edge_of_universality_of_sequential_p_systems} we will present the current state of our work, mainly from theoretic viewpoint (computational power, decidability of behavioral properties), including the published results in sections \ref{sec:inhibitors}, \ref{sec:active_membranes} and \ref{sec:notions_from_reaction_systems}.


This chapter is about some basic notions of computer science which will be used through the work. We start by defining formal languages and basic models (grammars, machines) that define language families and end by defining multiset languages.

\section{Formal languages} % (fold)
\label{sec:formal_languages}

Our study is based on the classical theory of formal languages. We will recall some definitions:

\begin{definition}
An {\bf alphabet} is a finite nonempty set of symbols.
\end{definition}

\begin{definition}
A {\bf string} over an alphabet is a finite sequence of symbols from alphabet.
\end{definition}

The length of the string $s$ is denoted by $|s|$. We denote by $V^*$ the set of all strings over an alphabet $V$. By $V^+$ = $V^* - \{\eps\}$ we denote the set of all nonempty strings over V.

\begin{definition}
A {\bf language} over the alphabet $V$ is any subset of $V^*$.
\end{definition}

\begin{definition}
A {\bf family of languages} is a set of languages.
\end{definition}


\section{Formal grammars} % (fold)
\label{sec:formal_grammars}

\begin{definition}
A {\bf formal grammar} is a tuple $G = (N,T,P,\sigma)$, where
\begin{itemize}
  \item $N, T$ are disjoint alphabets of non-terminal and terminal symbols,
  \item $\sigma\in N$ is the initial non-terminal,
  \item $P$ is a finite set of rewriting rules of the form $u\rightarrow v$, with $u\in (N\cup T)^*N(N\cup T)^*$ and $v\in (N\cup T)^*$.
\end{itemize}
\end{definition}

\begin{definition}
A {\bf rewriting step} in the grammar $G$ is a binary relation $\Rightarrow$ on $(N\cup T)^*$, where $x\Rightarrow y$ only if $\exists w_1, w_2\in (N\cup T)^+$ and a rule $u\rightarrow v \in P$ such that $x=w_1uw_2$ and $y=w_1vw_2$.
\end{definition}

\begin{definition}
Language defined by a grammar $G$ is a set $L(G)=\{w\in T^*|\sigma\Rightarrow w\}$.
\end{definition}

Languages that can be generated by a formal grammar are the recursively enumerable languages $RE$.

% section formal_languages (end)

% section formal_grammars (end)

\section{Chomsky hierarchy} % (fold)
\label{sec:chomsky_hierarchy}

In this section we introduce several well-known families of languages.

\begin{definition}
A {\bf regular grammar} is a formal grammar, where the rewriting rules are of the form $u\rightarrow v$, where $u\in N$ and $v\in T^*(N\cup \{\eps\})$.
\end{definition}

\begin{definition}
A {\bf regular language} is a language generated by a regular grammar. The family of regular languages is denoted $R$.
\end{definition}

\begin{definition}
A {\bf context-free grammar} is a formal grammar, where rewriting rules are of the form $u\rightarrow v$, where $u\in N$ and $v\in (N\cup T)^*$.
\end{definition}

\begin{definition}
A {\bf context-free language} is a language generated by a context-free grammar. The family of context-free languages is denoted $CF$.
\end{definition}

\begin{definition}
A {\bf context-sensitive grammar} is a formal grammar, where rewriting rules are of the form $u\rightarrow v$, where $u\in (N\cup T)^*N(N\cup T)^*$, $v\in (N\cup T)^*$ and $|u| < |v|$.
\end{definition}

\begin{definition}
A {\bf context-sensitive language} is a language generated by a context-sensitive grammar. The family of context-sensitive languages is denoted $CS$.
\end{definition}

These families of languages forms the Chomsky hierarchy by means of inclusions: $R \subset CF \subset CS \subset RE$.

% section chomsky_hierarchy (end)

\section{Matrix grammars} % (fold)
\label{sec:matrix_grammars}

\begin{definition}
A {\bf matrix grammar} is a tuple $G = (N,T,M,\sigma)$, where:
\begin{itemize}
  \item $N, T$ are disjoint alphabets of non-terminal and terminal symbols,
  \item $\sigma\in N$ is the initial non-terminal,
  \item $M$ is a finite set of matrices, which are sequences of context-free rules of the form $u\rightarrow v$, where $u\in N$ and $v\in (N\cup T)^*$.
\end{itemize}
\end{definition}

\begin{definition}
A {\bf rewriting step} $x\Rightarrow y$ holds only if there is a matrix $(u_1\rightarrow v_1, u_2\rightarrow v_2, \dots, u_n\rightarrow v_n) \in M$ such that for each $1\leq i\leq n$ the following holds: $x_i = x_i^{\prime}u_ix_i^{\prime\prime}$ and $x_{i+1} = x_i^{\prime}v_ix_i^{\prime\prime}$, where $x_i, x_i^{\prime}, x_i^{\prime\prime} \in (N\cup T)^*$ and $x_1 = x$ and $x_{n+1} = y$.
\end{definition}

\begin{example}
Consider the matrix grammar $G=(\{\sigma, X,Y\}, \{ a,b,c\}, M, \sigma)$, where $M$ contains three matrices: $[S\rightarrow XY], [X\rightarrow aXb, Y\rightarrow cY], [X\rightarrow ab, Y\rightarrow c]$. There are only context-free rules, yet the grammar generate the context-sensitive language $\{a^nb^nc^n|n\geq 1\}$.
\end{example}

The family of matrix grammars is denoted $MAT$.

It is known that $CF \subset MAT \subset RE$. Interestingly, $MAT \cap {a}^* \subset R$ (see \cite{Besozzi:PhD:2004}).

% section matrix_grammars (end)

\section{Register machines} % (fold)
\label{sec:register_machines}

% We will use the notion of register machine as defined in our article

\begin{definition}
  A {\bf $n$-register machine} is a tuple $M = (n,P,i,h)$, where:
  \begin{itemize}
    \item $n$ is the number of registers,
    \item $P$ is a set of labeled instructions of the form $j : (op(r),k,l)$, where $op(r)$ is an operation on register $r$ of $M$, and $j$, $k$, $l$ are labels from the set $Lab(M)$ (which numbers the instructions in a one-to-one manner),
    \item $i$ is the initial label, and
    \item $h$ is the final label.
  \end{itemize}
\end{definition}

The machine is capable of the following instructions:
\begin{itemize}
  \item $(add(r),k,l)$ : Add one to the contents of register $r$ and proceed to instruction $k$ or to instruction $l$; in the deterministic variants usually considered in the literature we demand $k = l$.
  \item $(sub(r),k,l)$ : If register $r$ is not empty, then subtract one from its contents and go to instruction $k$, otherwise proceed to instruction $l$.
  \item $halt$ : This instruction stops the machine. This additional instruction can only be assigned to the final label $h$.
\end{itemize}

A deterministic $m$-register machine can analyze an input $(n_1,\dots,n_m)\in N_0^m$ in registers 1 to $m$, which is recognized if the register machine finally stops by the halt instruction with all its registers being empty (this last requirement is not necessary). If the machine does not halt, the analysis was not successful.

% section register_machines (end)

\section{Lindenmayer systems} % (fold)
\label{sec:lindenmayer_systems}

In 1968, a Hungarian botanist and theoretical biologist Aristid Lindenmayer introduced \cite{Lindenmayer68} a new string rewriting algorithm named Lindenmayer systems (or L-systems for short). They are used by biologists and theoretical computer scientists to mathematically model growth processes of living organisms, especially plants. The difference with Chomsky grammars is that rewriting is parallel, not sequential.

The simplest version of L-systems assumes that the development of a cell is free of influence of other cells.
This type of L-systems is called $0L$ systems, where ``0'' stands for zero-sided communication between cells.

\begin{definition}
A $0L$ system is a triple $(\Sigma, P, \omega)$, where $\Sigma$ is an alphabet, $\omega$ is a word over $\Sigma$ and $P$ is a finite set of rewriting rules of the form $a\rightarrow x$, where $a\in\Sigma, x\in\Sigma^*$.
\end{definition}

It is assumed there is at least one rewriting rule for each letter of $\Sigma$. $0L$ system works in parallel way, so all the symbols are rewritten in each step.

\begin{example}
Consider a $0L$ system with alphabet $\Sigma = \{a,b\}$, initial word $\omega = a$ and rewriting rules $P = \{a\rightarrow b, b\rightarrow ab\}$.
Since in this system there is exactly one rule for every letter of the alphabet, the rewriting is thus deterministic and the generated words will be $\{a, b, ab, bab, abbab, \dots \}$. 
\end{example}

$1L$ systems allows the rewriting rules to include context of size 1, so it allows for rules of type $yaz\rightarrow x$.

L-systems with tables ($T$) have several sets of rewriting rules instead of just one set. At one step of the rewriting process, rules belonging to the same set have to be applied. The biological motivation for introducing tables is that one may want different rules to take care of different environmental conditions (heat, light, etc.) or of different stages of development.

\begin{definition}
An extended ($E0L$) system is a pair $G_1 = (G, \Sigma_T)$, where $G = (\Sigma, P, \omega)$ is an $0L$ system, where $\Sigma_T \subseteq \Sigma$, referred to as the terminal alphabet. The language generated by $G_1$ is defined by $L(G_1) = L(G)\cap \Sigma_T^*$.
\end{definition}

Such languages are called $E0L$ languages. $E0L$ languages with tables are called $ET0L$ languages.

It is known that $CF \subset E0L \subset ET0L \subset CS$ (see section \ref{sec:chomsky_hierarchy} for definitions of $CF$ and $CS$).
% section lindenmayer_systems (end)

\section{Semilinear sets} % (fold)
\label{sec:semilinear_sets}

% section semilinear_sets (end)

\section{Vector addition systems} % (fold)
\label{sec:vector_addition_systems}

% section vector_addition_systems (end)

\section{Petri nets} % (fold)
\label{sec:petri_nets}

% section petri_nets (end)

\section{Büchi automaton} % (fold)
\label{sec:buchi_automaton}

% section buchi_automaton (end)

\section{Calculi of looping sequences} % (fold)
\label{sec:calculi_of_looping_sequences}

% section calculi_of_looping_sequences (end)

\section{Graph theory} % (fold)
\label{sec:graph_theory}

% section graph_theory (end)

\section{Multisets} % (fold)
\label{sec:multisets}

\begin{definition}
A multiset over a set $X$ is a mapping $M: X\rightarrow \mathbb N$.
\end{definition}

We denote by $M(x), x\in X$ the multiplicity of $x$ in the multiset $M$.

\begin{definition}
The {\bf support} of a multiset $M$ is the set $supp(M)=\{x\in X|M(x)\geq 1\}$.
\end{definition}

It is the set of items with at least one occurrence.

\begin{definition}
A multiset is {\bf empty} when its support is empty.
\end{definition}

A multiset $M$ with finite support $X = \{x_1, x_2, \dots, x_n\}$ can be represented by the string $x_1^{M(x_1)}x_2^{M(x_2)}\dots x_n^{M(x_n)}$.
As elements of a multiset can also be strings, we separate them with the pipe symbol, e.g. $element|element|other\_element$.

\begin{definition}
Multiset inclusion. We say that multiset $M_1$ is included in multiset $M_2$ if $\forall x \in X: M_1(x)\leq M_2(x)$. We denote it by $M_1\subseteq M_2$.
\end{definition}

\begin{definition}
The {\bf union} of two multisets $M_1\cup M_2$ is a multiset where $\forall x \in X: (M_1\cup M_2)(x)=M_1(x)+M_2(x)$.
\end{definition}

\begin{definition}
The {\bf difference} of two multisets $M_1-M_2$ is a multiset where $\forall x \in X: (M_1-M_2)(x)=M_1(x)-M_2(x)$.
\end{definition}

\begin{definition}
Product of multiset $M$ with natural number $n\in \mathbb N$ is a multiset where $\forall x \in X: (n\cdot M)(x)=n\cdot M(x)$.  
\end{definition}

% section multisets (end)

\section{Multiset languages} % (fold)
\label{sec:multiset_languages}

The number of occurrences of a given symbol $a\in V$ in the string $w\in V^*$ is denoted by $|w|_a$.

\begin{definition}
$\Psi_V(w)=(|w|_{a_1},|w|_{a_2},\dots,|w|_{a_n})$ is called a Parikh vector associated with the string $w\in V^*$, where $V=\{a_1,a_2,\dots a_n\}$.
\end{definition}

\begin{definition}
For a language $L\subseteq V^*$, $\Psi_V(L)=\{\Psi_V(w)|w\in L\}$ is the Parikh mapping associated with $V$.
\end{definition}

\begin{example}
Consider an alphabet $V=\{a,b\}$ and a language $L=\{a, ab, ba\}$.
$\Psi_V(L)=\{(1,0), (1,1)\}$. Notice that Parikh image of $L$ has only 2 element while $L$ has 3 elements.
\end{example}

\begin{definition}
If $FL$ is a family of languages, by $PsFL$ we denote the family of Parikh images of languages in $FL$.
\end{definition}

% section multiset_languages (end)

\section{Bisimulations} % (fold)
\label{sec:bisimulations}
\begin{definition}
  A {\bf state transition system} is a pair $(S, \rightarrow)$, where $S$ is a set of states and $\rightarrow\subseteq S\times S$ is a binary transition relation over $S$.
\end{definition}
  If $p,q\in S$, then $(p,q)\in \rightarrow$ is usually written as $p\rightarrow q$. This represents the fact that there is a transition from state $p$ to state $q$.

\begin{definition}
  A {\bf labelled state transition system} (LTS) is a tuple $(S, A, \rightarrow)$, where $S$ is a set of states, $A$ is a set of labels and $\rightarrow\subseteq S\times A\times S$ is a ternary transition relation.
\end{definition}
  If $p,q\in S$ and $a\in A$, then $(p,a,q)\in \rightarrow$ is usually written as $p\xrightarrow{a} q$. This represents the fact that there is a transition from state $p$ to state $q$ with a label $a$.

\begin{definition}
  Let $(S_1, A, \rightarrow)$ and $(S_2, A, \rightarrow)$ be two labelled transition systems.
  A {\bf simulation} is a binary relation $R\subseteq S_1\times S_2$ such that if $(s_1,s_2)\in R$ then for each $s_1\xrightarrow{a} t_1$ there is some $s_2\xrightarrow{a} t_2$ such that $(t_1, t_2)\in R$.
\end{definition}

\begin{definition}
  Let $(S_1, A, \rightarrow)$ and $(S_2, A, \rightarrow)$ be two labelled transition systems.
  A {\bf bisimulation} is a binary relation $R\subseteq S_1\times S_2$ such that if $(s_1,s_2)\in R$ then:
  \begin{enumerate}
    \item for each $s_1\xrightarrow{a} t_1$ there is some $s_2\xrightarrow{a} t_2$ such that $(t_1, t_2)\in R$,
    \item for each $s_2\xrightarrow{a} t_2$ there is some $s_1\xrightarrow{a} t_1$ such that $(t_1, t_2)\in R$.
  \end{enumerate}
\end{definition}
% section bisimulations (end)

\chapter{Membrane computing} % (fold)
\label{cha:membrane_computing}

Recently, an interdisciplinary research between the fields of Computer Science and Biology has been rapidly growing. 

% Bioinformatics (slaves of biologists)

Bioinformatic has undergone a fast evolving process, especially the areas of genomics and proteomics. Bioinformatics can be seen as the application of computing tools and techniques for the management of biological data. Just to mention a few, the design of efficient algorithms for sequence alignment, the investigation of methods for prediction of the 3D structure of molecules and proteins and the development of data structures to effectively store huge amount of structured data.

% Natural computing (those inspired by nature)

On the other hand, the birth of biologically inspired frameworks started the investigation of mathematical models and their properties and technological requirements for their implementation by biological hardware.
Those frameworks are inspired by the nature in the way it "computes", and has gone through the evolution for billions of years.

Neural networks, genetic algorithms and DNA computing are already well established research fields.

However, nature computes not only at the neural or genetic level, but also at the cellular level. In general, any non-trivial biological system has a hierarchical structure where objects and information flows between regions, what can be interpreted as a computation process.

% The notion of membrane

The regions are typically delimited by various types of membranes at different levels from cell membranes, through skin membrane to virtual membranes which delimits different parts of an ecosystem.
This hierarchical system can be seen in other field such as distributed computing, where again well delimited computing units coexist and are hierarchically arranged in complex systems from single processors to the internet.

Membranes keep together certain chemicals or information and selectively determines which of them may pass through.

% The notion of membrane structure

From these observations, Paun \cite{Paun98} introduces the notion of a membrane structure as a mathematical representation of hierarchical architectures composed of membranes. It is usually represented as a Venn diagram with all the considered sets being subsets of a unique set and not allowed to be intersected. Every two sets are either one the subset of the other, or disjoint.

% chapter membrane_computing (end)

\chapter{P systems} % (fold)
\label{cha:p_systems}

In previous chapter we introduced the notions of membrane and membrane structure.

% Place objects in the regions.

The next step is to place certain objects in the regions delimited by the membranes. The objects are identified by their names, mathematically symbols from a given alphabet.

% Multisets of objects.

Several copies of the same object can appear in a region, so we will work with multisets of objects.

% Evolution rules

In order to obtain a computing device, we will allow the objects to evolve according to evolution rules. Any object, alone or together with another objects, can be transformed in other objects, can pass through a membrane, and can dissolve the membrane in which it is placed.

% Parallelism

All objects evolve at the same time, in parallel manner across all membranes.

% Priorities

The evolution rules are hierarchizes by a priority relation, which is a partial order.

% P system

These aspects all together forms a P system as introduced in \cite{Paun98}.

In section~\ref{sec:definitions} we will provide formal definition of a P system.

\section{Definitions} % (fold)
\label{sec:definitions}

% Definition taken from my article

{\bf P system with active objects} is a tuple $(V, \mu, w_1, w_2,\dots , w_m, R_1, R_2, \dots , R_m)$, where:
\begin{itemize}
  \item $V$ is the alphabet of symbols,
  \item $\mu$ is a membrane structure consisting of $m$ membranes labeled with numbers $1,2,\dots,m$,
  \item $w_1,w_2,\dots w_m$ are multisets of symbols present in the regions $1,2,\dots,m$ of the membrane structure,
  \item $R_1,R_2,\dots R_m$ are finite sets of rewriting rules associated with the regions $1,2,\dots,m$ of the membrane structure. $R_i\in V^+\times\dots$
\end{itemize}

Each rewriting rule may specify for each symbol on the right side, whether it stays in the current region, moves through the membrane to the parent region ($\uparrow$)
or through membrane to all of the child regions ($\downarrow$)
or to a specific child region ($\downarrow_m$, where $m$ is a label of a membrane).
We denote these transfers with arrows immediately after the symbol.
An example of such rule is the following: $a|b|b\rightarrow a|b\downarrow |c\uparrow|c$.

A {\bf configuration} of a P system is represented by its membrane structure and the multisets of objects in the regions.

A {\bf computation step} of P system is a relation $\Rightarrow$ on the set of configurations such that $C_1 \Rightarrow C_2$ iff:

For every region in $C_1$ (suppose it contains a multiset of objects $w$) the multiset in corresponding region in $C_2$ is the result of applying a multiset of simultaneously applicable multiset rewriting rules to $w$.

In the default P system, which works in maximal parallel mode, a maximal multiset of these rules is applied in each step and region.

For example, let us have two regions with multisets $a|a$ and $b$. In the first region there is a rule $a\rightarrow b$ and in the second membrane there is a rule $b\rightarrow a|a$. The only possible result of a computation step is $b|b$, $a|a$. The first rule was applied twice and the second rule once. No more object could be consumed by rewriting rules.

A {\bf computation} of a P system consists of a sequence of steps. The step $S_i$ is applied to result of previous step $S_{i-1}$. So when $S_i = (C_j,C_{j+1})$, $S_{i-1} = (C_{j-1},C_j)$.

% Result of a computation

There are two possible ways of assigning a result of a computation:

\begin{enumerate}
    \item By considering the multiplicity of objects present in a designated membrane in a halting configuration. In this case we obtain a vector of natural numbers. We can also represent this vector as a multiset of objects or as Parikh image of a language.
    \item By concatenating the symbols which leave the system, in the order they are sent out of the skin membrane (if several symbols are expelled at the same time, then any ordering of them is considered). In this case we generate a language.
\end{enumerate}

The result of a computation is clearly only one multiset or a string, but for one initial configuration there can be multiple possible computations. It follows from the fact that there exist more than one maximal multiset of rules that can be applied in each step.

Each rewriting rule may specify for each symbol on the right side, whether it stays in the current region, moves through the membrane to the parent region or through membrane to one of the child regions. An example of such rule is the following: $abb\rightarrow (a,here)(b,in)(c,out)(c,here)$.

% Configuration

A {\bf configuration} of a P system is represented by it's membrane structure and the multisets of objects in the regions.

% Step

A {\bf computation step} of P system is a relation $\Rightarrow$ on the set of configurations such that $C_1 \Rightarrow C_2$ iff:

For every region in $C_1$ (suppose it contains a multiset of objects $w$) the corresponding multiset in $C_2$ is the result of applying a multiset of maximal simultaneously applicable multiset rewriting rules in $R^{msap}_w$ to $w$.

In other words, a maximal multiset of rules is applied in each region.

For example, let's have two regions with multisets $aa$ and $b$. In the first region there is a rule $a\rightarrow b$ and in the second membrane there is a rule $b\rightarrow aa$. The only possible result of a computation step is $bb$, $aa$. The first rule was applied twice and the second rule once. No more object could be consumed by rewriting rules.

% Computation

{\bf Computation} of a P system consists of a sequence of steps. The step $S_i$ is appied to result of previous step $S_{i-1}$. So when $S_i = (C_j,C_{j+1})$, $S_{i-1} = (C_{j-1},C_j)$.

Result of computation is multiset of symbols that left the skin membrane in the configuration after the last computation step. For one initial configuration there can be multiple possible results. It follows from the fact that there exist more than one maximal multiset of rules that can be applied in each step.

P system defines a parikh image of a language: the set of possible results of computations.


% TODO: quality vs quantity aspects.


% section definitions (end)

\section{P system variants} % (fold)
\label{sec:p_system_variants}
Besozzi in his PhD thesis (see \cite{Besozzi:PhD:2004}) formulates three criteria that a good P system variant should satisfy:

\begin{enumerate}
	\item It should be as much realistic as possible from the biological point of view, in order not to widen the distance between the inspiring cellular reality and the idealized theory.
	\item It should result in computational completeness and efficiency, which would mean to obtain universal (and hence, programmable) computing devices, with a powerful and useful intrinsic parallelism;
	\item It should present mathematical minimality and elegance, to the aim of proposing an alternative framework for the analysis of computational models.
\end{enumerate}

In membrane computing, many models are equal in power with Turing machines. We should say they are Turing complete (or computationally complete), but because the proofs are always constructive, starting the constructions from these proofs from universal Turing machines or from equivalent devices, we obtain universal P systems (able
to simulate any other P system of the given type). That is why we speak about universality results, and not about computational completeness.

\subsection{Accepting vs generating} % (fold)
\label{sub:accepting_vs_generating}

In the Chomsky hierarchy, there are language acceptors (finite automata, Turing machines) and language generators (formal grammars).

% Accepting grammars

Bordhin in \cite{Bordihn99acceptingpure} extends grammars to allow for accepting languages by interchanging the left side with the right side of a rule. The mode will apply rewriting rules to an input word and accept it when it reaches the starting nonterminal. However, the input word consists of terminal symbols, which could not be rewritten when using original definition, hence they consider the pure version of various grammar types where they give up the distinction between terminal and nonterminal symbols.

% Accepting vs generating common results

The regular, context-free, context-sensitive and recursively enumerable languages were shown to have equal power in accepting and generating mode.
Some other grammars (programmed grammars with appearance checking) are shown to be more powerful in accepting mode than in generating mode.
For deterministic Lindenmayer systems, the generating and accepting mode are incomparable.

% Accepting vs generating P system results

It can be interesting to investigate accepting and generating mode also in P system variants. Barbuti in \cite{Barbuti:2010:AcceptingGenerating} shown that in the nondeterministic case, when either promoters or cooperative rules are allowed, acceptor P systems have shown to be universal. The same in known to hold for the corresponding classes of nondeterministic generator P systems. In the deterministic case, acceptor P systems have been shown to be universal only if cooperative rules are allowed. Universality has been shown not to hold for the corresponding classes of generator P systems.

% subsection accepting_vs_generating (end)

\subsection{Active vs passive membranes} % (fold)
\label{sub:active_vs_passive_membranes}

% TODO: need citations
Most of the studied P system variants assumes that the number of membranes can only decrease during a computation, by dissolving membranes as a result of applying evolution rules to the objects present in the system.
A natural possibility is to let the number of membranes also to increase during a computation, for instance, by division, as it is well-known in biology. Actually, the membranes from biochemistry are not at all passive, like those in the models briefly described above.
For example, the passing of a chemical compound through a membrane is often done by a direct interaction with the membrane itself (with the so-called protein channels or protein gates present in the membrane); during this interaction, the chemical compound which passes through membrane can be modified, while the membrane itself can in this way be modified (at least locally).

In \cite{Paun99ActiveMembranes} P\u{a}un considers P systems with active membranes where the central role in the computation is played by the membranes: evolution rules are associated both with objects and membranes, while the communication through membranes is performed with the direct participation of the membranes; moreover, the membranes can not only be dissolved, but they also can multiply by division. An elementary membrane can be divided by means of an interaction with an object from that membrane.

% Polarization

Each membrane is supposed to have an electrical polarization (we will say charge), one of the three possible: positive, negative, or neutral. If in a membrane we have two immediately lower membranes of opposite polarizations, one positive and one negative, then that membrane can also divide in such a way that the two membranes of opposite charge are separated; all membranes of neutral charge and all objects are duplicated and a copy of each of them is introduced in each of the two new membranes.
The skin is never divided.
If at the same time a membrane is divided and there are objects in this membrane which are being rewritten in the same step, then in the new copies of the membrane the result of the evolution is included.

In this way, the number of membranes can grow, even exponentially. As expected, by making use of this increased parallelism we can compute faster.
For example, the SAT problem, which is NP complete, can be solved in linear time, when we consider the steps of computation as the time units.
Moreover, the model is shown to be computationally universal.

% subsection active_vs_passive_membranes (end)

\subsection{Context in rules} % (fold)
\label{sub:context_in_rules}

% Cooperative / Non-cooperative

Rewriting rules in P systems can be cooperative and non-cooperative, like in Chomsky's context-free and context-sensitive grammars. Non-cooperative rules are restricted to use only one object on the left side and cooperative rules do not have this restriction.
P systems with cooperative rules are universal \cite{Paun98}, while P systems with non-cooperative rules only characterize Parikh image of context-free languages ($PsCF$) \cite{Sburlan05dragos}.

% Catalytic P systems

P\u{a}un \cite{Paun98} also defines P systems with catalysts where catalysts are a specified subset of the alphabet. Rewriting rules can contain catalysts, which are not modified by applying the rule. Surprisingly, P systems with catalytic rules are universal, actually two membranes in the P system are sufficient to achieve universality.

In systems where only catalytic rules (purely catalytic systems \cite{Ibarra:03:Catalytic}), three catalysts are enough \cite{Freund2005TwoCatalysts}.

% Two catalysts

Freund in \cite{Freund2005TwoCatalysts} also shows that two catalysts and one membrane are enough and raised an open problem whether one catalyst is sufficient. He conjectured that for computationally universal P systems the results obtained in this paper are optimal not only with respect to the number of membranes (2), but also with respect to the number of catalysts.

% Catalysts are too powerful

From some point of view, catalysts are way too powerful in restricting the parallelism - they directly participate in the rules, hence the number of catalytic rules that can be applied in one step, is bounded by number of catalysts.

A variant with promoters and inhibitors have been proposed (see \cite{Ionescu:jucs_10_5:on_p_systems_with}).

% Promoters

In the case of promoters, the rules are possible only in the presence of certain symbols. An object $p$ is a promoter for a rule $u\rightarrow v$ and we denote this by $u\rightarrow v|_{p}$, if the rule is active only in the presence of object $p$. Note that unlike in the case with catalysts, promoters allow the associated rules to be applied as many times as possible.

% Inhibitors

An object $i$ is inhibitor for a rule $u\rightarrow v$ and we denote this by $u\rightarrow v|_{\neg i}$, if the rule is active only if inhibitor $i$ is not present in the region.
One of our results (see section \ref{sec:inhibitors}) uses inhibitors as a tool to achieve universality for sequential P systems.


% One catalyst with promoters / inhibitors

Ionescu in \cite{Ionescu:jucs_10_5:on_p_systems_with} shows that P systems with non-cooperative catalytic rules with only one catalyst and with promoters / inhibitors are universal.

% Zero catalysts with inhibitors

Non-cooperative rules with no catalysts and with inhibitors were studied in \cite{Sburlan:2006:FurtherResultsPromotersInhibitors}, the equivalence with Lindenmayer systems ($ET0L$ as defined in section \ref{sec:lindenmayer_systems}) was proved.

% Simple cooperative system

Dang \cite{Ibarra04dang} proposes a simple cooperative system ($SCO$) as a P system where the only rules allowed are of the form $a\rightarrow v$ or of the form $aa\rightarrow v$, where $a$ is a symbol and $v$ is a (possibly null) string of symbols not containing $a$. This variant is investigated with various modes of parallelism, so their results will be mentioned in the subsection \ref{sub:parallelism_options}

% subsection context_in_rules (end)

\subsection{Rules with priorities} % (fold)
\label{sub:rules_with_priorities}

In the original definition of a P system \cite{Paun98}, a partial order relation over set of rewriting rules have been specified. The rule can be used only if no rule of a higher priority in the region can be applied at the same time.

Sos\'ik in \cite{Sosik:2002:WithoutPriorities} showed that the priorities may be omitted from the model without loss of computational power.

% subsection rules_with_priorities (end)

\subsection{Energy in P systems} % (fold)
\label{sub:energy_in_p_systems}

Various notions of energy has been proposed for use in P systems. P\u{a}un in \cite{Paun:2001:Energy} considers a P system where each evolution rule ``produces'' or ``consumes'' some quantity of energy, in amounts which are expressed as integer numbers. In each moment and in each membrane the total energy involved in an evolution step should be positive, but if ``Too much'' energy is present in a membrane, then the membrane will be destroyed (dissolved). This variant was investigated in two cases, both were shown to be universal:

\begin{enumerate}
	\item when using only two membranes and unbounded amount of energy,
	\item when using arbitrarily many membranes and a bounded energy associated with rules
\end{enumerate}

Freund in \cite{Freund:2004:SequentialEnergy} introduced a new variant where the rules are assigned directly to membranes (every rule consume objects on one side of the membrane and produce objects on the other side) and every membrane carries an energy value that can be changed during a computation by objects passing through the membrane.

This variant is universal even in sequential mode if we allow priorities on the objects. When omitting the priority relation, only the family of Parikh sets generated by context-free matrix grammars ($PsMAT$ as defined in section \ref{sec:matrix_grammars}) is obtained.

% subsection energy_in_p_systems (end)

\subsection{Symport / antiport rules} % (fold)
\label{sub:symport_antiport_rules}

P\u{a}un in \cite{Paun:2002:SymportAntiport} proposes a new way of communicating between membranes.

Symports allow two chemicals to pass together through a membrane in the same direction using symport rules of type $(ab,in)$ or $(ab,out)$.
Antiports allow two chemicals to pass simultaneously through a membrane in opposite directions using antiport rules of type $(a,in;b,out)$.

Surprisingly, a P system variant, where only the symport / antiport rules are used are computationally complete. Five membranes are enough for this result. If more than two chemicals may collaborate when passing through membranes, two membranes are sufficient for universality. These results are proven in \cite{Paun:2002:SymportAntiport}.

% subsection symport_antiport_rules (end)

\subsection{Parallelism options} % (fold)
\label{sub:parallelism_options}

Original definition of P system (see \cite{Paun98}) uses maximal parallelism when doing a step of computation. There is an obvious biological motivation relying on the assumption that ``if we wait long enough, then all reaction which may take place will take place''. This condition is rather powerful, because it decreases the non-determinism of the system's evolution. For various reasons ranging from looking for more realistic models to just the mathematical challenge, the maximal parallelism was questioned.

% Sequential mode

Dang in \cite{Dang04Sequential} investigates the sequential mode. In each step, from the set of applicable rules across all membrane one is nondeterministically chosen and applied. For purely catalytic systems with 1 membrane, the sequential mode generates only the semilinear sets and thus is strictly weaker than the maximally parallel version.
% TODO: add semilinear sets
Sequential version of symport / antiport systems are equivalent to vector addition systems making it strictly weaker than the original maximally parallel version.

Investigation of the sequential mode continues in \cite{Ibarra05Active}. Sequential P system without priorities with cooperative rules with rules for membrane dissolution are not universal by showing they can be simulated by vector addition systems with states (VASS).
This holds even when the membrane creation is allowed for bounded number of created membranes. However, if any number of membranes are allowed to be created, the system becomes universal. This result was shown by simulation of the register machine (see section \ref{sec:register_machines}).

We have further investigated this variant (sequential P system without priorities with cooperative rules) in chapter \ref{cha:on_the_edge_of_universality_of_sequential_p_systems} by allowing rules with inhibitors, which resulted in universality.


% Restricting maximal parallelism

Dang in \cite{Ibarra04dang} proposes several restricted versions of parallelism.
$n${\bf -Max-Parallel} version nondeterministically selects a maximal subset of at most n rules to apply. It is proved that 9{\bf -Max-Parallel} SCO (defined in the subsection \ref{sub:context_in_rules}) is universal.
$\leq n${\bf -Parallel} version is similar, but does not require the condition of a maximal subset of rules. It is shown to be weaker than $n${\bf -Max-Parallel} version.
$n${\bf -Parallel} version requires the size of the subset of rules to apply to be exactly $n$.
All three versions are equal to the sequential mode when $n=1$. For non-universality results, Dang used the proof technique by simulation by vector addition systems. Our future research may be inspired by this technique.

% Minimal parallelism

Ciobanu in \cite{Ciobanu:2007:MinimalParallelism} proposes a minimal parallelism: for each region if at least one rule can be applied, then at least one rule will be applied. The symport / antiport rules variant and variant with active membranes were both shown to be universal.

% Asynchronous

Freund in \cite{Freund:2004:Async} studied the asynchronous mode of P systems, where in each step, arbitrary many rules can be applied. The application of rules is hence done in parallel way, but are not synchronized or somewhat controlled. In many cases the sequential and asynchronous modes were shown to be equivalent.

% subsection parallelism_options (end)


% section p_system_variants (end)

\section{Case studies} % (fold)
\label{sec:case_studies}

Vultures in Pyrenees, Scavangers of Pyrenees.

% section case_studies (end)

% chapter p_systems (end)

% !TEX root = diz.tex
\chapter*{Conclusions}
\addcontentsline{toc}{chapter}{Conclusions}
We have studied several variants of sequential P systems in order to obtain universality without using maximal parallelism. A variant with rewriting rules that can use inhibitors was shown to be universal in both generating and accepting case. The generating model is able to simulate maximal parallel P system and the accepting model can simulate a register machine.
The constructive proof for the generating case is valuable not only for the universality, but also can be seen as a method of conversion between P systems in sequential manner and maximally parallel manner, which may be essential for future works on P systems and other multiset rewriting systems. The simulation also shows a method how to synchronize application of multiple rules in a membrane and how to synchronize this parallel rule application across whole membrane structure. Sequential variants are promising alternative to traditional maximal parallel variants and will be good subject for the further research. Future plans include research of other more restricted variants such as omitting cooperation in the rules or restricting the power of inhibitors.

In addition, we have defined a new variants of zero-testing, aiming to fit in layers between mere reformulations of the basic sequential P system and universal sequential P systems with inhibitors and possibly to reveal some unexpected connection with other models of computation. We studies variants with various forms of detection of empty membranes - a notion specific for membrane systems. The results obtained have been just the computational completeness. However, one variant with objects avoiding empty regions is more promising for our goal because the standard contruction of register machine do not work. We conjecture this variant is not universal, possibly equivalent with Petri nets or other model of computation weaker than Turing machine.

There are many features not yet combined, so we suggest them for the further research (non-cooperative rules, rules with priorities, decaying objects, deterministic steps, \ldots).

Aside from the research of the computational power, there are many open problems in the area of decision problems of certain properties. Interesting ideas for future work can be taken from \cite{Bottoni06Inhibitors}. They define an abstract notion of negative application conditions for general rewriting systems, which is for multiset rewriting rendered as the usage of inhibitors. Although they considered only nondeleting rules (after application of each rule the resulting multiset is a superset of the current multiset), interesting results were shown that the termination of rewriting was shown to be decidable.

We have investigated the decidability problems of existence of (in)finite computation for a universal class of P systems with active membranes. We have shown and published our results that are on both sides of the decidability barrier. Regarding the open problem stated in \cite{Ibarra05Active} about sequential active P systems with hard membranes (without communication between membranes), it could be interesting to find a connection between the universality and decidability of these termination problems.

We research sequential P systems with active membranes also in combination with notions inspired by reaction systems. Variants using sets instead of multisets are shown to be computational complete. We have provided a proof by a simulation of a register machine. We have proposed alternative definitions for membrane creation: inject-or-create and wrap-or-create. In either case the resulting system has been shown to be universal. We suggest investigating of decidability properties of these models as well as other inspirations from reaction systems, e. g. non-permanency of objects. There are no results yet in this area and our proposals could be set as a single topic for the future study.


\backmatter

% \begin{thebibliography}{1}
\bibliography{diz}
% \end{thebibliography}

% !TEX root = diz.tex
\appendix
\ifdefined\godzilla
  \chapter*{Appendix}
  Some statistics:
  \begin{itemize}
    \item This thesis consists of 98 definitions, 6 theorems, 15 proofs, 9 lemmas, 18 figures, 19 examples, 11 chapters, 33 sections, 36 subsections, 21 subsubsections, 348 begins, 40 tex files
    \item Github repository consists of 15 milestones, 164 issues, 342 commits
    \item Editors used: vim, SublimeText, Inkscape, Gimp, Github, Google docs
    \item LaTeX packages: cmap, inputenc, lmodern, fontenc, babel, amsfonts, amsmath, msthm, mathtools, import, algorithm, noend, algpseudocode, graphicx, graphics, rotating, tikz, caption, hyperref, varioref, imakeidx, scrpage2, etoolbox, bibunits, pdfpages
    \item Grants requested: 3
    \item Grants approved: 0
    \item Papers submitted: 4
    \item Papers accepted: 2
  \end{itemize}
  \begin{sidewaysfigure}
    \centering
    \def\svgwidth{\columnwidth}
    \input{punchcard.pdf_tex}
    \caption{Github punchcard}
  \end{sidewaysfigure}
\fi


\printindex

\end{document}
