\documentclass{article}
\usepackage[utf8]{inputenc}
\usepackage{lmodern}
\usepackage[T1]{fontenc}
\usepackage{amsfonts}
\usepackage{slovak}
\usepackage{graphicx}
\let\eps=\varepsilon
\let\then=\Rightarrow
\def\header#1#2{
\centerline{\Large\sc #1}
\medskip
\centerline{\large #2}
\vspace{-\baselineskip}\rightline{\today}
\bigskip\hrule\bigskip
}
\addtocounter{section}{1}
\begin{document}

\header{Biologicky motivované výpočtové modely}{Michal Kováč}

V teoretickej informatike je veľa oblastí, ktoré sú motivované inými vednými disciplínami. Veľkú skupinu tvoria modely motivované biológiou. Patria sem napríklad neurónové siete, DNA výpočtové modely, evolučné algoritmy, ktoré si už našli svoje významné uplatnenie v informatike a dokázali, že sa oplatí inšpirovať biológiou. Ďaľšie rozvíjajúce sa oblasti ešte čakajú na svoje uplatnenie. Jednou z nich sú membránové systémy. Je pomerne mladá oblasť - prvý článok bol publikovaný v roku 1998.

Živé organizmy sa skladajú zo sústav orgánov. Tie sa skladajú z orgánov. Orgány pozostávajú z tkanív, tie z buniek. Je to vlastne hierarchická štruktúra membrán. Každá membrána predstavuja rozhranie medzi susednými regiónmi. Oddeľuje tak konečné vnútro a nekonečný vonkajšok.

V regiónoch sa nachádzajú chemické látky, paralelne a nedeterministicky v nich prebiehajú isté procesy. Membrány môžu za istých okolností prepúšťať niektoré látky. Niektoré látky dokážu membránu zničiť tak, že sa vnútro "vyleje" von. Membránový systém (alebo P-systém) je výpočtový model, ktorý tieto procesy definuje.



\end{document}
