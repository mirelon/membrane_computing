\documentclass[llncs,submission,copyright,creativecommons]{llncs}
\providecommand{\event}{Cie-CS 2014} % Name of the event you are submitting to

\usepackage[utf8]{inputenc}
\usepackage{lmodern}
\usepackage[T1]{fontenc}
\pagestyle{headings}

\begin{document}
\title{Using Inhibitors to Achieve Universality of Sequential P Systems}
\author{Michal Kováč}
\institute{Faculty of Mathematics, Physics and Informatics, Comenius University}
\date{\today}
\maketitle

\begin{abstract}
P systems are formal models of distributed parallel multiset processing. Many variants make use of the maximal parallelism to achieve universality.
It is known that P systems with catalytic rules with only one catalyst are not universal, but when using promoters and inhibitors, the universality is achieved.
The sequential variant of P system is also not universal and we will show how the computational universality can be reached by using sequential P systems with inhibitors. Both accepting and generating case are investigated.

The talk will begin with a brief overview of P systems and how far has the research gone, followed by a presentation of sequential P systems, their properties and ways how to achieve universality. Our result is the constructive proof of generative case for sequential P systems with inhibitors. Although the accepting case is similar to proof technique for the Petri nets, the generative case is more difficult to prove. The presentation will include an overview of the simulation of maximally parallel P system.
\end{abstract}

\end{document}
