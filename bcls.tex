\documentclass[a4paper,10pt]{article}

\usepackage[utf8]{inputenc}
\usepackage{lmodern}
\usepackage[T1]{fontenc}
\usepackage{amsfonts}

\usepackage{slovak}
\usepackage{fontenc}
\usepackage{graphicx}
\usepackage{graphics}
\usepackage{graphicx}
\usepackage{hyperref}
\usepackage{makeidx}

\newtheorem{definition}{Definition}[section]
\newtheorem{HLPpoznamka}{Note}[section]
\newtheorem{HLPpriklad}{Example}[section]
\newtheorem{HLPcvicenie}[HLPpriklad]{Exercise}
\newtheorem{HLPdokaz}{Proof}[section]
\newtheorem{zadanie}{Task}[section]
\newenvironment{poznamka}{\begin{HLPpoznamka}\rm}{\end{HLPpoznamka}}
\newenvironment{example}{\begin{HLPpriklad}\rm}{\end{HLPpriklad}}
\newenvironment{cvicenie}{\begin{HLPcvicenie}\rm}{\end{HLPcvicenie}}
\newenvironment{dokaz}{\begin{HLPdokaz}\rm}{\end{HLPdokaz}}
\newtheorem{veta}{Theorem}[section]
\newtheorem{lemma}[veta]{Lemma}
\newtheorem{dosledok}[veta]{Corollary}
\newtheorem{teza}[veta]{Proposition}


\pagestyle{headings}

\def\eps{\varepsilon}
\def\goodgap{\hspace{\subfigcapskip}}

% Page margins
\addtolength{\oddsidemargin}{-.875in}
\addtolength{\evensidemargin}{-.875in}
\addtolength{\textwidth}{1.75in}

\addtolength{\topmargin}{-.875in}
\addtolength{\textheight}{1.75in}

% Itemize bulet types
\renewcommand{\labelitemi}{$\bullet$}
\renewcommand{\labelitemii}{$\cdot$}

% Compact itemize
\newenvironment{itemize*}%
  {\begin{itemize}%
    \setlength{\itemsep}{0pt}%
    \setlength{\parskip}{0pt}}%
  {\end{itemize}}

% Compact enumerate
\newenvironment{enumerate*}%
  {\begin{enumerate}%
    \setlength{\itemsep}{0pt}%
    \setlength{\parskip}{0pt}}%
  {\end{enumerate}}



\begin{document}
\title{Mebrane structure languages generated by Büchi grammar}
\author{Michal Kováč}
\date{\today}
\maketitle

\begin{abstract}
Tu bude abstract
\end{abstract}

\section{Introduction}
Last years there have been many formalisms developed by computer scientists, used to model biological systems. One of them is Calculi of Looping Sequences (CLS). In this paper, we define and study properties of languages of membrane structures generated by simplified version of rewriting rules.
\section{Büchi CLS}
In this section we define Büchichichi CLS (BCLS) as class of languages over membrane structures.
\begin{definicia}
  $A$ is {\bf CLS membrane strucure} iff:
  \begin{enumerate*}
    \item $A\in\Sigma^*$
    \item $A=B|C$ for membrane structures $B$ and $C$
    \item $A=(D)^L\rfloor C$ for membrane structure $C$ and sequence $D\in \Sigma^*$
  \end{enumerate*}
\end{definicia}

\begin{definicia}
  $G$ is {\bf BCLS grammar} such that $G=(\Sigma, S, P, \sigma)$, where:
  \begin{itemize*}
    \item $\Sigma$ is set of objects
    \item $S\subseteq\Sigma$ is set of sequence objects. They won't generate membranes nor parallel structures.
    \item $P$ is set of rewriting rules of type:
      \begin{itemize*}
        \item $s\rightarrow t$ where $s\in S$ and $t\in S^*$
        \item $a\rightarrow b|c$ where $a,b,c\in\Sigma$
        \item $a\rightarrow (s^*)^L\rfloor b$ where $a,b\in\Sigma$ and $s\in S$
      \end{itemize*}
    \item $\sigma\in\Sigma$ is start object
  \end{itemize*}
\end{definicia}

\begin{definicia}
  Step of generation is binary relation $\Rightarrow$ on set of membrane structures defined as follows:
  $M_1\Rightarrow M_2\Leftrightarrow \exists p\in P$ \dots TODO %TODO
\end{definicia}

\begin{definicia}
  Generation is sequence of membrane structures $\{M_i\}_{i=0}^\infty$ such that $\sigma\Rightarrow M_0$ and $\forall i: M_i\Rightarrow M_{i+1}$.
\end{definicia}

\begin{definicia}
  $L(G)$ (language generated by grammar $G$), is a set of membrane structures $M: \exists$ generation $\{M_i\}_{i=0}^\infty$ such that $\forall n\exists m>n: M_m=M$.
\end{definicia}

\begin{poznamka}
  Last definition says that if generation sequence contains infinite subsequence where each element equals $M$, we consider $M$ as membrane structure contained in language.
\end{poznamka}

\begin{poznamka}
  Note that a generation may have arbitrary many membrane structures that alternate so each of them will be contained in language.
\end{poznamka}

\begin{definicia}
  Buchi CLS is class of languages $BCLS=\{L|\exists grammar G: L=L(G)\}$.
\end{definicia}

\section{Normal forms}
Definition of language generated by grammar $G$ from previos section can be simplified such that two occurences of $M$ in generation are sufficient.

\section{Closure properties}
\begin{veta}
  BCLS is closed under union.
\end{veta}

\begin{dokaz}
  Suppose $L_1, L_2\in BCLS$. We want to prove that $L_1\cup L_2\in BCLS$.
  Let $G_1=(\Sigma_1, S_1, P_1, \sigma_1)$ and $G_2=(\Sigma_2, S_2, P_2, \sigma_2)$ such that $L(G_1)=L_1$ and $L(G_2)=L_2$.
  We can construct grammar $G=(\Sigma_1\cup\Sigma_2\cup\sigma, S_1\cup S_2, P_1\cup P_2 \cup \{\sigma\rightarrow\sigma_1, \sigma\rightarrow\sigma_2\}, \sigma)$ such that $\sigma\notin\Sigma_1\cup\Sigma_2$.
  
  Let's prove the inclusion $L_1\cup L_2\subseteq L(G)$:

  Suppose $M\in L_1$. The case $M\in L_2$ would be done by analogy. So there is a generation in $G_1$ that has infinite number of occurences of $M$. Grammar $G$ contains all the rewrite rules from $G_1$, so the generation appended after initial step $\sigma\Rightarrow\sigma_1$ is valid generation in $G$. Thus $M\in L(G)$.

  Now let's prove the opposite inclusion $L(G)\subseteq L_1\cup\L_2$:

  We have $M\in L(G)$. There exists generation that contains infinite occurences of $M$. Since the only rules applicable to $\sigma$ are $\sigma\rightarrow\sigma_1$ and $\sigma\rightarrow\sigma_2$. If the first applied rule was $\sigma\rightarrow\sigma_1$, rest of rules are from set $P_1$ \dots TODO %TODO BUG NEED DISJUNCTIVE OBJECTS AND RULES

\end{dokaz}

\begin{thebibliography}{1}
\end{thebibliography}

\end{document}
