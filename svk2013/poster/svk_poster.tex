% !Mode:: "TeX::UTF-8"
\documentclass[myposter,portrait]{sciposter}

%% uzitocne package
\usepackage{multicol}
\usepackage{color}
\usepackage{graphicx}

%% znaky s diakritikou
\usepackage[utf8]{inputenc}
\usepackage[T1]{fontenc}
% \usepackage[slovak]{babel} % slovenske delenie slov

%% definicia farieb
% \definecolor{mainCol}{rgb}{0.91,0.82,0.74} % farba pozadia posteru
\definecolor{mainCol}{rgb}{1,1,1} % farba pozadia posteru
\definecolor{sectionCol}{rgb}{0,0,0} % farba nadpisu
\definecolor{textCol}{rgb}{0.2,0,0} % farba hlavneho textu
\definecolor{BoxCol}{rgb}{1,1,0.8} % farba boxu okolo nadpisov

\def\mysection#1{
{\color{sectionCol}\section*{\sc\bfseries #1}}}

\begin{document}
\setlength{\logowidth}{20cm}
\setlength{\titlewidth}{\textwidth}
\addtolength{\titlewidth}{-\logowidth}
\rightlogo[0.9]{fmfilogo-farebne}
\useleftlogofalse

\color{textCol}

\title{Membrane computing}
\author{Michal Kováč\\
        Supervisor: Damas Gruska}
\institute{%
Katedra aplikovanej informatiky,
FMFI UK, Mlynská Dolina, 842~48~Bratislava\\
}
\maketitle

\begin{multicols*}{3}

\mysection{Introduction}
Well established computation models motivated by biology such as neural networks and evolutionary algorithms has already proven that it is worth to be inspired by biology. Other emerging areas are still awaiting for their more significant uses.

One of them is the membrane computing. It is relatively young field of natural computing - in comparison: neural networks have been researched since 1943 and membrane systems since 1998\cite{Paun98}.

\mysection{Membrane structure}

Biological systems usually have hierarchical structure where objects and information flows between regions, what can be interpreted as a computation process.

\includegraphics[width=\columnwidth]{membrane_structure}

\includegraphics[width=\columnwidth]{membrane_tree}

\mysection{P systems}

Membranes and regions delimited by them has clearly 1:1 correspondence. Each region contains a multiset of objects. These objects can evolve according to evolution rules which are associated with membranes. Evolution rules are applied in a maximally parallel manner - in each step, a maximal multiset of rules is nondeterministically chosen and applied.

This computing device is called P system\footnote{P in P systems stands for surname of it's founder Gheorghe P\u aun.} and is Turing complete.

Next figure demonstrates a computation of a P system for Fibonacci numbers.

\includegraphics[width=\columnwidth]{p_system_fibonacci}

\mysection{Variants}

Since the first publication in 1998, huge amount of variants has been proposed. Some of them are Turing complete, others are not. Maximal parallelismus is one of the most questioned attribute. P systems in sequential mode are not Turing complete. Various combinations of variants have been studied and some of them have been shown Turing complete also in sequential mode.

\mysection{Inhibitors}

Rewrite rules may contain inhibitors; their presence in the region prevents the application of the rule. We show that sequential P systems with inhibitors are Turing complete.


\mysection{SAT in linear time}

Linear time solutions to NP-complete problems by means of P systems are achieved by trading time (number of computation steps) for space (number of membranes and objects). This is inspired by the capabililty of cells to produce an exponential number of new membranes in linear time.

However, many simulators of P system are inefficient since they cannot handle the parallelism of these devices. Nowadays, we are witnessing the consolidation of the GPUs as a parallel framework to compute general purpose applications. The simulation of P systems with active membranes using GPUs is analysed in \cite{Cecilia10} and an efficient linear solution to the SAT problem is illustrated.

\includegraphics[width=\columnwidth]{paun}
\caption{Inventor of P systems}
 

%% zoznam literatury
\bibliographystyle{apalike}
\bibliography{references}

\end{multicols*}
\end{document}

