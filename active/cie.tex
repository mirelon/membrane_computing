\documentclass[llncs,submission,copyright,creativecommons]{../lib/lncs/llncs}
\providecommand{\event}{Cie-CS 2015} % Name of the event you are submitting to

\usepackage[utf8]{inputenc}
\usepackage{lmodern}
\usepackage[T1]{fontenc}
\usepackage{amsfonts}
\usepackage{amssymb}
\usepackage{amsmath}
\usepackage{color}

\usepackage{fontenc}
\usepackage{graphicx}
\usepackage{graphics}
\usepackage{graphicx}
\usepackage{hyperref}
\usepackage{makeidx}
\pagestyle{headings}
\bibliographystyle{../lib/lncs/splncs}

\def\eps{\varepsilon}
\def\goodgap{\hspace{\subfigcapskip}}
\renewcommand\refname{References}

% Itemize bulet types
\renewcommand{\labelitemi}{$\bullet$}
\renewcommand{\labelitemii}{$\cdot$}

% Narrow texts in boxes
\providecommand{\narrow}[1]{\scalebox{.8}[1.0]{#1}}

\begin{document}
\title{Decidability of the termination problem for sequential P systems with active membranes}
\author{Michal Kováč}
\institute{Faculty of Mathematics, Physics and Informatics, Comenius University}
\date{\today}
\maketitle

\begin{abstract}
Abstract
\end{abstract}

\section{Introduction}
\label{sec:introduction}

% Bio motivation

Membrane systems (P systems) were introduced by P\u{a}un (see \cite{Paun2000108}) as distributed parallel computing devices inspired by the structure and functionality of cells.
One of the objectives is to relax the condition of using the rules in a maximally parallel way in order to find more realistic P systems from a biological point of view.
In sequential systems, only one rewriting rule is used in each step of computation.


\section{Preliminaries}
\label{sec:preliminaries}

Here we recall several notions from the classical theory of formal languages.

An {\bf alphabet} is a finite nonempty set of symbols. Usually it is denoted by $\Sigma$. A {\bf string} over an alphabet is a finite sequence of symbols from alphabet. We denote by $\Sigma^*$ the set of all strings over an alphabet $\Sigma$. By $\Sigma^+$ = $\Sigma^* - \{\eps\}$ we denote the set of all nonempty strings over $\Sigma$. A {\bf language} over the alphabet $\Sigma$ is any subset of $\Sigma^*$.

The number of occurrences of a given symbol $a\in \Sigma$ in the string $w\in \Sigma^*$ is denoted by $|w|_a$. $\Psi_\Sigma(w)=(|w|_{a_1},|w|_{a_2},\dots,|w|_{a_n})$ is called a Parikh vector associated with the string $w\in \Sigma^*$, where $\Sigma=\{a_1,a_2,\dots a_n\}$. For a language $L\subseteq \Sigma^*$, $\Psi_\Sigma(L)=\{\Psi_\Sigma(w)|w\in L\}$ is the Parikh mapping associated with $L$. If FL is a family of languages, PsFL is denoted the family of Parikh images of languages in FL.

A multiset over a set $\Sigma$ is a mapping $M: \Sigma\rightarrow \mathbb N$.
We denote by $M(a), a\in \Sigma$ the multiplicity of $a$ in the multiset $M$.
The {\bf support} of a multiset $M$ is the set $supp(M)=\{a\in \Sigma|M(a)\geq 1\}$.
It is the set of items with at least one occurrence.
A multiset is {\bf empty} when its support is empty.
A multiset $M$ with finite support $\{a_1, a_2, \dots, a_n\}$ can be represented by the string $a_1^{M(a_1)}a_2^{M(a_2)}\dots a_n^{M(a_n)}$.
We say that multiset $M_1$ is included in multiset $M_2$ if $\forall a \in supp(M_1): M_1(a)\leq M_2(a)$.
We denote it by $M_1\subseteq M_2$.
If $M_1\subseteq M_2$, the {\bf difference} of two multisets $M_2-M_1$ is defined as a multiset where $\forall a \in supp(M_2): (M_2-M_1)(a)=\max(M_2(a)-M_1(a),0)$.
The {\bf union} of two multisets $M_1\cup M_2$ is a multiset where $\forall a \in supp(M_1)\cup supp(M_2): (M_1\cup M_2)(a)=M_1(a)+M_2(a)$.
The product of multiset $M$ with natural number $n\in \mathbb N$ is a multiset where $\forall a \in supp(M): (n\cdot M)(a)=n\cdot M(a)$.

Next, we recall notions from graph theory.

A {\bf rooted tree} is a tree, in which a particular node is distinguished from the others and called the root node.
Let $T$ be a rooted tree. We will denote its root node by $r_T$.
Let $t$ be a node of $T\setminus\{r_T\}$. The node adjacent to $t$ on the only path from $t$ to $r_T$ is called a {\bf parent node} of $t$ and is denoted by $parent(t)$.
We will denote the set of nodes of $T$ by $V(T)$ and set of its edges by $E(T)$.
Let $T_1, T_2$ be rooted trees. A bijection $f: V(T_1)\rightarrow V(T_2)$ is an {\bf isomorphism} iff $\{(f(u),f(v))|(u,v)\in E(V(T_1))\} = E(V(T_2))$. 

  
\section{Active P systems}
\label{sec:p systems}

% Membrane structure

The fundamental ingredient of a P system is the {\bf membrane structure} (see \cite{Paun2006Introduction}). It is a hierarchically arranged set of membranes, all contained in the {\bf skin membrane}. Each membrane determines a compartment, also called region, which is the space delimited from above by it and from below by the membranes placed directly inside, if any exists. Clearly, the correspondence membrane – region is one-to-one, that is why we sometimes use interchangeably these terms.
Membrane structure can be also viewed as a rooted tree with the skin membrane as the root node.

% P system

Let $\Sigma$ be a set of objects. Recall that $\mathbb N^\Sigma$ contains all multisets of objects from $\Sigma$. {\bf Membrane configuration} is a tuple $(T, l, c)$, where:
\begin{itemize}
  \item $T$ is a rooted tree,
  \item $l\in\mathbb N^{V(T)}$ is a mapping that assigns for each node of $T$ a number (label), where $l(r_T)=1$, so the skin membrane is always labeled with 1,
  \item $c\in(\mathbb N^\Sigma)^{V(T)}$ is a mapping that assigns for each node of $T$ a multiset of objects from $\Sigma$, so it represents the contents of the membrane.
\end{itemize}

{\bf Active P system} is a tuple $(\Sigma, C_0, R_1, R_2, \dots , R_m)$, where:
\begin{itemize}
  \item $\Sigma$ is a set of objects,
  \item $C_0$ is initial membrane configuration,
  \item $R_1,R_2,\dots R_m$ are finite sets of rewriting rules associated with the labels $1,2,\dots,m$ and can be of forms:
  \begin{itemize}
    \item $u\rightarrow w$, where $u\in \Sigma^+$, $w\in (\Sigma\times\{\cdot, \uparrow, \downarrow_j\})^*$ and $1\leq j\leq m$,
    \item $u\rightarrow w\delta$, where $u\in \Sigma^+$, $w\in (\Sigma\times\{\cdot, \uparrow, \downarrow_j\})^*$ and $1\leq j\leq m$,
    \item $u\rightarrow [_j v]_j$, where $u\in \Sigma^+, v\in \Sigma^*$ and $1\leq j\leq m$.
  \end{itemize}
\end{itemize}

Although rewriting rules are defined as strings, $u,v$ and $w$ represent multisets of objects from $\Sigma$. For the first two forms, each rewriting rule may specify for each objects on the right side, whether it stays in the current region (we will omit the symbol $\cdot$), moves through the membrane to the parent region ($\uparrow$)
or to a specific child region ($\downarrow_j$, where $j$ is a label of a membrane).
We denote these transfers with an arrow immediately after the symbol.
An example of such rule is the following: $abb\rightarrow ab\downarrow_2 c\uparrow c$.
Symbol $\delta$ at the end of the rule means that after the application of the rule, the membrane is dissolved and its contents (objects, child membranes) are propagated to the parent membrane.
Active P systems differs from classical (passive) P systems in ability to create new membranes by rules of the third form.

% applicable rule definition

For active P system $(\Sigma, C_0, R_1, R_2, \dots , R_m)$, configuration $C = (T, l, c)$, membrane $d\in V(T)$ with label $j = l(d)$ the rule $r\in R_j$ is {\bf applicable} iff:
\begin{itemize}
  \item $r = u\rightarrow w$ and $u\subseteq c(d)$ and $\forall (a,\downarrow_k)\in w \exists d_2\in V(T): l(d_2)=k \wedge parent(d_2) = d$,
  \item $r = u\rightarrow w\delta$ and $u\subseteq c(d)$ and $\forall (a,\downarrow_k)\in w \exists d_2\in V(T): l(d_2)=k \wedge parent(d_2) = d$ and $d\neq r_T$,
  \item $r = u\rightarrow [_k v]_k$ and $u\subseteq c(d)$.
\end{itemize}

% TODO result of the rule application

For the simplicity of proofs it is convenient to introduce a variant with a global limit upon the membrane structure. We achieve this by restricting the rule application such that if the rule would result in a structure exceeding the limit, the rule will not be applicable.

{\bf Active P system with a limit on total number of membranes} is a tuple $(\Sigma, L, C_0, R_1, R_2, \dots , R_m)$, where $(\Sigma, C_0, R_1, R_2, \dots , R_m)$ is an active P system and $L\in \mathbb N$ is a limit on total number of membranes. Anytime during the computation, a confuguration $(T, l, c)$ is not allowed to have more than $L$ membranes, so the following invariant holds: $|V(T)|\leq L$.

This is achieved by adding a constraint for rule of the form $r = u\rightarrow [_k v]_k$, which is defined to be applicable iff $u\subseteq c(d)$ and $|V(T)|<L$. If the number of membranes is equal to $L$, there is no space for newly created membrane, so in that case such rule is not applicable.

A {\bf computation step} of P system is a relation $\Rightarrow$ on the set of configurations such that $C_1 \Rightarrow C_2$ holds iff there is an applicable rule in a membrane in $C_1$ such that applying that rule would result in $C_2$.

An {\bf infinite computation} of a P system is an infinite sequence of configurations $\{C_i\}_{i=0}^\infty$, where $\forall i: C_i\Rightarrow C_{i+1}$.

A {\bf finite computation} of a P system is a finite sequence of configurations $\{C_i\}_{i=0}^n$, where $\forall i: C_i\Rightarrow C_{i+1}$.

A {\bf halting computation} of a P systems is a finite computation $\{C_i\}_{i=0}^n$, where there is no applicable rule in the last configuration $C_n$.

% Result of a computation

There are two possible ways of assigning a result of a computation:

\begin{enumerate}
    \item By considering the multiplicity of objects present in a designated membrane in a halting configuration. In this case we obtain a vector of natural numbers. We can also represent this vector as a multiset of objects or as Parikh image of a language.
    \item By concatenating the symbols which leave the system, in the order they are sent out of the skin membrane (if several symbols are expelled at the same time, then any ordering of them is considered). In this case we generate a language.
\end{enumerate}

The result of a single computation is clearly only one multiset or a string, but for one initial configuration there can be multiple possible computations. It follows from the fact that there can be more than one applicable rule in each configuration.

\subsection{Register machines} % (fold)
\label{sub:register_machines}
  As a referential universal language acceptor we will use Minsky's register machine. Such a machine runs a program consisting of numbered instructions of several simple types.

\begin{definition}
  A {\bf $n$-register machine} is a tuple $M = (n,P,i,h)$, where:
  \begin{itemize}
    \item $n$ is the number of registers,
    \item $P$ is a set of labeled instructions of the form $j : (op(r),k,l)$, where $op(r)$ is an operation on register $r$ of $M$, and $j$, $k$, $l$ are labels from the set $Lab(M)$ (which numbers the instructions in a one-to-one manner),
    \item $i$ is the initial label, and
    \item $h$ is the final label.
  \end{itemize}
\end{definition}

The machine is capable of the following instructions:
\begin{itemize}
\item $(add(r),k,l)$ : Add one to the contents of register $r$ and proceed to instruction $k$ or to instruction $l$; in the deterministic variants usually considered in the literature we demand $k = l$.
\item $(sub(r),k,l)$ : If register $r$ is not empty, then subtract one from its contents and go to instruction $k$, otherwise proceed to instruction $l$.
\item $halt$ : This instruction stops the machine. This additional instruction can only be assigned to the final label $h$.
\end{itemize}

A deterministic $m$-register machine can analyze an input $(n_1,\dots,n_m)\in N_0^m$ in registers 1 to $m$, which is recognized if the register machine finally stops by the halt instruction with all its registers being empty (this last requirement is not necessary). If the machine does not halt, the analysis was not successful.
  
% subsection register_machines (end)

\section{Termination problems} % (fold)
\label{sec:termination_problems}

In this section we recall the halting problem for Turing machines. The problem is to determine, given a deterministic Turing machine and an input, whether the machine running on that input will halt. It is one of the first known undecidable problems. On the other hand, for non-deterministic machines, there are two possible meanings for halting. We could be interested either in:
\begin{itemize}
  \item whether there exists an infinite computation (the machine can run forever), or
  \item whether there exists a finite computation (the machine can halt)
\end{itemize}

We will prove the (un)decidability of these problems on active P systems with limit on total number of membranes. The results are quite interesting, because:

\begin{theorem}
  Active P systems with limit on total number of membranes are universal.
\end{theorem}

\begin{proof}
  The proof of this theorem for active P system in \cite{Ibarra05Active} uses simulation of register machines and during the simulation, every configuration has at most three membranes, hence the universality holds also for active P systems with limit on total number of membranes.
\end{proof}

\subsubsection{Existence of infinite computation} % (fold)
\label{ssub:existence_of_infinite_computation}
We will propose an algorithm for deciding existence of infinite computation. Basic idea is to consider the minimal coverability graph (\cite{Rozenberg93MinimalCoverabilityGraph}), where nodes are configurations and edge leads from the configuration $C_1$ to the configuration $C_2$, when there is a rule applicable in $C_1$, which results in $C_2$. The construction in \cite{Rozenberg93MinimalCoverabilityGraph} is performed on Petri nets, where the configuration consists just of a vector of natural numbers. The situation is the same for single-membrane sequential P systems. We need to modify the construction for active P systems.

% TODO define covering order
A configuration $C_1$ covers configuration $C_2$, iff ...

% TODO state the rule applicability lemma
\begin{lemma}
  $C_1 \geq C_2$ with an isomorphism $f$, and
\end{lemma}

\begin{theorem}
  Existence of infinite computation for active P systems with limit on total number of membranes is decidable.
\end{theorem}

\begin{proof}
  From Dickson's lemma [citation needed] or from Karp-Miller reachability theorem [citation needed] implies that for a every inifite sequence of configurations there are two such that one covers the other.
\end{proof}

% subsubsection existence_of_infinite_computation (end)

\subsubsection{Existence of finite computation} % (fold)
\label{ssub:existence_of_finite_computation}

\begin{theorem}
  Existence of finite computation for active P systems with limit on total number of membranes is undecidable.
\end{theorem}

\begin{proof}
  Either reduce it to detection of empty language or to reachability.
\end{proof}

% subsubsection existence_of_finite_computation (end)

% section termination_problems (end)

\section{Conclusion}
\label{sec:conclusion}
We have studied \dots

\bibliography{cie}

\end{document}
