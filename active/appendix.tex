\section*{Appendix} % (fold)
\label{sec:appendix}

Now we will prove that $C$ is reachable for $\Pi$ if and only if there is a halting computation of $\Pi^\prime$. The first implication is the easy one.

\begin{lemma}
\label{if_reachable_then_halting_lemma}
  If $C$ is reachable for $\Pi$ then there is a halting computation of $\Pi^\prime$.
\end{lemma}

\begin{proof}
  Consider a computation $\Pi$ with $C$ as the last configuration. For $\Pi^\prime$ there is a corresponding computation as the rules of $\Pi$ are included in $\Pi^\prime$ with an exception of the rule $u\rightarrow [_k v]_k$, which has a corresponding rule with the same left side: $u\rightarrow [_k v\omega]_k$. The corresponding computation of $\Pi^\prime$ will result in a configuration, where in every membrane $d$, the contents will be $\omega c(d)$, and the skin membrane will contain an additional $\xi_{r_T}$. Then the cascade of applications of the rule $\xi_d\omega c(d)\rightarrow\sigma\xi_{d^\prime}\downarrow_{l(d^\prime)}$ can start in the skin membrane, cascading down from parents to children until all the membranes applied that rule. The objects $\omega c(d)$ will be replaced by $\sigma$ and the computation will halt. \qed
\end{proof}

The other direction of the implication is more complicated, so the following lemmas will state some properties about computations of $\Pi$ and $\Pi^\prime$.

\begin{lemma}
\label{no_dissolving_after_check_lemma}
  For all halting computations of $\Pi^\prime$ there is no rule of form $u\rightarrow w\delta$ applied in the membrane $d^\prime$ after the application of rule $\xi_d\omega c(d)\rightarrow\sigma\xi_{d^\prime}\downarrow_{l(d^\prime)}$ in membrane $d=parent(d^\prime)$.
\end{lemma}

\begin{proof}
  After the application of the rule $\xi_d\omega c(d)\rightarrow\sigma\xi_{d^\prime}\downarrow_{l(d^\prime)}$, the object $\sigma$ remains in the membrane. There is no possible interaction of $\sigma$ with other objects, only with another $\sigma$. If a child membrane $d^\prime$ is dissolved with a rule $u\rightarrow w\delta$, then two objects $\sigma$ will meet in the membrane $d$ and the computation will not halt, which contradicts the fact that the computation is halting. \qed
\end{proof}

\begin{lemma}
\label{no_creating_new_membrane_after_check_lemma}
  For all halting computations of $\Pi^\prime$ there is no rule of form $u\rightarrow [_k v\omega]_k$ applied in the membrane $d^\prime$ after the application of rule $\xi_d\omega c(d)\rightarrow\sigma\xi_{d^\prime}\downarrow_{l(d^\prime)}$.
\end{lemma}

\begin{proof}
  After the application of the rule $r = \xi_d\omega c(d)\rightarrow\sigma\xi_{d^\prime}\downarrow_{l(d^\prime)}$ in membrane $d$, there will not be any of $\xi$ objects present in the membrane, because they are only sent into child membranes, which cannot be dissolved because of lemma \ref{no_dissolving_after_check_lemma}. So there will be no more of rule $r$ applied in membrane $d$. The newly created membrane will never receive any of $\xi$ objects, so the object $\omega$ will never be erased and the computation will not halt, which contradicts the fact that the computation is halting. \qed
\end{proof}

\begin{lemma}
\label{check_at_last_lemma}
  If a halting computation of $\Pi^\prime$ exists then there is a halting computation, where for every membrane $d\in V(T)$ in the target configuration $C=(T,l,c)$ the last rule used is $r = \xi_d\omega c(d)\rightarrow\sigma\xi_{d^\prime}\downarrow_{l(d^\prime)}$.
\end{lemma}

\begin{proof}
  Consider membrane $d$: let $C_1$ be the configuration before the application of $r$, $C_2$ the configuration after the application of $r$ and $C_3$ the halting configuration.

  First consider elementary membranes, let such a membrane be $d$. As it has no child membranes, $C_2$ is contained within $C_1$ (with the exception of the $\sigma$ object). Therefore the sequence of steps from $C_2$ to $C_3$ can be applied starting from $C_1$ instead of $C_2$.
  It would result in a configuration $C_4$, where $d$ contains exactly the objects $\xi_d\omega c(d)$, assuming $d$ has not been dissolved, what is ensured by lemma \ref{no_dissolving_after_check_lemma}. So the rule $r = \xi_d\omega c(d)\rightarrow\sigma$ can be applied, resulting in configuration, where $d$ contains exactly one $\sigma$ and nothing else, so no more rule will be applied in $d$.

  Consider a membrane $d$, which is not elementary, so it has child membranes. In $C_1$ the verifier is in $d$ and no child membrane contains the verifier. In $C_2$ every child membrane contains the verifier. We cannot use the same argument, because $C_2$ is not contained within $C_1$.

  We will proceed by induction, starting from the elementary membranes to parent membranes. Assume the lemma holds for child membranes. Then we have a computation, which contains a configuration $C_5$, after which only verification rules are applied in child membranes. So the sequence of steps from $C_2$ to $C_5$ does not contain the verification rule in child membranes. Hence this sequence can be used starting from $C_1$ instead of $C_2$, resulting in a configuration $C_6$, where application of verification rule in $d$ results in $C_5$. So again, after $C_6$ only verification rules are applied.
  \qed
\end{proof}

\begin{lemma}
\label{if_halting_then_reachable_lemma}
  If there is a halting computation of $\Pi^\prime$ then $C$ is reachable for $\Pi$.
\end{lemma}

\begin{proof}
  According to lemma \ref{check_at_last_lemma} there is also a halting computation where in every membrane $d$ the last used rule is $r = \xi_d\omega c(d)\rightarrow\sigma\xi_{d^\prime}\downarrow_{l(d^\prime)}$. So the corresponding computation in $\Pi$ will result in the configuration $C_5$, where every membrane $d$ contains exactly $c(d)$. If $C_5$ contains a membrane not present in $C$, it will contain the object $\omega$, which will not be reached by any of objects $\xi$ so the computation will not halt. If $C$ contains a membrane $d^\prime$ not present in $C_5$, then the membrane $d = parent(d^\prime)$ will never get rid of $\omega$ because the rule $r = \xi_d\omega c(d)\rightarrow\sigma\xi_{d^\prime}\downarrow_{l(d^\prime)}$ cannot be applied due to lack of the child membrane $d^\prime$. Hence $C = C_5$ and there is a computation of $\Pi$ that will result in $C$. \qed
\end{proof}

\existenceofhaltingtheorem*

\begin{proof}
  For a given P system $\Pi$ and a target configuration $C$ we have constructed a P system $\Pi^\prime$ such that there is a halting computation of $\Pi^\prime$ iff the $\Pi$ can reach configuration $C$. The two directions of the equivalence have been proven in lemmas \ref{if_reachable_then_halting_lemma} and \ref{if_halting_then_reachable_lemma}. Using this construction, we can reduce the existence of halting computation to reachability of register machines \cite{Ibarra05Active}, which is known to be undecidable. \qed
\end{proof}

% section appendix (end)