Let $\Sigma$ be a set of objects. Recall that $\mathbb N^\Sigma$ contains all multisets of objects from $\Sigma$. {\bf Membrane configuration} is a tuple $(T, l, c)$, where:
\begin{itemize}
  \item $T$ is a rooted tree,
  \item $l\in\mathbb N^{V(T)}$ is a mapping that assigns for each node of $T$ a number (label), where $l(r_T)=1$, so the skin membrane is always labeled with 1,
  \item $c\in(\mathbb N^\Sigma)^{V(T)}$ is a mapping that assigns for each node of $T$ a multiset of objects from $\Sigma$, so it represents the contents of the membrane.
\end{itemize}

{\bf Active P system} is a tuple $(\Sigma, C_0, R_1, R_2, \dots , R_m)$, where:
\begin{itemize}
  \item $\Sigma$ is a set of objects,
  \item $C_0$ is initial membrane configuration,
  \item $R_1,R_2,\dots R_m$ are finite sets of rewriting rules associated with the labels $1,2,\dots,m$ and can be of forms:
  \begin{itemize}
    \item $u\rightarrow w$, where $u\in \Sigma^+$, $w\in (\Sigma\times\{\cdot, \uparrow, \downarrow_j\})^*$ and $1\leq j\leq m$,
    \item $u\rightarrow w\delta$, where $u\in \Sigma^+$, $w\in (\Sigma\times\{\cdot, \uparrow, \downarrow_j\})^*$ and $1\leq j\leq m$,
    \item $u\rightarrow [_j v]_j$, where $u\in \Sigma^+, v\in \Sigma^*$ and $1\leq j\leq m$.
  \end{itemize}
\end{itemize}

Although rewriting rules are defined as strings, $u,v$ and $w$ represent multisets of objects from $\Sigma$. For the first two forms, each rewriting rule may specify for each object on the right side, whether it stays in the current region (we will omit the symbol $\cdot$), moves through the membrane to the parent region ($\uparrow$)
or to a specific child region ($\downarrow_j$, where $j$ is a label of a membrane). If there are more child membranes with the same label, one is chosen nondeterministically.
We denote these transfers with an arrow immediately after the symbol.
An example of such rule is the following: $abb\rightarrow ab\downarrow_2 c\uparrow c\delta$.
Symbol $\delta$ at the end of the rule means that after the application of the rule, the membrane is dissolved and its contents (objects, child membranes) are propagated to the parent membrane.
Active P systems differs from classic (passive) P systems in ability to create new membranes by rules of the third form. Such rule will create new child membrane with a given label $j$ and a given multiset of objects $v$ as its contents.

% applicable rule definition

For active P system $(\Sigma, C_0, R_1, R_2, \dots , R_m)$, configuration $C = (T, l, c)$, membrane $d\in V(T)$ the rule $r\in R_{l(d)}$ is {\bf applicable} iff:
\begin{itemize}
  \item $r = u\rightarrow w$ and $u\subseteq c(d)$ and $\forall (a,\downarrow_k)\in w \exists d_2\in V(T): l(d_2)=k \wedge parent(d_2) = d$,
  \item $r = u\rightarrow w\delta$ and $u\subseteq c(d)$ and $\forall (a,\downarrow_k)\in w \exists d_2\in V(T): l(d_2)=k \wedge parent(d_2) = d$ and $d\neq r_T$,
  \item $r = u\rightarrow [_j v]_j$ and $u\subseteq c(d)$.
\end{itemize}

In this paper we assume only sequential systems, so in each step of the computation, there is one rule nondeterministically chosen among all applicable rules in all membranes to be applied.

% Computation step

A {\bf computation step} of P system is a relation $\Rightarrow$ on the set of configurations such that $C_1 \Rightarrow C_2$ holds iff there is an applicable rule in a membrane in $C_1$ such that applying that rule can result in $C_2$.

An {\bf infinite computation} of a P system is an infinite sequence of configurations $\{C_i\}_{i=0}^\infty$, where $\forall i: C_i\Rightarrow C_{i+1}$.

A {\bf finite computation} of a P system is a finite sequence of configurations $\{C_i\}_{i=0}^n$, where $\forall i: C_i\Rightarrow C_{i+1}$.

A {\bf halting computation} of a P systems is a finite computation $\{C_i\}_{i=0}^n$, where there is no applicable rule in the last configuration $C_n$.

% Result of a computation

The P system can work in generating or in accepting mode. For the generating mode there are two possible ways of assigning a result of a computation:

\begin{enumerate}
    \item By considering the multiplicity of objects present in a designated membrane in a halting configuration. In this case we obtain a vector of natural numbers. We can also represent this vector as a multiset of objects or as Parikh image of a language.
    \item By concatenating the objects which leave the system, in the order they are sent out of the skin membrane (if several symbols are expelled at the same time, then any ordering of them is considered). In this case we generate a language.
\end{enumerate}

The result of a single computation is clearly only one multiset or a string, but for one initial configuration there can be multiple possible computations. It follows from the fact that there can be more than one applicable rule in each configuration and they are chosen nondeterministically.
For the accepting mode the input multiset is inserted in the skin membrane and it is accepted if and only if a given accepting configuration can be reached\cite{Ibarra05Active}.

% Global limit

For the simplicity of proofs it is convenient to introduce a variant with a global limit upon the membrane structure. We achieve this by restricting the rule application such that if the rule would result in a structure exceeding the limit, the rule will not be applicable.

{\bf Active P system with a limit on total number of membranes} is a tuple $(\Sigma, L, C_0, R_1, R_2, \dots , R_m)$, where $(\Sigma, C_0, R_1, R_2, \dots , R_m)$ is an active P system and $L\in \mathbb N$ is a limit on total number of membranes. Anytime during the computation, a confuguration $(T, l, c)$ is not allowed to have more than $L$ membranes, so the following invariant holds: $|V(T)|\leq L$.

This is achieved by adding a constraint for rule of the form $r = u\rightarrow [_k v]_k$, which is defined to be applicable iff $u\subseteq c(d)$ and $|V(T)|<L$. If the number of membranes is equal to $L$, there is no space for newly created membrane, so in that case such rule is not applicable.