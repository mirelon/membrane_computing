{\bf P system with active objects} is a tuple $(V, \mu, R_1, R_2, \dots , R_m)$, where:
\begin{itemize}
  \item $V$ is the alphabet of symbols,
  \item $\mu$ is initial membrane structure containing multisets of objects from $V$, defined recursively as:
  \begin{align*}
    &[_1 M]_1 \\
    &M ::= \eps | M a | M [_j M]_j\text{, where }a\in V\text{ and } 1\leq j\leq m
  \end{align*}
  \item $R_1,R_2,\dots R_m$ are finite sets of rewriting rules associated with the membranes labeled $1,2,\dots,m$ and can be of forms:
  \begin{itemize}
    \item $u\rightarrow v$, where $u\in V^+$, $v\in (V\times\{\cdot, \uparrow, \downarrow_j\})$ and $1\leq j\leq m$
    \item $u\rightarrow v\delta$, where $u\in V^+$, $v\in (V\times\{\cdot, \uparrow, \downarrow_j\})$ and $1\leq j\leq m$
    \item $u\rightarrow [_j w]_j$, where $w\in V^*$ and $1\leq j\leq m$
  \end{itemize}
\end{itemize}

Each rewriting rule may specify for each symbol on the right side, whether it stays in the current region (we will omit the symbol $\cdot$), moves through the membrane to the parent region ($\uparrow$)
or to a specific child region ($\downarrow_m$, where $m$ is a label of a membrane).
We denote these transfers with arrows immediately after the symbol.
An example of such rule is the following: $abb\rightarrow ab\downarrow_2 c\uparrow c$.

Symbol $\delta$ at the end of the rule means that after application of the rule, the membrane is dissolved and it's contents (objects, child membranes) are propagated to the parent membrane.

A {\bf configuration} of a P system is represented by its membrane structure and the multisets of objects in the regions.

A {\bf computation step} of P system is a relation $\Rightarrow$ on the set of configurations such that $C_1 \Rightarrow C_2$ iff there is an applicable rule in a membrane in $C_1$ such that applying that rule would result in $C_2$.

A {\bf computation} of a P system consists of a sequence of steps. The step $S_i$ is applied to result of previous step $S_{i-1}$. So when $S_i = (C_j,C_{j+1})$, $S_{i-1} = (C_{j-1},C_j)$.

% Result of a computation

There are two possible ways of assigning a result of a computation:

\begin{enumerate}
    \item By considering the multiplicity of objects present in a designated membrane in a halting configuration. In this case we obtain a vector of natural numbers. We can also represent this vector as a multiset of objects or as Parikh image of a language.
    \item By concatenating the symbols which leave the system, in the order they are sent out of the skin membrane (if several symbols are expelled at the same time, then any ordering of them is considered). In this case we generate a language.
\end{enumerate}

The result of a computation is clearly only one multiset or a string, but for one initial configuration there can be multiple possible computations. It follows from the fact that there exist more than one maximal multiset of rules that can be applied in each step.